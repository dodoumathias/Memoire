\phantomsection
\addcontentsline{toc}{section}{Introduction}
\section*{Introduction}

La déclaration de vol de véhicule constitue une étape essentielle pour engager les poursuites judiciaires et faciliter la recherche du véhicule. 
Plusieurs plateformes numériques et services administratifs accompagnent les victimes dans cette démarche. 
Ce chapitre présente les concepts fondamentaux liés à la gestion et au suivi des véhicules volés, analyse les solutions existantes et met en évidence leurs limites afin de justifier la solution proposée.

%====================================================
\section{Généralités sur la gestion et le suivi des vols de véhicules}

\subsection{Définition du suivi des véhicules volés}

Le suivi des véhicules volés désigne l’ensemble des mécanismes organisationnels et numériques permettant d’enregistrer, surveiller et mettre à jour l’état d’un véhicule déclaré volé jusqu’à sa récupération ou sa régularisation.

\subsection{Importance du suivi}

Le suivi constitue un élément central dans la lutte contre la criminalité automobile. Il permet la centralisation des informations, la coordination des interventions et la réduction de la circulation illégale des véhicules.

Un suivi structuré améliore la continuité des investigations, réduit les délais de réaction et augmente les chances de récupération des véhicules. Il renforce également la confiance des citoyens envers les institutions en évitant le sentiment d’abandon après la déclaration.

\subsection{Processus de déclaration et de gestion}

Le processus débute par une déclaration officielle auprès des autorités compétentes. Cette déclaration entraîne l’ouverture d’un dossier administratif et judiciaire contenant les informations essentielles du véhicule et les circonstances du vol.

Le dossier suit ensuite plusieurs étapes : vérification des informations, diffusion du signalement, actions de recherche et, le cas échéant, clôture. Dans la plupart des systèmes actuels, ce processus reste interne aux institutions et offre peu de visibilité au citoyen.

\subsection{Composantes clés d’un système de gestion}

Les systèmes modernes reposent sur :

\begin{itemize}
    \item l’enregistrement structuré des déclarations ;
    \item la centralisation des données ;
    \item la mise à jour régulière des statuts ;
    \item la communication entre acteurs ;
    \item la conservation de l’historique des événements.
\end{itemize}

\subsection{Défis liés au suivi}

Malgré les dispositifs existants, plusieurs difficultés persistent :

\begin{itemize}
    \item lenteur des procédures ;
    \item manque de visibilité pour les citoyens ;
    \item faible coordination entre plateformes ;
    \item absence de suivi en temps réel accessible au public.
\end{itemize}

%====================================================
\section{Étude de l’existant}

\subsection{Plateformes institutionnelles nationales}

\subsubsection{DGPR – Bénin}

La plateforme de la DGPR permet la déclaration en ligne d’un vol de véhicule \cite{dgpr2023}. 
Toutefois, elle ne propose ni tableau de bord de suivi en temps réel ni notifications automatiques.

\textbf{Limite principale :} absence de suivi dynamique post-déclaration.

\subsubsection{Service Public France}

La plateforme permet la déclaration en ligne et la génération de documents administratifs \cite{servicepublic2023}. 
Le suivi reste cependant administratif et non interactif.

\textbf{Limite principale :} interaction limitée aux démarches administratives.

\subsection{Systèmes institutionnels internationaux}

\subsubsection{Interpol – SMV}

Base mondiale réservée aux forces de l’ordre pour l’identification des véhicules volés \cite{interpol2023}.

\textbf{Limite principale :} aucune implication directe des citoyens.

\subsubsection{Europol – SIS II}

Système sécurisé de coopération transfrontalière entre États membres \cite{europol2023}.

\textbf{Limite principale :} accès strictement institutionnel.

\subsection{Plateformes privées}

\subsubsection{DIGITPOL}

Base internationale accessible aux particuliers \cite{digitpol2023}. 
Cependant, elle ne possède pas de reconnaissance légale officielle et n’est pas intégrée aux systèmes nationaux.

\textbf{Limite principale :} absence d’intégration institutionnelle.

%====================================================
\section{Analyse comparative des plateformes existantes}

L’analyse comparative repose sur les critères suivants :

\begin{itemize}
    \item déclaration citoyenne ;
    \item suivi du dossier ;
    \item interaction citoyen--autorité ;
    \item collaboration communautaire ;
    \item suivi en temps réel ;
    \item notifications automatiques ;
    \item portée géographique ;
    \item statut légal.
\end{itemize}

% Insérer ici le tableau comparatif principal

%====================================================
\section{Analyse transversale}

L’étude met en évidence une fragmentation des rôles :

\begin{itemize}
    \item Les plateformes institutionnelles privilégient la sécurité et la conformité légale.
    \item Les plateformes ouvertes privilégient l’accessibilité mais manquent de reconnaissance juridique.
\end{itemize}

Aucune plateforme ne combine simultanément :

\begin{itemize}
    \item déclaration officielle ;
    \item suivi en temps réel ;
    \item collaboration citoyenne active ;
    \item intégration directe avec les forces de l’ordre.
\end{itemize}

%====================================================
\section{Limites des systèmes actuels}

\subsection{Faible collaboration citoyenne}

Le rôle du citoyen se limite généralement à la déclaration initiale.

\subsection{Manque de visibilité}

L’absence de notifications et d’indicateurs clairs réduit la transparence du processus.

\subsection{Absence d’intégration des données}

Les systèmes fonctionnent en silos, limitant l’efficacité coordonnée à l’échelle nationale et internationale.

%====================================================
\section{Synthèse et positionnement du travail}

L’analyse de l’existant révèle l’absence d’une plateforme hybride combinant déclaration officielle, interaction continue, suivi en temps réel et collaboration sécurisée.

%====================================================
\section{Notre solution}

La solution proposée vise à :

\begin{itemize}
    \item permettre la déclaration officielle en ligne ;
    \item offrir un suivi en temps réel accessible aux citoyens ;
    \item intégrer des notifications automatiques ;
    \item favoriser la collaboration citoyenne sécurisée ;
    \item assurer l’intégration avec les autorités compétentes.
\end{itemize}

%====================================================
\section{Conclusion}

Le vol de véhicules demeure un défi majeur de sécurité publique nécessitant une approche intégrée. 
L’analyse des plateformes existantes met en évidence un manque de collaboration dynamique et de suivi interactif. 
Ces constats justifient le développement d’une solution collaborative et innovante, adaptée aux exigences institutionnelles et aux attentes des citoyens.












\chapter{Revue de littérature et analyse des solutions existantes}

%====================================================
\section{Introduction}

\subsection{Contexte général du vol de véhicules}

Le vol de véhicules constitue une problématique majeure de sécurité publique dans de nombreux pays. 
L’évolution des réseaux criminels organisés et la mobilité rapide des véhicules volés rendent les méthodes traditionnelles de gestion et de recherche moins efficaces. 
Face à cette situation, les systèmes numériques apparaissent comme des outils stratégiques pour améliorer la coordination entre citoyens et institutions.

\subsection{Problématique}

Malgré l’existence de plateformes nationales et internationales dédiées à la déclaration et au signalement des véhicules volés, 
le suivi des dossiers demeure fragmenté et peu interactif. 
Les citoyens jouent un rôle essentiellement passif après la déclaration initiale, 
et les systèmes existants ne favorisent ni la collaboration continue ni le suivi en temps réel accessible au public.

\subsection{Objectif de la revue de littérature}

Cette revue de littérature vise à analyser les concepts fondamentaux liés à la gestion et au suivi des véhicules volés, 
à étudier les solutions existantes et à identifier les limites justifiant la mise en place d’une plateforme collaborative intégrée.

%====================================================
\section{Généralités sur la gestion et le suivi des vols de véhicules}

\subsection{Définition et concepts fondamentaux}

\subsubsection{Définition du suivi des véhicules volés}

Le suivi des véhicules volés désigne l’ensemble des mécanismes organisationnels et numériques permettant d’enregistrer, 
de surveiller et de mettre à jour l’état d’un véhicule déclaré volé jusqu’à sa récupération ou sa régularisation.

\subsubsection{Notion de gestion administrative et judiciaire}

La gestion d’un vol de véhicule implique une dimension administrative (enregistrement, formalités, assurances) 
et une dimension judiciaire (enquête, poursuites, décisions juridiques).

\subsection{Importance du suivi des véhicules volés}

\subsubsection{Impact sur la lutte contre la criminalité}

Un système structuré améliore la coordination des investigations et augmente les probabilités de récupération des véhicules.

\subsubsection{Importance pour la coordination institutionnelle}

La centralisation des données facilite la communication entre services compétents et renforce l’efficacité opérationnelle.

\subsubsection{Impact sur la confiance citoyenne}

Un suivi transparent et accessible renforce la confiance des citoyens envers les institutions publiques.

\subsection{Processus de déclaration et de gestion}

\subsubsection{Déclaration officielle}

Le processus débute par une déclaration formelle auprès des autorités compétentes.

\subsubsection{Ouverture de dossier}

Cette déclaration entraîne l’enregistrement des informations essentielles dans un dossier administratif et judiciaire.

\subsubsection{Étapes de traitement}

Le dossier suit plusieurs phases : vérification, diffusion du signalement, recherche et éventuelle clôture.

\subsubsection{Limites du modèle linéaire actuel}

Le modèle actuel reste principalement interne aux institutions et offre peu de visibilité aux citoyens.

\subsection{Composantes clés d’un système de gestion}

\begin{itemize}
    \item Enregistrement structuré des déclarations ;
    \item Centralisation des données ;
    \item Mise à jour régulière des statuts ;
    \item Communication entre acteurs ;
    \item Conservation de l’historique des événements.
\end{itemize}

\subsection{Défis liés au suivi}

\begin{itemize}
    \item Lenteur des procédures ;
    \item Manque de visibilité pour les citoyens ;
    \item Faible coordination entre plateformes ;
    \item Absence de suivi en temps réel.
\end{itemize}

%====================================================
\section{Étude de l’existant}

\subsection{Plateformes institutionnelles nationales}

\subsubsection{DGPR (Bénin)}
Plateforme permettant la déclaration officielle en ligne, avec des limites en matière de suivi interactif.

\subsubsection{Service Public France}
Système dématérialisé facilitant la déclaration administrative mais offrant un suivi limité.

\subsubsection{Collectivité de Saint-Martin}
Plateforme locale similaire au modèle français, avec une portée restreinte.

\subsection{Systèmes institutionnels internationaux}

\subsubsection{Interpol (SMV)}
Base mondiale destinée aux forces de l’ordre pour l’identification des véhicules volés.

\subsubsection{Europol (SIS II)}
Système européen sécurisé favorisant la coopération transfrontalière.

\subsection{Plateformes privées}

\subsubsection{DIGITPOL}
Base internationale accessible au public, sans reconnaissance juridique officielle intégrée aux systèmes nationaux.

%====================================================
\section{Analyse comparative des solutions existantes}

\subsection{Critères d’analyse}

\begin{itemize}
    \item Déclaration ;
    \item Suivi ;
    \item Interaction ;
    \item Collaboration ;
    \item Temps réel ;
    \item Notifications ;
    \item Portée géographique ;
    \item Statut légal.
\end{itemize}

\subsection{Tableau comparatif général}

% Insérer ici le premier tableau comparatif

\subsection{Tableau orienté collaboration citoyenne}

% Insérer ici le second tableau comparatif

%====================================================
\section{Analyse transversale}

\subsection{Interaction citoyenne}

Dans la majorité des systèmes, le rôle du citoyen se limite à la déclaration initiale, sans mécanisme de participation continue.

\subsection{Suivi en temps réel}

Le suivi en temps réel est généralement réservé aux autorités et rarement accessible aux citoyens.

\subsection{Collaboration et intelligence collective}

Les systèmes publics et privés fonctionnent de manière cloisonnée, limitant l’exploitation de l’intelligence collective.

\subsection{Limites globales identifiées}

\begin{itemize}
    \item Faible collaboration citoyenne ;
    \item Manque de visibilité ;
    \item Absence d’intégration des données ;
    \item Systèmes cloisonnés.
\end{itemize}

%====================================================
\section{Synthèse et positionnement du travail}

\subsection{Constats majeurs}

\begin{itemize}
    \item Fragmentation des systèmes ;
    \item Rupture entre institutions et citoyens ;
    \item Absence de plateforme hybride intégrée.
\end{itemize}

\subsection{Justification de la solution proposée}

\begin{itemize}
    \item Besoin d’un système collaboratif ;
    \item Nécessité d’un suivi en temps réel ;
    \item Intégration citoyen--autorité ;
    \item Adaptation au contexte africain.
\end{itemize}

%====================================================
\section{Présentation de notre solution}

\subsection{Vision générale}

La solution proposée repose sur une plateforme intégrée combinant déclaration officielle, suivi dynamique et collaboration sécurisée.

\subsection{Fonctionnalités principales}

\begin{itemize}
    \item Déclaration officielle en ligne ;
    \item Suivi en temps réel ;
    \item Notifications automatiques ;
    \item Signalement citoyen encadré ;
    \item Intégration avec les autorités compétentes.
\end{itemize}

\subsection{Valeur ajoutée}

La plateforme concilie sécurité institutionnelle et participation citoyenne active.

\subsection{Différenciation par rapport à l’existant}

Elle combine déclaration légale, suivi interactif et collaboration communautaire dans un système unifié.

%====================================================
\section{Conclusion}

Cette revue a mis en évidence les limites des systèmes actuels, notamment la fragmentation, le manque de visibilité et la faible collaboration citoyenne. 
Ces constats justifient le développement d’une plateforme intégrée favorisant le suivi en temps réel et l’interaction continue entre citoyens et autorités.






























\chapter{Revue de littérature et analyse de l’existant}

% \phantomsection
% \addcontentsline{toc}{section}{Introduction}
\section*{Introduction}

La déclaration de vol de véhicule est une étape essentielle pour engager les poursuites judiciaires et faciliter la recherche du véhicule.  
Plusieurs plateformes numériques et services administratifs existent pour accompagner les victimes dans cette démarche.  
Ce document présente une analyse détaillée des principaux sites utilisés pour la déclaration de vol de véhicule.


\section{Généralité sur la gestion et le suivi des vols de véhicules}

\subsection{Définition et importance du suivi des véhicules volés}

Le suivi des véhicules volés désigne l’ensemble des mécanismes organisationnels et numériques permettant d’enregistrer, surveiller et mettre à jour l’état d’un véhicule déclaré volé jusqu’à sa récupération ou sa régularisation.  
Il constitue un élément central dans la lutte contre la criminalité automobile, car il permet de centraliser les informations, de coordonner les interventions et de limiter la circulation illégale des véhicules.

Un suivi efficace contribue également à rassurer les citoyens, à faciliter les procédures administratives et à renforcer la confiance envers les institutions chargées de la sécurité.



\subsection{Importance du suivi des véhicules volés}

Le suivi des véhicules volés constitue un élément fondamental dans la lutte contre la criminalité automobile. Une fois le vol déclaré, la capacité à suivre l’évolution du dossier influence directement l’efficacité des actions menées par les forces de l’ordre ainsi que la perception de confiance des citoyens envers les institutions publiques.

Un suivi structuré permet de centraliser les informations relatives au véhicule, d’assurer la continuité des investigations et de faciliter la communication entre les différents acteurs impliqués. Il contribue également à réduire les délais de réaction, à éviter les doublons d’informations et à améliorer les chances de récupération des véhicules volés.

Du point de vue du citoyen, le suivi du dossier représente un facteur clé de satisfaction. L’absence d’informations après la déclaration crée un sentiment d’abandon et d’impuissance, tandis qu’un suivi clair et régulier renforce l’implication et la coopération avec les autorités.

\subsection{Processus de déclaration et de gestion}

Le processus de déclaration et de gestion d’un vol de véhicule débute généralement par une déclaration officielle effectuée par le propriétaire auprès des services compétents. Cette déclaration donne lieu à l’ouverture d’un dossier administratif et judiciaire, dans lequel sont enregistrées les informations essentielles du véhicule et les circonstances du vol.

Après la déclaration, le dossier suit un cycle de gestion comprenant plusieurs étapes, notamment la vérification des informations, la diffusion interne du signalement, les actions de recherche et, le cas échéant, la clôture du dossier.  
Dans de nombreux systèmes existants, ce processus reste largement interne aux institutions, avec peu de visibilité offerte au citoyen sur l’état d’avancement du dossier.

Cette approche linéaire et centralisée limite la participation active des citoyens et réduit les possibilités de contribution collective à la recherche des véhicules volés.

---


\subsection{Composantes clés des systèmes de gestion des vols de véhicules}

Les systèmes modernes de gestion des vols de véhicules reposent sur plusieurs composantes essentielles :
\begin{itemize}
  \item l’enregistrement structuré des déclarations de vol ;
  \item la centralisation des informations relatives aux véhicules ;
  \item la mise à jour régulière du statut des dossiers ;
  \item la communication entre les acteurs impliqués ;
  \item la conservation historique des événements liés au véhicule.
\end{itemize}

L’efficacité du système dépend fortement de la capacité de ces composantes à fonctionner de manière intégrée.

\subsection{Défis liés au suivi des vols de véhicules}

Malgré les efforts déployés, plusieurs défis persistent dans la gestion des vols de véhicules. Parmi les plus importants figurent la lenteur des procédures, le manque de visibilité pour les citoyens après la déclaration initiale, la faible coordination entre les plateformes et l’absence de suivi en temps réel accessible au public.  
Ces difficultés limitent la réactivité des acteurs et réduisent les chances de récupération rapide des véhicules.

\section{Étude de l’existant}



\subsection{Plateforme de la Direction Générale de la Police Républicaine (DGPR – Bénin)}

La plateforme de la DGPR permet aux citoyens béninois de déclarer officiellement la perte ou le vol d’un véhicule via un formulaire en ligne \cite{dgpr2023}. Cette démarche vise à simplifier l’accès aux services de police et à réduire les délais de traitement administratif.

Toutefois, l’interaction citoyenne reste limitée à une simple déclaration initiale. Le déclarant ne dispose pas d’un tableau de bord de suivi en temps réel, ni d’un mécanisme de notification en cas d’évolution du dossier. De plus, la plateforme ne favorise pas la collaboration entre citoyens, par exemple par le partage d’alertes géolocalisées ou de signalements communautaires.

\textbf{Limite principale :} absence de communication bidirectionnelle et de suivi dynamique post-déclaration.

\section{Base de données des véhicules volés d’Interpol (SMV)}

La base SMV d’Interpol constitue une référence mondiale pour l’identification des véhicules volés \cite{interpol2023}. Elle permet aux forces de l’ordre de plus de 190 pays de vérifier instantanément le statut d’un véhicule.

Néanmoins, cette plateforme est exclusivement réservée aux autorités policières. Les citoyens ne peuvent ni consulter directement la base, ni contribuer activement au signalement ou au suivi d’un véhicule volé. Le modèle repose sur une collaboration institutionnelle forte, mais exclut totalement la participation citoyenne.

\textbf{Limite principale :} collaboration internationale efficace, mais absence totale d’implication directe des citoyens.

\subsection{DIGITPOL – Base internationale privée}

DIGITPOL propose une base de données internationale accessible aux particuliers et aux entreprises \cite{digitpol2023}. Elle permet aux citoyens de vérifier si un véhicule figure parmi les véhicules déclarés volés et d’améliorer la prévention lors d’achats de véhicules d’occasion.

Cependant, le système ne garantit pas la valeur légale des déclarations et ne propose pas de suivi en temps réel coordonné avec les autorités nationales. La collaboration citoyenne est essentiellement passive, limitée à la consultation d’informations.

\textbf{Limite principale :} participation citoyenne existante mais non intégrée dans un écosystème officiel de sécurité publique.

\subsection{Système d’Information Schengen (SIS II – Europol)}

Le SIS II est un système hautement sécurisé utilisé par les États membres de l’Union européenne pour le partage d’informations policières, y compris les véhicules volés \cite{europol2023}. Il permet une coopération transfrontalière rapide et efficace.

Toutefois, comme Interpol, le SIS II est inaccessible aux citoyens. Les informations circulent uniquement entre institutions, ce qui limite la rapidité de détection communautaire et empêche toute forme de suivi citoyen en temps réel.

\textbf{Limite principale :} efficacité institutionnelle élevée mais absence d’intelligence collective citoyenne.

\subsection{Service Public France}

La plateforme Service Public France permet aux citoyens de déclarer le vol d’un véhicule en ligne et de générer des documents officiels utilisables auprès des assurances \cite{servicepublic2023}. Elle représente une avancée notable en matière de dématérialisation administrative.

Cependant, le suivi reste administratif et statique. Il n’existe pas de système d’alerte communautaire, ni de visualisation en temps réel de l’état des recherches. La communication se fait essentiellement de manière descendante.

\textbf{Limite principale :} interaction citoyenne limitée aux démarches administratives.

\subsection{Plateforme de la Collectivité de Saint-Martin}

La collectivité de Saint-Martin propose un service numérique similaire à celui de la France métropolitaine pour la déclaration de vols \cite{saintmartin2023}. Bien que fonctionnelle, la plateforme reste isolée et ne favorise ni la coopération interterritoriale ni la participation communautaire active.

\section{Analyse transversale : collaboration et suivi en temps réel}

L’analyse des plateformes existantes met en évidence une fragmentation des rôles :
\begin{itemize}
    \item Les plateformes institutionnelles privilégient la sécurité et la confidentialité.
    \item Les plateformes ouvertes privilégient l’accessibilité mais manquent de reconnaissance légale.
\end{itemize}

Aucune plateforme ne combine pleinement :
\begin{itemize}
    \item la déclaration officielle,
    \item le suivi en temps réel,
    \item la collaboration active entre citoyens,
    \item et l’intégration avec les forces de l’ordre.
\end{itemize}


\newpage

\section{Analyse comparative des plateformes existantes}

\begin{table}[h!]
\centering
\renewcommand{\arraystretch}{1.4}
\small
\begin{tabular}{|p{3.5cm}|p{2cm}|p{2cm}|p{2cm}|p{2cm}|p{2.5cm}|}
\hline
\textbf{Critères} 
& \textbf{DGPR (Bénin)} 
& \textbf{Interpol (SMV)} 
& \textbf{DIGITPOL} 
& \textbf{Europol (SIS II)} 
& \textbf{Service Public France} \\
\hline

Déclaration par le citoyen 
& Oui 
& Non 
& Oui 
& Non 
& Oui \\
\hline

Suivi du dossier par le citoyen 
& Non 
& Non 
& Partiel 
& Non 
& Limité \\
\hline

Interaction citoyen--autorité 
& Faible 
& Aucune 
& Faible 
& Aucune 
& Faible \\
\hline

Signalement collaboratif 
& Non 
& Non 
& Oui 
& Non 
& Non \\
\hline

Suivi en temps réel 
& Non 
& Oui (interne) 
& Partiel 
& Oui (interne) 
& Non \\
\hline

Notifications automatiques 
& Non 
& Non 
& Oui 
& Non 
& Partiel \\
\hline

Accès aux mises à jour 
& Manuel 
& Réservé police 
& Public 
& Réservé police 
& Administratif \\
\hline

Participation communautaire 
& Non 
& Non 
& Oui 
& Non 
& Non \\
\hline

Portée géographique 
& Nationale 
& Internationale 
& Internationale 
& Union Européenne 
& Nationale \\
\hline

Statut légal 
& Officiel 
& Officiel 
& Non officiel 
& Officiel 
& Officiel \\
\hline
\end{tabular}
\caption{Comparaison des plateformes de gestion et de suivi des vols de véhicules axée sur l’interaction citoyenne}
\label{tab:comparaison_interaction}
\end{table}





\newpage
\section{Tableau comparatif orienté collaboration citoyenne}

\begin{table}[h!]
\centering
\renewcommand{\arraystretch}{1.3}
\begin{tabular}{|p{3cm}|p{2.2cm}|p{2.2cm}|p{2.2cm}|p{2.2cm}|p{2.2cm}|}
\hline
\textbf{Critères} & \textbf{DGPR} & \textbf{Interpol} & \textbf{DIGITPOL} & \textbf{Europol} & \textbf{Service Public FR} \\
\hline
Participation citoyenne & Faible & Nulle & Moyenne & Nulle & Faible \\
\hline
Suivi en temps réel & Non & Oui (interne) & Partiel & Oui (interne) & Non \\
\hline
Alertes communautaires & Non & Non & Non & Non & Non \\
\hline
Interaction citoyen–police & Unidirectionnelle & Indirecte & Non officielle & Indirecte & Unidirectionnelle \\
\hline
Collaboration transfrontalière & Limitée & Très forte & Moyenne & Très forte & Faible \\
\hline
\end{tabular}
\caption{Comparaison des plateformes selon la collaboration et le suivi en temps réel}
\end{table}

\section{Synthèse et positionnement du projet}

Cette analyse révèle un manque significatif de plateformes hybrides intégrant à la fois les citoyens et les forces de l’ordre dans un système collaboratif et dynamique. Le projet proposé dans ce mémoire vise à combler cette lacune en introduisant un système de déclaration, de suivi en temps réel et de collaboration citoyenne sécurisée, adapté au contexte africain et extensible à l’international.


Plusieurs plateformes nationales et internationales ont été développées pour répondre à la problématique des vols de véhicules. Les plateformes institutionnelles, telles que celles mises en place par les forces de l’ordre, permettent une déclaration officielle et servent de base juridique aux enquêtes. Toutefois, elles offrent peu de mécanismes de suivi interactif pour les citoyens.

À l’échelle internationale, certaines bases de données facilitent la coopération entre pays, mais restent strictement réservées aux autorités compétentes. En parallèle, des plateformes privées proposent des espaces de signalement ouverts, favorisant la sensibilisation, sans toutefois disposer d’une reconnaissance légale.

L’existant se caractérise donc par une séparation marquée entre systèmes officiels et outils orientés vers le grand public.

\section{Comparaison des solutions existantes}

L’analyse comparative des solutions existantes met en évidence plusieurs différences notables :
\begin{itemize}
  \item les plateformes institutionnelles privilégient la sécurité et la conformité légale ;
  \item les plateformes ouvertes favorisent l’accessibilité et la diffusion de l’information ;
  \item aucune solution ne propose une collaboration structurée et continue entre citoyens et autorités.
\end{itemize}

Cette situation entraîne une rupture dans le suivi des dossiers et une faible implication citoyenne après la phase de déclaration.

\section{Notre solution}

Afin de répondre aux limites observées, notre solution propose une plateforme intégrée axée sur la collaboration et le suivi en temps réel. Elle permet aux citoyens de suivre l’évolution de leur déclaration, de recevoir des notifications et de contribuer activement par des signalements vérifiés.  
Les forces de l’ordre disposent quant à elles d’un outil centralisé favorisant une meilleure coordination et une prise de décision plus rapide.

Cette approche vise à renforcer l’efficacité globale du système de gestion des vols de véhicules tout en améliorant l’expérience des utilisateurs.





\section*{Conclusion}

Cette revue de littérature a mis en évidence les enjeux liés à la gestion et au suivi des vols de véhicules ainsi que les limites des solutions existantes. L’absence de collaboration dynamique et de suivi continu constitue un frein majeur à l’efficacité des dispositifs actuels. Ces constats justifient la conception d’une plateforme collaborative et interactive, qui fera l’objet des chapitres suivants.



\section{Analyse comparative des solutions existantes}

\subsection{Critères fonctionnels}

L’analyse comparative des plateformes existantes repose sur plusieurs critères fonctionnels, notamment la déclaration du vol, la gestion des dossiers, l’accès aux informations, la communication entre acteurs et les mécanismes de suivi.

Les plateformes institutionnelles privilégient la sécurité et la conformité légale, tandis que les plateformes privées mettent davantage l’accent sur l’accessibilité et la diffusion de l’information. Toutefois, aucune solution ne parvient à concilier pleinement ces deux dimensions.

\subsection{Interaction citoyenne}

L’interaction citoyenne constitue l’un des points faibles majeurs des systèmes actuels. Dans la majorité des cas, le rôle du citoyen se limite à la phase de déclaration initiale, sans possibilité d’interaction continue avec le système ou les autorités.

Cette absence d’interaction réduit l’implication des citoyens et limite les opportunités de collaboration collective, pourtant essentielles dans un contexte où les informations locales et communautaires peuvent s’avérer déterminantes.

\subsection{Suivi en temps réel et notifications}

Le suivi en temps réel et les mécanismes de notification sont rarement proposés aux citoyens. Lorsqu’ils existent, ils sont généralement réservés à un usage interne par les forces de l’ordre.

L’absence de notifications automatiques empêche les citoyens d’être informés des évolutions de leur dossier, ce qui nuit à la transparence et à la réactivité du système. Cette limitation constitue un frein important à l’efficacité globale des solutions existantes.

---

\section{Limites des systèmes actuels}

\subsection{Faible collaboration citoyenne}

Les systèmes actuels offrent peu de mécanismes favorisant la collaboration citoyenne. L’information circule principalement de manière descendante, sans exploitation du potentiel collectif des communautés locales.

Cette approche centralisée limite les possibilités de signalement participatif et de contribution citoyenne à la recherche des véhicules volés.

\subsection{Manque de visibilité sur le suivi}

Le manque de visibilité sur l’état d’avancement des dossiers constitue une source majeure d’insatisfaction pour les citoyens. L’absence d’indicateurs clairs et de mises à jour régulières crée un sentiment d’opacité et réduit la confiance envers les dispositifs existants.

\subsection{Absence d’intégration des données}

Les plateformes existantes fonctionnent souvent de manière isolée, sans interconnexion entre systèmes nationaux, internationaux et privés. Cette absence d’intégration des données entraîne des silos d’information et réduit l’efficacité des actions coordonnées.

---

\section{Synthèse et positionnement du travail}

\subsection{Justification de la solution proposée}

L’analyse de l’existant met en évidence un besoin réel de solutions intégrées favorisant la collaboration et le suivi continu des vols de véhicules. L’absence de plateformes combinant déclaration officielle, interaction citoyenne, suivi en temps réel et notifications automatiques constitue une lacune majeure.

La solution proposée dans ce travail vise à combler ce vide en mettant l’accent sur une approche collaborative, interactive et centrée sur l’utilisateur, tout en respectant les exigences institutionnelles. Elle se positionne ainsi comme une réponse innovante et adaptée aux limites identifiées dans les systèmes actuels.


\section{Conclusion}

Le vol de véhicules demeure un défi majeur de sécurité publique nécessitant une approche multidimensionnelle. Les plateformes de lutte contre le vol de véhicules, qu’elles soient institutionnelles, technologiques ou communautaires, jouent un rôle essentiel dans la prévention, la détection et la récupération des véhicules volés.

Aucune solution unique ne peut répondre à l’ensemble des problématiques liées au vol automobile. Une combinaison de technologies, associée à une coopération internationale renforcée et à une sensibilisation des usagers, constitue la stratégie la plus efficace pour réduire durablement ce phénomène.




















\chapter{Revue de littérature et analyse de l’existant}

%====================================================
\section{Généralités sur la gestion et le suivi des vols de véhicules}

\subsection{Définition du suivi des véhicules volés}

Le suivi des véhicules volés désigne l’ensemble des mécanismes organisationnels et numériques permettant d’enregistrer, surveiller et mettre à jour l’état d’un véhicule déclaré volé jusqu’à sa récupération ou sa régularisation.

\subsection{Importance du suivi}

Le suivi constitue un élément fondamental dans la lutte contre la criminalité automobile. Il permet la centralisation des informations, la coordination des interventions et la réduction de la circulation illégale des véhicules.

Un suivi structuré améliore également la continuité des investigations, réduit les délais de réaction et augmente les chances de récupération des véhicules. Du point de vue du citoyen, il renforce la confiance envers les institutions et évite le sentiment d’abandon après la déclaration.

\subsection{Processus de déclaration et de gestion}

Le processus débute par une déclaration officielle effectuée auprès des services compétents. Cette déclaration entraîne l’ouverture d’un dossier administratif et judiciaire contenant les informations essentielles du véhicule et les circonstances du vol.

Le dossier suit ensuite plusieurs étapes : vérification des informations, diffusion du signalement, actions de recherche et éventuellement clôture. Toutefois, dans de nombreux systèmes, ce processus reste interne aux institutions, offrant peu de visibilité au citoyen.

\subsection{Composantes clés d’un système de gestion}

Les systèmes modernes reposent sur :

\begin{itemize}
    \item l’enregistrement structuré des déclarations ;
    \item la centralisation des informations ;
    \item la mise à jour régulière des statuts ;
    \item la communication entre acteurs ;
    \item l’archivage des événements.
\end{itemize}

\subsection{Défis liés au suivi}

Malgré les dispositifs existants, plusieurs défis persistent :

\begin{itemize}
    \item lenteur des procédures ;
    \item manque de visibilité pour les citoyens ;
    \item faible coordination entre plateformes ;
    \item absence de suivi en temps réel accessible au public.
\end{itemize}

%====================================================
\section{Étude de l’existant}

\subsection{Plateformes institutionnelles nationales}

\subsubsection{DGPR – Bénin}

La plateforme de la DGPR permet aux citoyens de déclarer officiellement un vol via un formulaire en ligne \cite{dgpr2023}.  
Cependant, elle ne propose ni suivi en temps réel ni notifications automatiques.

\textbf{Limite principale :} absence de suivi dynamique et de communication bidirectionnelle.

\subsubsection{Service Public France}

La plateforme permet la déclaration en ligne et la génération de documents officiels \cite{servicepublic2023}.  
Le suivi reste cependant administratif et non interactif.

\textbf{Limite principale :} interaction limitée aux démarches administratives.

\subsection{Systèmes institutionnels internationaux}

\subsubsection{Interpol – SMV}

Base mondiale permettant aux forces de l’ordre de vérifier les véhicules volés \cite{interpol2023}.  
Elle est exclusivement réservée aux autorités.

\textbf{Limite principale :} absence totale d’implication citoyenne.

\subsubsection{Europol – SIS II}

Système sécurisé de coopération transfrontalière \cite{europol2023}.  
Accès strictement institutionnel.

\textbf{Limite principale :} aucune intelligence collective citoyenne.

\subsection{Plateformes privées}

\subsubsection{DIGITPOL}

Base internationale accessible au public \cite{digitpol2023}.  
Toutefois, elle ne possède pas de reconnaissance légale officielle.

\textbf{Limite principale :} non-intégration avec les autorités nationales.

%====================================================
\section{Analyse comparative des plateformes existantes}

\subsection{Critères d’analyse}

L’analyse repose sur :

\begin{itemize}
    \item déclaration citoyenne ;
    \item suivi du dossier ;
    \item interaction citoyen--autorité ;
    \item collaboration communautaire ;
    \item suivi en temps réel ;
    \item notifications automatiques ;
    \item portée géographique ;
    \item statut légal.
\end{itemize}

% (Insérer ici ton tableau comparatif principal)

%====================================================
\section{Analyse transversale}

L’étude des plateformes existantes révèle une fragmentation :

\begin{itemize}
    \item Les plateformes institutionnelles privilégient la sécurité et la confidentialité.
    \item Les plateformes ouvertes privilégient l’accessibilité mais manquent de reconnaissance légale.
\end{itemize}

Aucune solution ne combine simultanément :

\begin{itemize}
    \item déclaration officielle,
    \item suivi en temps réel,
    \item collaboration citoyenne active,
    \item intégration avec les forces de l’ordre.
\end{itemize}

%====================================================
\section{Limites des systèmes actuels}

\subsection{Faible collaboration citoyenne}

Le rôle du citoyen se limite généralement à la déclaration initiale.

\subsection{Manque de visibilité}

L’absence d’indicateurs clairs et de notifications réduit la transparence.

\subsection{Absence d’intégration des données}

Les systèmes fonctionnent en silos, limitant l’efficacité coordonnée.

%====================================================
\section{Synthèse et positionnement du travail}

L’analyse met en évidence un besoin de plateforme hybride combinant déclaration officielle, interaction continue, suivi en temps réel et collaboration sécurisée.

%====================================================
\section{Notre solution}

La solution proposée vise à :

\begin{itemize}
    \item permettre la déclaration officielle en ligne ;
    \item offrir un suivi en temps réel ;
    \item intégrer des notifications automatiques ;
    \item favoriser la collaboration citoyenne sécurisée ;
    \item assurer l’intégration avec les autorités compétentes.
\end{itemize}

%====================================================
\section*{Conclusion}

Cette revue de littérature a permis d’identifier les limites majeures des systèmes actuels, notamment l’absence de collaboration dynamique et de suivi interactif. Ces constats justifient le développement d’une plateforme intégrée et innovante.



















% \section{Définition et Concepts Clés}
% Avant de discuter des solutions actuelles, il est essentiel de définir quelques concepts clés pour mieux comprendre les technologies et méthodes utilisées dans la lutte contre le vol de véhicules.

% \subsection{Systèmes de Sécurité pour Véhicules}
% Les systèmes de sécurité pour véhicules sont des dispositifs technologiques conçus pour les protéger contre le vol et le vandalisme. Ces systèmes incluent des alarmes, des dispositifs de verrouillage électronique, des dispositifs de géolocalisation, ainsi que d’autres mécanismes visant à prévenir ou à détecter un vol en cours.

% \subsection{Géolocalisation et Traçabilité}
% La géolocalisation permet de suivre en temps réel la position d’un véhicule, notamment grâce à des technologies comme le GPS (Global Positioning System). Ces dispositifs permettent aux autorités de localiser un véhicule volé et de faciliter sa récupération rapide.


% \section*{Plateformes principales de gestion et suivi des vols de véhicules}


% \subsection{Interpol-Fichier des Véhicules Volés (FVV)}

% \textbf{Description:}  
% Interpol met à disposition la base de données \textit{Stolen Motor Vehicles Database (SMV)}, qui recense les véhicules volés à l’échelle internationale et facilite le suivi transfrontalier. Cette plateforme permet aux forces de l’ordre de consulter et d’échanger des informations sur les véhicules volés. \cite{interpol2023}

% \textbf{Fonctionnalités:}
% \begin{itemize}
%     \item Enregistrement des véhicules volés par les pays membres.
%     \item Consultation en temps réel par les forces de l’ordre.
%     \item Échange sécurisé d’informations entre États.
%     \item Identification des véhicules retrouvés à l’étranger.
%     \item Base de données mondiale des véhicules volés.
%     \item Accès en temps réel pour les forces de l’ordre.
%     \item Coopération internationale grâce aux Bureaux Nationaux Interpol (NCB).
% \end{itemize}

% \textbf{Avantages:}
% \begin{itemize}
%     \item Portée mondiale.
%     \item Améliore la rapidité des enquêtes transfrontalières.
%     \item Données fiables mises à jour régulièrement.
%     \item Couverture internationale permettant le suivi des véhicules volés dans plusieurs pays.  
%     \item Facilite la coopération entre forces de l’ordre et augmente les chances de récupération des véhicules.  
%     \item Accès centralisé aux informations sur les vols, réduisant les redondances.
% \end{itemize}

% \textbf{Limites:}
% \begin{itemize}


%      \item Accès réservé aux forces de l’ordre et aux agences partenaires, donc limité pour les particuliers.  
%     \item Dépendance à la mise à jour régulière des informations par chaque pays.  
%     \item Ne couvre pas les vols non déclarés ou informels.
% \end{itemize}





% \subsection{ Europol – Schengen Information System (SIS II)}
% \textbf{Description :}  
% Le \textit{Schengen Information System (SIS II)} est une base de données européenne utilisée par les États membres pour suivre les véhicules volés et autres informations relatives à la sécurité. Elle permet le signalement immédiat et la consultation en temps réel des véhicules volés dans l’espace Schengen. \cite{europol2023}
% \textbf{Fonctionnalités :}
% \begin{itemize}
%     \item Signalement immédiat des véhicules volés.
%     \item Consultation en temps réel par les autorités frontalières et policières.
%     \item Interconnexion avec d’autres systèmes de sécurité européens.
% \end{itemize}
% \textbf{Avantages :}  
% \begin{itemize}
%     \item Couverture efficace des pays membres de l’espace Schengen.  
%     \item Permet une réaction rapide des autorités grâce à la consultation en temps réel.  
%     \item Intègre les véhicules volés dans une approche globale de sécurité publique.
% \end{itemize}

% \textbf{Limites :}  
% \begin{itemize}
%     \item Réservé aux autorités publiques et aux agents habilités.  
%     \item Limité à l’espace Schengen, ne couvre pas les véhicules volés en dehors de cette zone.  
%     \item Nécessite une infrastructure informatique fiable pour un accès et une mise à jour efficaces.
% \end{itemize}

% \subsection{Service Public France – Déclarer un vol de véhicule}
% \textbf{Description :}  
% Le site officiel du \textit{Service Public} permet aux particuliers de déclarer rapidement un vol de véhicule en ligne, facilitant la prise en charge par les forces de l’ordre et la transmission aux compagnies d’assurance. \cite{servicepublic2023}

% \textbf{Fonctionnalités :}
% \begin{itemize}
%     \item Déclaration en ligne du vol de véhicule.
%     \item Génération d’un récépissé officiel.
%     \item Transmission automatique aux forces de l’ordre.
%     \item Utilisation du dossier pour les assurances.
% \end{itemize}

% \textbf{Avantages :}  
% \begin{itemize}
%     \item Accès facile pour tous les citoyens français, simplifiant la procédure de déclaration.  
%     \item Permet une traçabilité officielle des véhicules volés.  
%     \item Peut accélérer les démarches administratives et d’assurance.
% \end{itemize}

% \textbf{Limites :}  
% \begin{itemize}
%     \item Limité au territoire français.  
%     \item Ne fournit pas de suivi en temps réel ni de géolocalisation du véhicule.  
%     \item La récupération du véhicule dépend entièrement des forces de l’ordre et du suivi administratif.
% \end{itemize}


% \section{Plateforme de déclaration de vol/perte – DGPR (Bénin)}

% \subsection{Description}
% La plateforme de la \textit{Direction Générale de la Police Républicaine} (DGPR) du Bénin permet de déclarer en ligne des vols ou pertes, y compris potentiellement pour un véhicule.  
% Ce service est utilisé pour lancer une procédure officielle auprès de la police républicaine du Bénin.

% \subsection{Lien}
% \begin{itemize}
%     \item \url{https://www.dgpr.bj/declaration-de-vol-perte/}
% \end{itemize}

% \subsection{Fonctionnalités}
% \begin{itemize}
%     \item Formulaire en ligne pour déclarer un vol ou une perte.
%     \item Saisie des informations personnelles et de l’objet volé (éventuellement véhicule).
%     \item Transmission directement aux services de police compétents.
%     \item Possibilité d’être contacté par la police suite à la déclaration.
% \end{itemize}

% \subsection{Avantages}
% \begin{itemize}
%     \item Service officiel de la police nationale du Bénin.
%     \item Permet de déclarer rapidement un vol sans présence immédiate en commissariat.
%     \item Accessible via Internet.
% \end{itemize}

% \subsection{Limites}
% \begin{itemize}
%     \item La plate-forme ne garantit pas à elle seule une plainte complète (un suivi peut être requis).
%     \item Possible nécessité de se rendre physiquement au poste de police pour finaliser la procédure.
%     \item Interface limitée selon les capacités techniques de la plateforme.
% \end{itemize}

% %------------------------------------------------

% \section{Déclaration de vol de véhicule – Collectivité de Saint-Martin}

% \subsection{Description}
% La Collectivité de Saint-Martin propose une page dédiée aux démarches administratives pour les véhicules, incluant la procédure à suivre en cas de vol de véhicule à Saint-Martin (Antilles françaises). :contentReference[oaicite:2]{index=2}

% \subsection{Lien}
% \begin{itemize}
%     \item \url{https://www.comstmartin.fr/demarches_administratives}
% \end{itemize}

% \subsection{Fonctionnalités}
% \begin{itemize}
%     \item Informations sur la procédure de déclaration de vol de véhicule.
%     \item Indications claires pour effectuer le dépôt de plainte auprès de la Gendarmerie.
%     \item Instructions pour transmettre la déclaration à l’assurance et au Service des titres de circulation (via e-mail ou contact administratif). :contentReference[oaicite:3]{index=3}
%     \item Contacts utiles du \textit{Service des titres de circulation} (adresse, téléphone, e-mail). :contentReference[oaicite:4]{index=4}
% \end{itemize}

% \subsection{Avantages}
% \begin{itemize}
%     \item Adapté aux démarches spécifiques à Saint-Martin.
%     \item Fournit des informations claires sur les contacts administratifs locaux. :contentReference[oaicite:5]{index=5}
%     \item Permet de connaître l’ordre des démarches (police, assurance, services administratifs). :contentReference[oaicite:6]{index=6}
% \end{itemize}

% \subsection{Limites}
% \begin{itemize}
%     \item Il ne s’agit pas d’un dépôt de plainte en ligne automatisé.
%     \item L’usager doit effectuer physiquement certaines démarches (déposer plainte en gendarmerie).
%     \item Nécessite souvent l’envoi de courriels ou de documents physiques par le propriétaire. :contentReference[oaicite:7]{index=7}
% \end{itemize}

% %------------------------------------------------

% \section{DIGITPOL – Base internationale de véhicules volés}

% \subsection{Description}
% DIGITPOL est une plateforme internationale permettant d’enregistrer un véhicule volé dans une base de données mondiale.

% \subsection{Fonctionnalités}
% \begin{itemize}
%     \item Enregistrement international du véhicule volé.
%     \item Diffusion des informations aux partenaires internationaux.
%     \item Vérification du statut d’un véhicule.
% \end{itemize}

% \subsection{Avantages}
% \begin{itemize}
%     \item Portée internationale.
%     \item Utile pour les vols transfrontaliers.
%     \item Complément aux démarches nationales.
% \end{itemize}

% \subsection{Limites}
% \begin{itemize}
%     \item Ne constitue pas une plainte officielle.
%     \item Dépend de la coopération internationale.
% \end{itemize}







% \section{Plateformes principales de gestion et de suivi des vols de véhicules}




























% %------------------------------------------------

% \section{Plateforme de déclaration de vol/perte – DGPR (Bénin)}

% \subsection{Fonctionnalités}
% La plateforme de la \textit{Direction Générale de la Police Républicaine (DGPR)} du Bénin permet aux citoyens de déclarer en ligne des vols ou pertes, y compris ceux concernant des véhicules \cite{dgpr2023}.

% \subsection{Fonctionnalités :}
% \begin{itemize}
%     \item Formulaire de déclaration en ligne.
%     \item Transmission aux services de police compétents.
%     \item Possibilité de contact ultérieur par la police.
% \end{itemize}

% \subsection{Avantages :}
% \begin{itemize}
%     \item Service officiel de la police béninoise.
%     \item Réduction des déplacements initiaux.
%     \item Accessibilité via Internet.
% \end{itemize}

% \subsection{Limites :}
% \begin{itemize}
%     \item Une validation physique peut être exigée.
%     \item Fonctionnalités numériques limitées.
% \end{itemize}

% %------------------------------------------------

% \section{Interpol – Fichier des Véhicules Volés (FVV)}

% \subsection{Description :}  
% Interpol met à disposition la base de données \textit{Stolen Motor Vehicles Database (SMV)}, qui recense les véhicules volés à l’échelle internationale et facilite le suivi transfrontalier.  
% Cette plateforme est utilisée exclusivement par les forces de l’ordre afin de consulter et d’échanger des informations sur les véhicules volés \cite{interpol2023}.

% \subsection{Fonctionnalités :}
% \begin{itemize}
%     \item Enregistrement des véhicules volés par les pays membres.
%     \item Consultation en temps réel par les forces de l’ordre.
%     \item Échange sécurisé d’informations entre États.
%     \item Identification des véhicules retrouvés à l’étranger.
%     \item Coopération internationale via les Bureaux Nationaux Interpol (NCB).
% \end{itemize}

% \subsection{Avantages :}
% \begin{itemize}
%     \item Portée mondiale.
%     \item Amélioration des enquêtes transfrontalières.
%     \item Données fiables et régulièrement mises à jour.
%     \item Centralisation des informations sur les véhicules volés.
% \end{itemize}

% \subsection{Limites :}
% \begin{itemize}
%     \item Accès strictement réservé aux forces de l’ordre.
%     \item Dépendance à la qualité des mises à jour nationales.
%     \item Les vols non déclarés ne sont pas pris en compte.
% \end{itemize}
% %------------------------------------------------


% \section{DIGITPOL Automobile– Expertise et sécurité des véhicules}
% \subsection{Description}
% Digitpol Automotive est une unité spécialisée dédiée aux véhicules et à la sécurité, combinant criminalistique automobile et télématique avancée.  
% Elle offre une expertise approfondie dans la localisation des véhicules volés, les enquêtes sur les fraudes à l’assurance, la sécurité embarquée et les applications télématiques sur mesure pour les services gouvernementaux et de sécurité \cite{digitpol2023}.  
% L’équipe est composée d’experts hautement qualifiés, ayant une expérience dans les forces de l’ordre et les opérations techniques secrètes, capables de réaliser des enquêtes approfondies et de fournir des analyses médico-légales de véhicules.

% \subsection{Fonctionnalités}
% \begin{itemize}
%     \item Lecture automatique des plaques d’immatriculation (LAPI).
%     \item Développement et intégration de solutions IoT pour véhicules.
%     \item Dispositifs audio discrets et installation de traceurs GPS.
%     \item Applications télématiques embarquées et personnalisées pour services gouvernementaux.
%     \item Expertises médico-légales de véhicules et analyses criminologiques.
%     \item Enquêtes sur les fraudes à l’assurance et vols de véhicules.
%     \item Localisation et traçage de véhicules volés.
% \end{itemize}

% \subsection{Avantages}
% \begin{itemize}
%     \item Expertise combinant criminalistique, télématique et sécurité des véhicules.
%     \item Solutions sur mesure adaptées aux besoins gouvernementaux et sécuritaires.
%     \item Renforcement de la prévention et de la détection des vols et fraudes.
%     \item Capacité à mener des enquêtes complexes et à fournir des preuves fiables.
%     \item Technologies avancées pour la localisation et le suivi des véhicules.
% \end{itemize}

% \subsection{Limites}
% \begin{itemize}
%     \item Les services sont principalement destinés aux autorités ou organisations spécialisées, pas aux particuliers.
%     \item Dépendance à l’accès aux technologies embarquées et aux données télématiques.
%     \item Ne remplace pas les procédures légales officielles pour déclarer un vol.
%     \item Certaines fonctionnalités requièrent une expertise technique pour leur installation et utilisation.
% \end{itemize}




% %------------------------------------------------

% \section{Europol – Schengen Information System (SIS II)}

% \subsection{Description :}  
% Le \textit{Schengen Information System (SIS II)} est une base de données européenne permettant aux États membres de signaler et de consulter les informations relatives aux véhicules volés dans l’espace Schengen \cite{europol2023}.

% \textbf{Fonctionnalités :}
% \begin{itemize}
%     \item Signalement immédiat des véhicules volés.
%     \item Consultation en temps réel par les autorités policières et frontalières.
%     \item Interconnexion avec d’autres systèmes européens de sécurité.
% \end{itemize}

% \subsection{Avantages :}
% \begin{itemize}
%     \item Couverture efficace de l’espace Schengen.
%     \item Réactivité accrue des forces de l’ordre.
%     \item Intégration dans une stratégie globale de sécurité européenne.
% \end{itemize}

% \subsection{Limites :}
% \begin{itemize}
%     \item Europol ne reçoit pas directement les signalements du public.
%     \item Elle ne dispose d’aucun pouvoir d’arrestation ou d’enquête autonome.
%     \item Son action dépend entièrement de la coopération et des informations fournies par les États membres.
%     \item Les interventions sont limitées aux affaires présentant une dimension internationale.
%     \item Réservé aux autorités habilitées.
%     \item Limité géographiquement à l’espace Schengen.
%     \item Dépendance à une infrastructure informatique performante.
% \end{itemize}

% %------------------------------------------------

% \section{Service Public France – Déclaration de vol de véhicule}

% \subsection{Description :}  
% La plainte en ligne permet aux victimes de déclarer à distance le vol de leur véhicule et d’obtenir les informations nécessaires pour les démarches auprès des forces de l’ordre et des compagnies d’assurance \cite{servicepublic2023}.
% Ce dispositif officiel, gratuit et accessible via Internet, vise à simplifier le dépôt de plainte sans nécessiter un déplacement immédiat en commissariat ou en brigade de gendarmerie.  
% Il s’applique aux vols de véhicules commis sur le territoire français et constitue une première étape essentielle pour les démarches administratives et assurantielles.

% \subsection{Fonctionnalités}
% \begin{itemize}
%     \item Déclaration en ligne du vol de véhicule.
%     \item Authentification sécurisée, notamment via FranceConnect.
%     \item Formulaire détaillé permettant de renseigner les informations du véhicule (immatriculation, circonstances du vol, lieu, date).
%     \item Génération d’un accusé de réception après validation de la plainte.
%     \item Mise à disposition du procès-verbal de plainte dans l’espace usager.
%     \item Possibilité d’être contacté par la police ou la gendarmerie pour compléter la déclaration.
% \end{itemize}

% \subsection{Avantages}
% \begin{itemize}
%     \item Gain de temps grâce au dépôt de plainte à distance.
%     \item Réduction des déplacements initiaux en commissariat ou gendarmerie.
%     \item Procédure officielle reconnue par les assurances.
%     \item Accessibilité pour les résidents et les visiteurs étrangers victimes d’un vol en France.
% \end{itemize}

% \subsection{Limites}
% \begin{itemize}
%     \item Limité aux vols de véhicules dont l’auteur est inconnu.
%     \item Un déplacement physique peut être exigé pour finaliser ou compléter la plainte.
%     \item Ne permet pas le suivi en temps réel de la recherche du véhicule.
%     \item L’annulation de la plainte ne peut pas être effectuée en ligne.
% \end{itemize}


% %------------------------------------------------

% \section{Déclaration de vol de véhicule – Collectivité de Saint-Martin}

% \subsection{Description :}  
% La Collectivité de Saint-Martin met à disposition une page d’information détaillant les démarches administratives à effectuer en cas de vol de véhicule sur son territoire \cite{saintmartin2023}.

% \textbf{Fonctionnalités :}
% \begin{itemize}
%     \item Présentation de la procédure officielle.
%     \item Orientation vers la gendarmerie.
%     \item Instructions pour les démarches auprès de l’assurance et des services administratifs.
% \end{itemize}

% \subsection{Avantages :}
% \begin{itemize}
%     \item Adapté au contexte local de Saint-Martin.
%     \item Informations claires et structurées.
% \end{itemize}

% \textbf{Limites :}
% \begin{itemize}
%     \item Absence de dépôt de plainte entièrement en ligne.
%     \item Démarches partiellement manuelles.
% \end{itemize}





















% \section{État de l’art}

% \subsection{Contexte général}

% Le vol de véhicules constitue un phénomène criminel transnational qui touche aussi bien les pays développés que les pays en développement. La mondialisation des échanges, la porosité des frontières et la sophistication croissante des réseaux criminels ont conduit les États et les organisations internationales à mettre en place des systèmes numériques dédiés à la déclaration, à la centralisation et au partage des informations relatives aux véhicules volés.

% Ces systèmes ont pour principaux objectifs de faciliter la déclaration rapide des vols, d’améliorer la traçabilité des véhicules, de renforcer la coopération entre les forces de l’ordre et de limiter la revente ou l’exportation illégale des véhicules.

% \subsection{Plateformes nationales de déclaration}

% Plusieurs pays ont développé des plateformes nationales permettant aux citoyens de déclarer le vol ou la perte de leurs biens, notamment les véhicules. Au Bénin, la Direction Générale de la Police Républicaine (DGPR) propose une plateforme officielle de déclaration de vol ou de perte, servant de point d’entrée administratif et judiciaire. Cette démarche permet l’enregistrement du véhicule dans les bases internes de la police et facilite les procédures d’enquête et d’indemnisation.

% En France, le portail Service-Public.fr centralise les démarches administratives liées à la déclaration de vol de véhicule. Ce portail s’appuie sur une coopération étroite entre les services de police, de gendarmerie et les compagnies d’assurance.

% Cependant, ces plateformes nationales présentent une portée géographique limitée, ce qui réduit leur efficacité lorsque les véhicules sont rapidement déplacés hors du territoire national.

% \subsection{Bases de données internationales institutionnelles}

% Afin de renforcer la coopération transfrontalière, des bases de données internationales ont été mises en place. Interpol gère la base mondiale des véhicules volés (Stolen Motor Vehicles Database - SMV), accessible uniquement aux forces de l’ordre des pays membres. Cette base permet l’identification rapide des véhicules signalés volés lors des contrôles aux frontières.

% Dans l’espace européen, le Schengen Information System de deuxième génération (SIS II), exploité par Europol, permet le partage en temps réel des signalements de véhicules volés entre les États membres de l’Union européenne. Ce système constitue un outil central dans la lutte contre la criminalité transfrontalière.

% \subsection{Plateformes privées et hybrides}

% En complément des systèmes institutionnels, des plateformes privées ou hybrides ont émergé afin d’améliorer la visibilité internationale des véhicules volés. La plateforme DIGITPOL permet aux particuliers et aux organisations de signaler des véhicules volés dans une base consultable à l’échelle internationale.

% Ces plateformes jouent un rôle de sensibilisation et de prévention, mais ne remplacent pas les procédures officielles menées par les forces de l’ordre.

% \subsection{Limites des solutions existantes}

% Malgré la diversité des solutions disponibles, plusieurs limites persistent, notamment le manque d’interopérabilité entre les plateformes, l’accès restreint aux bases officielles pour les citoyens, ainsi que l’insuffisante digitalisation des procédures dans certains pays africains. Ces constats justifient la mise en place de plateformes intégrées et interconnectées, adaptées aux réalités locales.













% \subsection{Évolution des systèmes de déclaration de vol}

% Historiquement, la déclaration de vol de véhicule reposait essentiellement sur des démarches physiques effectuées auprès des services de police ou de gendarmerie. Ce mode opératoire, bien que juridiquement fiable, présentait plusieurs contraintes, notamment des délais importants, une accessibilité limitée pour les citoyens et une faible capacité de partage rapide de l’information entre les différentes institutions.

% Avec l’avènement des technologies de l’information et de la communication, les autorités publiques ont progressivement amorcé la digitalisation des procédures de déclaration. Cette évolution a permis une amélioration notable de la rapidité de traitement des signalements, une meilleure conservation des données et une facilitation de l’exploitation statistique des informations relatives aux vols de véhicules.

% Cependant, le niveau de maturité de ces systèmes numériques varie fortement selon les pays, créant des disparités importantes en matière de lutte contre le vol automobile.

% \subsection{Interopérabilité et partage de l’information}

% L’un des enjeux majeurs des plateformes de déclaration de vol réside dans leur capacité à interagir avec d’autres systèmes nationaux et internationaux. L’interopérabilité permet non seulement d’éviter la duplication des données, mais également d’assurer une circulation fluide et sécurisée des informations entre les forces de l’ordre, les autorités douanières et les organismes partenaires.

% Les bases de données institutionnelles telles que celles d’Interpol et d’Europol illustrent l’importance du partage d’informations à l’échelle internationale. Toutefois, ces systèmes restent majoritairement fermés au grand public, ce qui limite l’implication directe des citoyens dans le processus de détection et de signalement des véhicules volés.

% Cette situation met en évidence la nécessité de solutions intermédiaires capables de concilier sécurité, confidentialité et accessibilité.

% \subsection{Rôle des citoyens et des acteurs privés}

% Les citoyens et les acteurs privés, tels que les compagnies d’assurance, les concessionnaires automobiles et les plateformes numériques spécialisées, jouent un rôle croissant dans la lutte contre le vol de véhicules. Les plateformes privées comme DIGITPOL offrent aux particuliers la possibilité de diffuser rapidement l’information relative à un vol, augmentant ainsi les chances de localisation du véhicule.

% Néanmoins, l’absence de cadre juridique clair encadrant l’exploitation de certaines plateformes privées peut poser des problèmes de fiabilité des données et de reconnaissance légale des signalements effectués en dehors des circuits institutionnels.

% \subsection{Problématiques spécifiques aux pays en développement}

% Dans plusieurs pays africains, dont le Bénin, la lutte contre le vol de véhicules se heurte à des contraintes structurelles telles que la faible couverture numérique, le manque de systèmes centralisés et l’insuffisance des mécanismes de coopération transfrontalière automatisée.

% Ces limites entraînent une dépendance accrue aux procédures manuelles et ralentissent la diffusion des informations relatives aux véhicules volés. Par ailleurs, l’absence de statistiques publiques fiables complique l’analyse du phénomène et la mise en place de politiques de prévention adaptées.

% Ces constats renforcent l’intérêt de concevoir des plateformes nationales modernisées, intégrant des fonctionnalités de déclaration en ligne, de géolocalisation, d’alertes automatiques et de connexion avec les bases de données internationales existantes.

% \subsection{Synthèse et perspectives}

% L’analyse des solutions existantes met en évidence une complémentarité entre les plateformes nationales, les bases de données internationales et les solutions privées. Toutefois, aucune de ces solutions ne répond de manière exhaustive aux besoins spécifiques des pays en développement.

% Ainsi, l’état de l’art met en lumière la nécessité de proposer une plateforme intégrée, sécurisée et accessible, capable de centraliser les déclarations de vol, de faciliter la coopération entre les acteurs institutionnels et de renforcer la participation citoyenne. Cette réflexion constitue le fondement de la solution proposée dans la suite de ce travail.




































% \section{Analyse comparative des plateformes existantes}

% \begin{table}[h!]
% \centering
% \renewcommand{\arraystretch}{1.3}
% \begin{tabular}{|p{3cm}|p{2cm}|p{2cm}|p{2.5cm}|p{2.5cm}|p{2.5cm}|}
% \hline
% \textbf{Critères} & \textbf{DGPR (Bénin)} & \textbf{Interpol (SMV)} & \textbf{DIGITPOL} & \textbf{Europol (SIS II)} & \textbf{Service Public France} \\
% \hline
% Type de plateforme & Nationale & Internationale & Privée / Hybride & Régionale (UE) & Nationale \\
% \hline
% Accès citoyen & Oui & Non & Oui & Non & Oui \\
% \hline
% Accès forces de l’ordre & Oui & Oui & Partiel & Oui & Oui \\
% \hline
% Portée géographique & Nationale & Mondiale & Internationale & Union Européenne & Nationale \\
% \hline
% Objectif principal & Déclaration officielle & Coopération policière & Diffusion et prévention & Partage policier & Démarche administrative \\
% \hline
% Lien avec assurance & Indirect & Non & Non & Non & Oui \\
% \hline
% Signalement transfrontalier & Limité & Oui & Partiel & Oui & Limité \\
% \hline
% \textbf{Interaction avec les citoyens} & 
% Déclaration et suivi basiques & 
% Aucune interaction directe & 
% Signalement, consultation et diffusion & 
% Aucune interaction directe & 
% Déclaration et accompagnement administratif \\
% \hline
% Caractère légal & Officiel & Officiel & Non officiel & Officiel & Officiel \\
% \hline
% \end{tabular}
% \caption{Comparaison des principales plateformes de déclaration et de gestion des vols de véhicules}
% \label{tab:comparaison_plateformes}
% \end{table}



























% \section{Introduction générale}

% Le vol de véhicules constitue un enjeu majeur de sécurité publique et économique à l’échelle mondiale. Il affecte aussi bien les particuliers que les entreprises, les administrations publiques et les compagnies d’assurance. Face à l’augmentation de la criminalité organisée et à la mobilité transfrontalière des véhicules volés, les États et les organisations internationales ont progressivement mis en place des plateformes numériques dédiées à la déclaration, à la centralisation et au partage des informations relatives aux vols de véhicules.

% Ce chapitre présente un état de l’art des principales plateformes institutionnelles et privées utilisées pour la déclaration et la gestion des vols de véhicules, en mettant l’accent sur leurs objectifs, leurs fonctionnalités, leurs limites et leur complémentarité.

% \section{Plateformes nationales de déclaration}

% \subsection{Plateforme de la Direction Générale de la Police Républicaine (DGPR)}

% La Direction Générale de la Police Républicaine (DGPR) du Bénin propose une plateforme officielle permettant aux citoyens de déclarer la perte ou le vol de biens, y compris les véhicules. Cette plateforme constitue un outil essentiel pour l’enregistrement administratif et judiciaire des infractions.

% La déclaration effectuée via la DGPR permet l’ouverture d’une procédure officielle, l’inscription du véhicule dans les bases internes de la police et le déclenchement d’éventuelles enquêtes. Elle joue également un rôle fondamental dans les démarches auprès des compagnies d’assurance.

% Cependant, cette plateforme présente certaines limites, notamment une portée strictement nationale et une absence d’interconnexion directe avec les bases de données internationales accessibles au public.

% \subsection{Service Public France}

% Le portail Service-Public.fr centralise les démarches administratives des citoyens français, y compris la déclaration de vol de véhicule. Il permet une orientation claire vers les services compétents (police ou gendarmerie) et fournit des informations détaillées sur les procédures à suivre.

% Ce portail se distingue par son intégration avec les systèmes administratifs français et les compagnies d’assurance. Toutefois, il reste principalement informatif et administratif, sans offrir un accès direct aux bases de données internationales de véhicules volés.

% \subsection{Collectivité de Saint-Martin}

% La Collectivité de Saint-Martin propose également un service en ligne pour la déclaration de vol de véhicule. Cette plateforme locale s’inscrit dans une logique de proximité administrative, facilitant les démarches pour les résidents de la collectivité.

% Néanmoins, à l’instar des autres plateformes locales, son champ d’action reste limité au territoire concerné et dépend fortement des procédures manuelles pour la transmission des informations aux autorités compétentes.

% \section{Bases de données internationales institutionnelles}

% \subsection{Interpol – Stolen Motor Vehicles Database (SMV)}

% Interpol gère la base mondiale des véhicules volés, connue sous le nom de Stolen Motor Vehicles Database (SMV). Cette base de données est accessible exclusivement aux forces de l’ordre des pays membres et constitue l’un des outils les plus puissants de lutte contre le vol de véhicules à l’échelle internationale.

% La base SMV permet l’identification rapide des véhicules signalés volés lors des contrôles routiers, portuaires et frontaliers. Son caractère sécurisé garantit une forte fiabilité des données, mais limite l’accès direct aux citoyens.

% \subsection{Europol et le Schengen Information System (SIS II)}

% Le Schengen Information System de deuxième génération (SIS II), exploité par Europol, facilite le partage en temps réel des informations relatives aux véhicules volés entre les États membres de l’Union européenne. Ce système constitue un pilier de la coopération policière européenne.

% Le SIS II renforce considérablement la capacité des forces de l’ordre à détecter les véhicules volés circulant dans l’espace Schengen. Toutefois, son accès est strictement réservé aux autorités, excluant toute interaction directe avec les citoyens.

% \section{Plateformes privées et hybrides}

% \subsection{DIGITPOL – International Stolen Vehicle Database}

% DIGITPOL est une plateforme privée internationale permettant aux particuliers et aux organisations de signaler des véhicules volés dans une base accessible à l’échelle mondiale. Elle vise à accroître la visibilité des véhicules volés et à prévenir leur revente frauduleuse.

% Contrairement aux bases institutionnelles, DIGITPOL favorise l’implication citoyenne et la diffusion de l’information. Cependant, les signalements effectués sur cette plateforme n’ont pas de valeur légale directe et doivent être complétés par une déclaration officielle auprès des autorités compétentes.

% \section{Analyse comparative des plateformes}

% \subsection{Comparaison fonctionnelle}

% Les plateformes étudiées présentent des approches complémentaires. Les plateformes nationales privilégient l’aspect légal et administratif, tandis que les bases internationales institutionnelles se concentrent sur la coopération policière. Les plateformes privées, quant à elles, mettent l’accent sur la diffusion rapide de l’information et la participation citoyenne.

% \subsection{Tableau comparatif}

% \begin{table}[h!]
% \centering
% \renewcommand{\arraystretch}{1.4}
% \begin{tabular}{|p{4cm}|p{2.5cm}|p{2.5cm}|p{2.5cm}|p{2.5cm}|}
% \hline
% \textbf{Critères} & \textbf{DGPR} & \textbf{Interpol (SMV)} & \textbf{DIGITPOL} & \textbf{Europol (SIS II)} \\
% \hline
% Type de plateforme & Nationale & Internationale & Privée & Régionale \\
% \hline
% Accès citoyen & Oui & Non & Oui & Non \\
% \hline
% Accès forces de l’ordre & Oui & Oui & Partiel & Oui \\
% \hline
% Portée géographique & Nationale & Mondiale & Mondiale & UE \\
% \hline
% Valeur légale & Officielle & Officielle & Non officielle & Officielle \\
% \hline
% Interaction citoyenne & Déclaration & Aucune & Signalement actif & Aucune \\
% \hline
% Interopérabilité & Faible & Élevée & Moyenne & Élevée \\
% \hline
% \end{tabular}
% \caption{Comparaison des principales plateformes de gestion des vols de véhicules}
% \label{tab:comparaison_etat_art}
% \end{table}

% \section{Limites et perspectives}

% L’analyse des plateformes existantes met en évidence un manque d’intégration entre les systèmes nationaux, internationaux et privés. L’absence d’une plateforme unifiée accessible aux citoyens tout en étant connectée aux bases institutionnelles constitue une lacune importante, particulièrement dans les pays en développement.

% \section{Synthèse de l’état de l’art}

% L’état de l’art démontre que la lutte contre le vol de véhicules repose sur une complémentarité entre plateformes administratives, bases policières internationales et solutions privées. Toutefois, aucune plateforme ne répond de manière exhaustive aux besoins des citoyens et des autorités, ce qui justifie la proposition d’une solution intégrée présentée dans la suite de ce travail.






% \section{Introduction générale}

% Le vol de véhicules constitue une problématique majeure de sécurité publique, aussi bien dans les pays en développement que dans les pays industrialisés. Face à l’augmentation de la mobilité transfrontalière et à la sophistication des réseaux criminels, plusieurs plateformes numériques ont été mises en place afin de faciliter la déclaration des vols, le partage d’informations et la coopération entre les autorités compétentes.

% Cependant, la majorité de ces plateformes demeure centrée sur les forces de l’ordre et les démarches administratives, laissant une place limitée à la collaboration citoyenne et au suivi en temps réel des incidents. Cette section présente une analyse comparative des principales plateformes existantes, en mettant l’accent sur leur capacité à intégrer les citoyens comme acteurs actifs du processus de prévention et de résolution.



% \section*{Introduction}

% Le vol de véhicules représente une problématique majeure de sécurité publique et de gestion administrative dans de nombreux pays. Face à l’augmentation des réseaux criminels organisés et à la mobilité rapide des véhicules volés, les approches traditionnelles de déclaration et de recherche montrent leurs limites. Les systèmes numériques de gestion et de suivi des vols de véhicules apparaissent alors comme des outils essentiels pour améliorer la coordination entre les citoyens, les forces de l’ordre et les institutions publiques.

% Cette revue de littérature vise à analyser les concepts fondamentaux liés à la gestion et au suivi des vols de véhicules, à examiner les solutions existantes et à mettre en évidence les insuffisances qui justifient la mise en place d’une nouvelle approche collaborative.

