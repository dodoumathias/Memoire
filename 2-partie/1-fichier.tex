


\section{Introduction}
Dans cette section, nous allons aborder la modélisation et la conception du système en prenant en compte les différentes méthodologies et outils de modélisation disponibles. Nous justifierons le choix de la méthode UML, un standard largement utilisé dans la conception de systèmes informatiques, pour la modélisation de notre projet.

\section{UML (Unified Modeling Language)}
\gls{UML} est une méthode de modélisation orientée objet qui permet de représenter visuellement des systèmes complexes. Il est principalement utilisé pour modéliser les logiciels à travers des diagrammes illustrant les aspects statiques et dynamiques du système. UML se compose de plusieurs types de diagrammes, y compris les diagrammes de cas d’utilisation, les diagrammes de classes et les diagrammes de séquences.

\section{Choix de la méthode de modélisation}
Pour ce projet, nous avons opté pour l'utilisation de la méthode UML en raison de sa capacité à modéliser des systèmes orientés objet et de sa popularité dans le domaine du développement logiciel moderne. UML permet de créer des diagrammes  de cas d'utilisations, le diagramme de classes, des diagrammes de séquences  qui sont essentiels pour une bonne compréhension du système dans sa globalité.

\section{Modélisation UML}

\subsection{Outils de Modélisation}
Nous avons utilisé l’outil \textbf{\gls{Dia}}
pour réaliser les diagrammes de cas d’utilisation, de classes et de séquences.  
Il s’agit d’un exemple parmi plusieurs outils disponibles pour la modélisation UML, et il facilite la clarté de la documentation ainsi que la standardisation des modèles, contribuant à une meilleure compréhension et maintenance du système.

\subsection{Identification des acteurs du système}
L’identification des acteurs du système est une étape clé dans la modélisation d’un système basé sur UML. Un acteur représente un rôle joué par un utilisateur ou un autre système qui interagit avec le système à modéliser. Dans notre système, les acteurs principaux sont les citoyens, les administrateurs et la police. Ces acteurs interagiront avec le système pour déclarer un vol, signaler un véhicule retrouvé, accéder à des données ou effectuer des actions de sécurité.



\newpage
\subsection{ Diagramme de cas d’utilisation}
Le diagramme de cas d'utilisation est utilisé pour décrire les interactions entre les utilisateurs (acteurs) et le système. Chaque cas d'utilisation représente une fonctionnalité du système, et l’interaction entre l’acteur.


	% \textbf{\underline{le Diagramme de Cas d’Utilisation du système} }\\	
	
\begin{figure}[h]
\includegraphics[width = 16.5cm,height = 16cm]{Diagramme1.png}
\caption{le Diagramme de Cas d’Utilisation du système.}
 \label{fig:exemple_figure}
\end{figure}
%\vspace{0.5cm}



\begin{enumerate}
    \item Creer un compte\\
	\textbf{\underline{Description textuelle} }\\	
	\begin{itemize}
		\item L'utilisateur cré un compte en fournissant ces informations personnelles.
		\item L'application vérifie sur si le compte existe déjà.
		\item Si le compte existe déjà ,l'application affiche un message d'erreur
		\item Si le compte n'existe pas , l'application crée le compte en affichant un message de confirmation.  
	\end{itemize}
	\item Créer Declaration\\
	\textbf{\underline{Description textuelle} }\\
		\begin{itemize}			
			\item L'utilisateur cré une declaration en indiquant le type de la propriété volé ainsi que les informations disponibles qui l'identifient.
		\end{itemize}
	\item Signaler vehicule\\
	\textbf{\underline{Description textuelle} }\\
	\begin{itemize}
		\item L'utilisateur peut signaler la localisation d'un vehicule recherché.
	\end{itemize}
	\item Payer\\
	\textbf{\underline{Description textuelle} }\\
	\begin{itemize}
		\item Les utilisateurs devrons payer pour chaque declaration de vol.
	\end{itemize}
	\item Se renseigner\\
	\textbf{\underline{Description textuelle} }\\
	\begin{itemize}
		\item Les utilisateurs peuvent se renseigner par une discussion instantané au près de la police .
	\end{itemize}
	
	\item Consulter liste declaration\\
	\textbf{\underline{Description textuelle} }\\
	\begin{itemize}
		\item Les utilisateurs peuvent consulter la liste de toutes les declaration.
	\end{itemize}
	\item Gerer Declaration\\
	\textbf{\underline{Description textuelle} }\\
	\begin{itemize}
		\item L'orsqu'un vehicule déclaré est retrouvé , l'agent policier ou Administrateur modifie l'etat de la declaration concernée.
	\end{itemize}
	\item Gérer Rapport\\
	\textbf{\underline{Description textuelle} }\\
	\begin{itemize}
		\item Créer un rapport de vol quotidien contenant toutes les déclarations de vols effectuées.
	\end{itemize}
	\item Gérer compte\\
	\textbf{\underline{Description textuelle} }\\
	\begin{itemize}
		\item L'Administrateur ou le superadministrateur est responsable de la gestion des comptes utilisateurs et polices.Il peut suprimer , bloquer un compte.
	\end{itemize}
	\begin{itemize}
		\item Le superadministrateur est responsable de la gestion des administrateurs.Il peut ajouter , modifier et supprimer un administrateur.
	\end{itemize}
\end{enumerate}
\newpage

\subsection{Diagramme des classes}
Le diagramme de classes est utilisé pour décrire les objets du système et leurs relations. Dans ce diagramme, chaque classe représente une entité du système, et les relations entre ces classes (comme l'héritage, l'association, etc.) sont clairement indiquées. Par exemple, une classe "Véhicule" pourrait être liée à une classe "Propriétaire" avec une relation d'association.

\textbf{\underline{Le Diagramme de Classe du système} }
\begin{figure}[h]
\includegraphics[width = 17cm,height = 16cm]{classe.png}
\caption{Le Diagramme de Classe du système}
\label{fig:exemple_figure2}
\end{figure}
\vspace{0.5cm}
\begin{enumerate}
	\item Utilisateur\\
     	\textbf{\underline{Description textuelle} }\\
		\begin{itemize}
			\item La classe Utilisateur représente les différents types d'utilisateurs du système.
		\end{itemize}
		\item Declaration\\
	\textbf{\underline{Description textuelle} }\\
		\begin{itemize}
			\item La classe Declaration contient les informations sur les déclarations de vol, avec un lien vers les utilisateurs concernés.
		\end{itemize}
		\item Rapport\\
	\textbf{\underline{Description textuelle} }\\
		\begin{itemize}
			\item La classe Rapport regroupe les déclarations pour une date donnée, avec des listes pour les nouvelles, modifiées et résolues.
		\end{itemize}
		\item Police\\
	\textbf{\underline{Description textuelle} }\\
		\begin{itemize}
			\item La classe Police regroupe les differents types de force de l'ordre. 
		\end{itemize}
		\item Admin\\
	\textbf{\underline{Description textuelle} }\\
		\begin{itemize}
			\item La classe Admin représente les administrateurs du système. 
		\end{itemize}
	\end{enumerate}
\vspace{0.5cm}
Les liens entre les classes montrent les relations entre les différents éléments du système.\\
Ce diagramme de classe capture les principales entités et leurs interactions pour la gestion des cambriolages de véhicules dans ce système.\par
%\vspace{0.5cm}
\newpage





\subsection{Diagramme de séquences}
Le diagramme de séquences illustre l’ordre des messages échangés entre les objets du système pendant l’exécution d’un scénario particulier. Ce type de diagramme permet de comprendre comment les acteurs et le système interagissent au fil du temps pour accomplir une tâche donnée.  
Dans notre projet, les principaux cas d’utilisation représentés sont: déclarer un vol, diffuser une notice et alerter sur un véhicule recherché.

\textbf{\underline{Déclarer un vol:}}  
Le diagramme de séquence du scénario  Déclarer un vol  montre l’interaction entre l’utilisateur et le système.  
\begin{itemize}
    \item L’utilisateur remplit un formulaire en fournissant les informations sur le véhicule et les détails du vol (lieu, date, description).
    \item Le système vérifie la validité des données, enregistre la déclaration et associe l’information au compte de l’utilisateur.
    \item Une confirmation est ensuite envoyée à l’utilisateur, attestant que la déclaration a bien été prise en compte.
\end{itemize}

\begin{figure}[h]
    \centering
    \includegraphics[width=16cm, height=14.5cm]{sequence1.png}
    \caption{Diagramme de séquence pour le cas d’utilisation  Déclarer un vol .}
\end{figure}

\newpage

\textbf{\underline{Notice et diffusion:}}  
Le diagramme de séquence pour  Notice et diffusion  illustre le processus de gestion de l’information après la déclaration d’un vol.  
\begin{itemize}
    \item Le système génère automatiquement une notice contenant les informations essentielles sur le véhicule déclaré. La couleur de diffusion varie selon le statut de la déclaration : 
		\begin{itemize}
			\item \textbf{Rouge}: lorsque le véhicule est déclaré \textit{volé},  
			\item \textbf{Jaune}: lorsqu’un véhicule est \textit{signalé} mais pas encore retrouvé,  
			\item \textbf{Vert}: lorsque le véhicule est \textit{retrouvé}.  
		\end{itemize}

    \item Cette notice est diffusée en temps réel vers les utilisateurs de cette application web.
    % \item L’administrateur peut intervenir pour valider, modifier ou compléter les informations diffusées.
\end{itemize}

\begin{figure}[h]
    \centering
    \includegraphics[width=16cm, height=16cm]{sequence3.png}
    \caption{Diagramme de séquence pour le cas d’utilisation  Notice et diffusion .}
\end{figure}

\newpage

\textbf{\underline{Alerter sur un véhicule recherché :}}  
Le diagramme de séquence du scénario  Alerter sur un véhicule recherché  met en évidence la manière dont le système traite une alerte lorsqu’un utilisateur ou une autorité signale un véhicule suspect.  
\begin{itemize}
    \item L’utilisateur envoie une alerte avec des informations
    \item Le système compare les données reçues avec la base des véhicules déclarés volés.
    \item Si une correspondance est trouvée, une notification est envoyée aux polices et aux administrateurs.
    % \item L’administrateur peut ensuite initier des actions comme la confirmation de l’alerte ou la communication avec les forces de l’ordre.
\end{itemize}

\begin{figure}[h]
    \centering
    \includegraphics[width=16cm, height=15cm]{sequence2.png}
    \caption{Diagramme de séquence pour le cas d’utilisation  Alerter sur un véhicule recherché.}
\end{figure}


\newpage


\section{Choix Techniques}

Le développement du système de gestion et de déclaration de cambriolages de véhicules repose sur des choix technologiques adaptés aux besoins du projet. Ces choix sont guidés par des critères tels que la performance, la sécurité, la scalabilité et la maintenabilité.

% \subsection{ \gls{Langages} et  \gls{Frameworks} }
\subsection{\texorpdfstring{\gls{Langages} et \gls{Frameworks}}{Langages et Frameworks}}


% \subsubsection{\gls{Front-End}: ReactJS}
\subsubsection{\texorpdfstring{\gls{Front-End}: ReactJS}{Front-End: ReactJS}}

Pour l’interface utilisateur, nous avons choisi \textbf{ReactJS} \cite{reactjs}, une bibliothèque JavaScript permettant la création d’interfaces dynamiques et réactives.  
Ses principaux atouts sont : la réutilisabilité des composants, la réduction de la redondance du code et l’optimisation des performances grâce au \gls{DOM} virtuel.  
Ces caractéristiques améliorent l’expérience utilisateur et facilitent la maintenance.

\subsubsection{Back-End: NestJS avec GraphQL et Prisma}
Le \gls{Back-End} repose sur \textbf{NestJS} \cite{nestjs}, un framework modulaire basé sur \textbf{\gls{Node.js}} et \textbf{\gls{TypeScript}}, adapté aux applications évolutives.  
Nous avons adopté \textbf{GraphQL} \cite{graphql} avec l’approche \textbf{\gls{Code First}}, où le schéma est généré automatiquement à partir des classes TypeScript. Cela assure une forte cohérence entre le code et l’API.  

La gestion des données est assurée par \textbf{Prisma} \cite{prisma}, un \gls{ORM} moderne qui simplifie les opérations en base, génère automatiquement un client typé et facilite les migrations.  

\textbf{Avantages principaux :}  
\begin{itemize}
    \item Structure modulaire et maintenable avec NestJS
    \item Requêtes optimisées : GraphQL ne renvoie que les données nécessaires  
    \item Gestion simplifiée des données et typage strict grâce à Prisma
    \item Sécurité renforcée avec la validation et l’injection de dépendances  
\end{itemize}

\subsection{Base de Données}
Le projet utilise \textbf{PostgreSQL} \cite{postgresql}, un \gls{SGBDR} \gls{open-source} reconnu pour sa fiabilité et sa conformité aux standards SQL.  
Il offre une gestion avancée des transactions.

\subsection{Architecture du Système et Conteneurisation}
L’architecture adoptée est de type \textbf{\gls{client-serveur}} et conteneurisée avec \textbf{Docker} \cite{docker} :
\begin{itemize}
    \item \textbf{Front-end :} ReactJS communique avec le serveur via des requêtes et mutations GraphQL \cite{reactjs, graphql}.  
    \item \textbf{Back-end :} NestJS implémente l’API GraphQL (Code First) et interagit avec la base via Prisma \cite{nestjs, prisma}.  
    \item \textbf{Base de données :} PostgreSQL, assurant robustesse et intégrité des données \cite{postgresql}.  
    \item \textbf{Infrastructure :} Nginx agit comme \gls{reverse proxy} pour la gestion des requêtes HTTP/HTTPS et l’équilibrage de charge \cite{nginx}.  
    \item \textbf{Docker :} chaque composant (front-end, back-end, base de données) est isolé dans des conteneurs, garantissant portabilité, déploiement simplifié et cohérence entre les environnements \cite{docker}.
\end{itemize}

Cette architecture modulaire et conteneurisée garantit la scalabilité, la portabilité et une maintenance simplifiée.

\subsection{Sécurité}
La sécurité est un aspect central du projet, renforcé par plusieurs couches de protection :  
\begin{itemize}
    \item \textbf{Authentification sécurisée} avec \gls{JSON} Web Tokens (JWT) et expiration automatique des sessions
    \item \textbf{Chiffrement des communications} via \gls{HTTPS} et \gls{TLS}
    \item \textbf{Contrôle d’accès} basé sur les rôles et permissions granulaires
    \item \textbf{Hashage des mots de passe} avec bcrypt \cite{bcrypt} et salage renforcé
    \item \textbf{Sécurité des conteneurs Docker} : isolation des services, limitation des privilèges et mise à jour régulière des images
    \item \textbf{Protection contre les injections et attaques} : validation stricte des entrées utilisateur, prévention des injections \gls{SQL} et \gls{XSS}
    \item \textbf{Journalisation et audit} : traçabilité de toutes les actions sensibles pour détecter les anomalies
\end{itemize}

Ces mesures assurent la confidentialité, l’intégrité et la disponibilité des données.


\subsection{Avantages des Choix Techniques}
\begin{itemize}
    \item Interfaces dynamiques et performantes grâce à ReactJS
    \item \gls{API} robuste et typée avec NestJS, \gls{GraphQL} et Prisma
    \item Données fiables et extensibles via PostgreSQL
    \item Sécurité renforcée par JWT, HTTPS, \gls{bcrypt} et bonnes pratiques Docker
    \item Architecture modulaire, scalable et conteneurisée avec Docker et \gls{Nginx}
    \item Documentation claire grâce à UML
\end{itemize}

\subsection{Limites}
\begin{itemize}
    \item \textbf{Complexité technique} nécessitant une expertise en JavaScript/TypeScript, NestJS et Prisma  
    \item \textbf{Courbe d’apprentissage élevée} pour GraphQL, Docker et l’architecture sécurisée  
    \item \textbf{Consommation de ressources} plus importante avec PostgreSQL et Docker que des solutions légères
\end{itemize}


\section{Conclusion}
L'utilisation combinée des technologies modernes telles que \gls{ReactJS}, \gls{NestJS}, \gls{PostgreSQL}, \gls{Prisma} et \gls{Docker}, ainsi que des bonnes pratiques de sécurité, nous permet de concevoir un système robuste et performant pour la gestion des cambriolages de véhicules. Ce système facilitera la collaboration entre les utilisateurs, les administrateurs et la police, tout en garantissant la sécurité, la fiabilité et la portabilité des données traitées.






































































































































% \section{Conclusion}
% L’utilisation combinée des technologies modernes telles que \gls{react}, \gls{nestjs}, \gls{postgresql}, \gls{prisma} et \gls{docker}, associée à l’adoption de bonnes pratiques de sécurité, a permis de concevoir un système robuste et performant dédié à la gestion des cambriolages de véhicules.  
% Ce système favorise la collaboration entre les utilisateurs, les administrateurs et les forces de l’ordre, tout en garantissant la sécurité, la fiabilité et la portabilité des données traitées.

% Le frontend est développé avec \gls{react}.
% Le backend utilise \gls{nestjs} et \gls{postgresql}.
% Le déploiement est réalisé avec \gls{docker}.




















% \section{Introduction}
% Dans cette section, nous allons aborder la modélisation et la conception du système en prenant en compte les différentes méthodologies et outils de modélisation disponibles. Nous comparerons deux approches populaires, à savoir UML et MERISE, et justifierons le choix de la méthode la plus adaptée pour le projet. Ensuite, nous nous pencherons sur la mise en place de la modélisation UML, un standard largement utilisé dans la conception de systèmes informatiques.


% \section{Comparaison entre UML et MERISE}
% UML (Unified Modeling Language) et MERISE sont deux méthodologies populaires utilisées pour la modélisation des systèmes d’information. Cependant, elles diffèrent dans leur approche et leur utilisation.

% \subsection*{UML}
% UML est une méthode de modélisation orientée objet qui permet de représenter visuellement des systèmes complexes. Il est principalement utilisé pour modéliser les logiciels à travers des diagrammes qui illustrent les aspects statiques et dynamiques du système. UML se compose de plusieurs types de diagrammes, y compris les diagrammes de cas d’utilisation, les diagrammes de classes, et les diagrammes de séquences.

% \subsection*{MERISE}
% MERISE, quant à lui, est une méthode de modélisation orientée processus et données. Elle est principalement utilisée dans le cadre de la conception de bases de données et de systèmes d'information en général. Elle distingue trois niveaux :stratégique, conceptuel et logique. MERISE met l'accent sur la structuration des données et leur organisation dans le cadre de processus métiers.

% \subsection*{Comparaison}
% UML est plus flexible et centré sur les objets, ce qui est un avantage lorsqu'il s'agit de modéliser des systèmes informatiques complexes. MERISE, bien qu'il soit un peu plus ancien, est souvent préféré dans des contextes où l'organisation des données est primordiale. Le choix entre les deux méthodologies dépendra du type de projet et de la nature des exigences du système à modéliser.

% \section{ Choix de la méthode de modélisation}
% Après avoir comparé les deux méthodologies de modélisation, nous avons opté pour l'utilisation de la méthode UML pour ce projet. Cela est dû à sa capacité à modéliser des systèmes orientés objet et à sa popularité dans le domaine du développement logiciel moderne. UML permet de créer des diagrammes de classes, des diagrammes de séquences, et des cas d’utilisation qui sont essentiels pour une bonne compréhension du système dans sa globalité.

% \section{ Modélisation UML}
% \subsection{ Identification des acteurs du système}
% L’identification des acteurs du système est une étape clé dans la modélisation d’un système basé sur UML.Un acteur représente un rôle joué par un utilisateur ou un autre système qui interagit avec le système à modéliser. Dans notre système, les acteurs principaux sont les utilisateurs, les administrateurs et la police. Ces acteurs interagiront avec le système pour déclarer un vol, signaler un vehicule retrouvé, accéder à des données ou effectuer des actions de sécurité.




% \section{Choix Techniques}

% Le développement du système de gestion et de déclaration de cambriolages de véhicules repose sur des choix technologiques adaptés aux besoins du projet. Ces choix sont guidés par des critères tels que la performance, la sécurité, la scalabilité et la maintenabilité.

% \subsection{Langages et Frameworks}

% \subsubsection{Front-End: ReactJS}
% Pour l’interface utilisateur, nous avons choisi \textbf{ReactJS}, une bibliothèque JavaScript permettant la création d’interfaces dynamiques et réactives.  
% Ses principaux atouts sont: la réutilisabilité des composants, la réduction de la redondance du code et l’optimisation des performances grâce au DOM virtuel.  
% Ces caractéristiques améliorent l’expérience utilisateur et facilitent la maintenance.

% \subsubsection{Back-End: NestJS avec GraphQL et Prisma}
% Le back-end repose sur \textbf{NestJS}, un framework modulaire basé sur \textbf{Node.js} et \textbf{TypeScript}, adapté aux applications évolutives.  
% Nous avons adopté \textbf{GraphQL} avec l’approche \textbf{Code First}, où le schéma est généré automatiquement à partir des classes TypeScript. Cela assure une forte cohérence entre le code et l’API (Application Programming Interface).  

% La gestion des données est assurée par \textbf{Prisma}, un ORM moderne qui simplifie les opérations en base, génère automatiquement un client typé et facilite les migrations.  

% \textbf{Avantages principaux:}  
% \begin{itemize}
%     \item Structure modulaire et maintenable avec NestJS
%     \item Requêtes optimisées: GraphQL ne renvoie que les données nécessaires  
%     \item Gestion simplifiée des données et typage strict grâce à Prisma
%     \item Sécurité renforcée avec la validation et l’injection de dépendances  
% \end{itemize}

% La combinaison NestJS–GraphQL–Prisma constitue une solution moderne et robuste pour la construction d’API performantes.

% \subsection{Base de Données}
% Le projet utilise \textbf{PostgreSQL}, un SGBDR open-source reconnu pour sa fiabilité et sa conformité aux standards SQL.  
% Il offre une gestion avancée des transactions et, avec l’extension PostGIS, permet l’intégration de fonctionnalités géospatiales, utiles pour localiser les véhicules volés.

% \subsection{Architecture du Système}
% L’architecture adoptée est de type \textbf{client-serveur}:
% \begin{itemize}
%     \item \textbf{Front-end:} ReactJS communique avec le serveur via des requêtes et mutations GraphQL.  
%     \item \textbf{Back-end:} NestJS implémente l’API GraphQL (Code First) et interagit avec la base via Prisma.  
%     \item \textbf{Base de données:} PostgreSQL, assurant robustesse et intégrité des données.  
%     \item \textbf{Infrastructure:} Nginx agit comme reverse proxy pour la gestion des requêtes HTTP/HTTPS et l’équilibrage de charge.  
% \end{itemize}

% Cette architecture modulaire garantit la scalabilité et une maintenance simplifiée.

% \subsection{Sécurité}
% La sécurité est un aspect central du projet, avec les mesures suivantes:  
% \begin{itemize}
%     \item \textbf{Authentification sécurisée} avec JSON Web Tokens (JWT)
%     \item \textbf{Chiffrement des communications} via HTTPS
%     \item \textbf{Contrôle d’accès} basé sur les rôles
%     \item \textbf{Hashage des mots de passe} avec bcrypt
% \end{itemize}
% Ces mécanismes assurent la confidentialité et l’intégrité des données.

% \subsection{Outils de Modélisation}
% La conception a été réalisée en \textbf{UML}, avec l’outil \textbf{StarUML} pour les diagrammes de cas d’utilisation, de classes et de séquences.  
% Cet outil facilite la documentation et la standardisation des modèles.

% \subsection{Avantages des Choix Techniques}
% \begin{itemize}
%     \item Interfaces dynamiques et performantes grâce à ReactJS
%     \item API robuste et typée avec NestJS, GraphQL et Prisma
%     \item Données fiables et extensibles via PostgreSQL
%     \item Sécurité renforcée par JWT, HTTPS et bcrypt
%     \item Architecture modulaire et scalable avec Nginx
%     \item Documentation claire grâce à UML
% \end{itemize}

% \subsection{Limites}
% \begin{itemize}
%     \item \textbf{Complexité technique} nécessitant une expertise en JavaScript/TypeScript, NestJS et Prisma.  
%     \item \textbf{Courbe d’apprentissage élevée} pour GraphQL et l’architecture sécurisée.  
%     \item \textbf{Consommation de ressources} plus importante avec PostgreSQL que des solutions légères comme SQLite.  
% \end{itemize}

% \section{Conclusion}
% L'utilisation combinée des méthodologies MERISE et UML, ainsi que des technologies modernes telles que ReactJS, NestJS, PostgreSQL, Prisma et Docker, nous permet de concevoir un système robuste et performant pour la gestion des cambriolages de véhicules. Ce système facilitera la collaboration entre les utilisateurs, les administrateurs et la police, tout en garantissant la sécurité et la fiabilité des données traitées.

% \section{Perspectives et Améliorations Futures}

% Les perspectives futures de la lutte contre le vol et le cambriolage de véhicules reposent sur plusieurs axes technologiques et organisationnels majeurs :

% \begin{itemize}
%     \item \textbf{Intégration de l’intelligence artificielle (IA)} : utilisation d’algorithmes d’apprentissage automatique pour analyser les données issues des caméras, capteurs et plateformes de signalement, afin de détecter automatiquement les comportements suspects et de prédire les zones à haut risque de vol.

%     \item \textbf{Systèmes de surveillance intelligents} : déploiement de caméras intelligentes capables d’effectuer une reconnaissance automatique des plaques d’immatriculation et d’identifier des scénarios de cambriolage en temps réel.

%     \item \textbf{Généralisation de l’Internet des Objets (IoT)} : intégration de capteurs connectés (détecteurs de mouvement, d’ouverture de portières, de bris de vitre) dans les véhicules pour une surveillance continue et l’envoi d’alertes instantanées aux propriétaires et aux forces de l’ordre.

%     \item \textbf{Plateformes hybrides public–privé} : renforcement de la collaboration entre les forces de sécurité, les compagnies d’assurance, les collectivités locales et les acteurs privés afin d’améliorer le partage et l’exploitation des informations.

%     \item \textbf{Interconnexion des bases de données} : mise en réseau des bases de données nationales et internationales pour faciliter la traçabilité des véhicules volés et accélérer les procédures de récupération.

%     \item \textbf{Utilisation de la blockchain} : sécurisation des données de signalement des vols grâce à des registres distribués garantissant l’intégrité, la traçabilité et la transparence des informations.

%     \item \textbf{Applications mobiles et systèmes collaboratifs} : développement de solutions participatives permettant aux citoyens de signaler rapidement les incidents et de contribuer activement à la prévention du cambriolage de véhicules.

% \end{itemize}