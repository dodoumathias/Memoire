% ---------- Glossaire principal ----------

\newglossaryentry{acroglo}{
    name={acroglo},
    description={Terme apparaissant à la fois dans le glossaire et les acronymes}
}

\newglossaryentry{glossaire}{
    name={Glossaire},
    description={Liste des termes techniques utilisés dans ce document}
}
% Fichier : glossaire_reduit.tex

\newglossaryentry{reactjs}{
    name=ReactJS,
    description={Bibliothèque JavaScript pour la création d'interfaces utilisateur dynamiques}
}

\newglossaryentry{nestjs}{
    name=NestJS,
    description={Framework Node.js pour applications serveur modulaires et évolutives}
}

\newglossaryentry{graphql}{
    name=GraphQL,
    description={Langage de requête et runtime pour API}
}

\newglossaryentry{prisma}{
    name=Prisma,
    description={ORM moderne pour Node.js et TypeScript}
}

\newglossaryentry{postgresql}{
    name=PostgreSQL,
    description={Système de gestion de base de données relationnelle open-source}
}

\newglossaryentry{docker}{
    name=Docker,
    description={Plateforme de conteneurisation permettant de déployer des applications isolées}
}

\newglossaryentry{nginx}{
    name=Nginx,
    description={Serveur web et reverse proxy pour la gestion des requêtes HTTP/HTTPS}
}

\newglossaryentry{jwt}{
    name=JWT,
    description={JSON Web Token pour authentification sécurisée}
}

\newglossaryentry{bcrypt}{
    name=Bcrypt,
    description={Algorithme de hashage des mots de passe}
}

\newglossaryentry{frontend}{
    name=Front-End,
    description={Partie visible et interactive d'une application}
}

\newglossaryentry{backend}{
    name=Back-End,
    description={Partie serveur et logique métier d'une application}
}

\newglossaryentry{api}{
    name=API,
    description={Interface de programmation pour communiquer entre composants}
}


\newglossaryentry{https}{
    name=HTTPS/TLS,
    description={Protocoles de communication sécurisée pour le web}
}

\newglossaryentry{conteneurisation}{
    name=Conteneurisation,
    description={Technique d’isolation des applications dans des conteneurs}
}

\newglossaryentry{scalabilite}{
    name=Scalabilité,
    description={Capacité d’un système à gérer une augmentation de charge}
}

\newglossaryentry{modularite}{
    name=Modularité,
    description={Conception d’un système en composants indépendants}
}

\newglossaryentry{maintenabilite}{
    name=Maintenabilité,
    description={Facilité de maintenance et d’évolution d’un système}
}

\newglossaryentry{UML}{
    name=UML,
    description={Unified Modeling Language, méthode de modélisation orientée objet}
}
\newglossaryentry{Dia}{
  name=Dia,
  description={Outil pour réaliser les diagrammes UML}
}

\newglossaryentry{Front-End}{
    name=Front-End,
    description={Partie visible par l'utilisateur d'une application, interface utilisateur}
}

\newglossaryentry{Back-End}{
    name=Back-End,
    description={Partie serveur qui gère la logique métier et la base de données}
}

\newglossaryentry{back-end}{
    name={back-end},
    description={The server-side part of a software application}
}


\newglossaryentry{Langages}{
    name=Langages,
    description={Langages de programmation utilisés dans le projet, comme JavaScript et TypeScript}
}

\newglossaryentry{Frameworks}{
    name=Frameworks,
    description={Bibliothèques ou environnements facilitant le développement, comme ReactJS ou NestJS}
}
\newglossaryentry{DOM}{
    name={DOM},
    description={Document Object Model, structure représentant le contenu HTML d'une page et manipulable via JavaScript}
}

\newglossaryentry{TypeScript}{
    name={TypeScript},
    description={A typed superset of JavaScript that compiles to plain JavaScript}
}
\newglossaryentry{ReactJS}{
  name=ReactJS,
  description={Bibliothèque JavaScript pour la création d’interfaces utilisateur}
}

\newglossaryentry{NestJS}{
  name=NestJS,
  description={Framework Node.js pour le développement d’API côté serveur}
}

\newglossaryentry{PostgreSQL}{
  name=PostgreSQL,
  description={Système de gestion de base de données relationnelle open-source}
}

\newglossaryentry{Prisma}{
  name=Prisma,
  description={ORM moderne pour Node.js et TypeScript}
}

\newglossaryentry{Docker}{
  name=Docker,
  description={Plateforme de conteneurisation permettant le déploiement d’applications}
}

\newglossaryentry{HTTPS}{
  name=HTTPS,
  description={Protocole de communication sécurisé basé sur HTTP et TLS}
}

\newglossaryentry{TLS}{
  name=TLS,
  description={Protocole de chiffrement assurant la confidentialité et l’intégrité des échanges}
}

\newglossaryentry{API}{
  name=API,
  description={Interface de programmation permettant la communication entre applications}
}

\newglossaryentry{SQL}{
  name=SQL,
  description={Langage de requêtes utilisé pour interagir avec des bases de données relationnelles}
}

\newglossaryentry{XSS}{
  name=XSS,
  description={Attaque par injection de scripts malveillants dans des pages web}
}

\newglossaryentry{GraphQL}{
  name=GraphQL,
  description={Langage de requête pour API permettant de demander uniquement les données nécessaires}
}

\newglossaryentry{Nginx}{
  name=Nginx,
  description={Serveur web et reverse proxy haute performance}
}

\newglossaryentry{JWT}{
  name=JWT,
  description={Jeton d’authentification sécurisé au format JSON}
}

\newglossaryentry{SGBDR}{
  name=SGBDR,
  description={Système de Gestion de Base de Données Relationnelle}
}

\newglossaryentry{nodejs}{
  name=Node.js,
  description={Environnement d’exécution JavaScript côté serveur basé sur le moteur V8}
}

\newglossaryentry{typescript}{
  name=TypeScript,
  description={Sur-ensemble typé de JavaScript améliorant la robustesse et la maintenabilité du code}
}

\newglossaryentry{codefirst}{
  name=Code First,
  description={Approche de conception où le schéma de l’API est généré automatiquement à partir du code source}
}

\newglossaryentry{Node.js}{
    name={Node.js},
    description={Environnement d'exécution JavaScript côté serveur}
}

\newglossaryentry{Code First}{
    name={Code First},
    description={Une approche de développement où le schéma est généré à partir du code plutôt que défini manuellement}
}

\newglossaryentry{JSON}{
    name={JSON},
    description={JavaScript Object Notation, un format léger pour l’échange de données}
}

\newglossaryentry{ORM}{
    name={ORM},
    description={Object-Relational Mapping, un outil qui facilite l'interaction entre le code et la base de données}
}

\newglossaryentry{reverse proxy}{
    name={reverse proxy},
    description={Serveur intermédiaire qui reçoit les requêtes des clients et les redirige vers un ou plusieurs serveurs internes, souvent utilisé pour la sécurité, la répartition de charge ou le cache}
}
\newglossaryentry{client-serveur}{
    name={client-serveur},
    description={Modèle informatique dans lequel un client demande des services à un serveur, qui les fournit}
}
\newglossaryentry{open-source}{
    name={open-source},
    description={Logiciel dont le code source est librement accessible, modifiable et redistribuable}
}
