\conclusion

La gestion des cambriolages de véhicules nécessite une approche proactive qui intègre à la fois la technologie et l'engagement des communautés. Le système d'alerte que nous proposons, fondé sur une application web, représente une solution innovante permettant une réactivité accrue face aux actes de vol. En permettant aux citoyens, aux forces de l'ordre et aux administrateurs d'interagir efficacement, ce système favorise un environnement plus sécurisé.

Les résultats obtenus lors des tests pilotes montrent une amélioration significative de la rapidité de réaction et de la couverture des alertes. Cependant, des recherches futures devraient se concentrer sur l'amélioration continue du système, notamment par l'intégration de nouvelles technologies, comme l'intelligence artificielle ou les dispositifs de surveillance plus avancés. De plus, l'adaptation du système aux besoins spécifiques des différentes régions et l'extension de sa portée géographique sont essentielles pour maximiser son efficacité.

Ainsi, bien que des progrès aient été réalisés, il est crucial de maintenir un processus d'innovation et d'adaptation pour faire face aux évolutions constantes des menaces liées à la sécurité des véhicules. Le système d'alerte communautaire offre une base solide pour une collaboration accrue entre les citoyens, les autorités locales et les technologistes, contribuant ainsi à un renforcement global de la sécurité publique.
\cite{ehrig2006graph}