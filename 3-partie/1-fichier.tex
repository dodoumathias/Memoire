% \addcontentsline{toc}{section}{Introduction}
% \section*{Introduction}

% \section{-}
% blablabla

% \section{-}
% \begin{algorithm}
% 	\KwData{$x$}
% 	\KwResult{$r$}
% 	\Begin{
% 		\If{$x \neq 0$}{
% 			$ r \leftarrow 1/x$\;
% 		}
% 	}
	
% 	\caption{Inverse}\label{alg:Inverse}
% \end{algorithm}

% \addcontentsline{toc}{section}{Conclusion}
% \section*{Conclusion}





% \subsection{Introduction}
% Les cambriolages de véhicules sont des infractions qui nuisent à la sécurité publique, et leur gestion nécessite une approche organisée impliquant différentes parties prenantes. Cette gestion englobe les dispositifs de sécurité, la surveillance, mais aussi la coopération entre les citoyens, les forces de l'ordre, les autorités locales et des acteurs privés. Le rôle des différents utilisateurs de systèmes de gestion de sécurité, tels que les administrateurs, la police et les superadmins, est crucial pour assurer une surveillance effective et une réponse rapide aux incidents de cambriolage. Ce chapitre explore les résultats obtenus concernant l'intégration de ces rôles dans les stratégies de prévention et de réponse aux cambriolages de véhicules.

% \subsection{Présentation des mesures de gestion des cambriolages}

% \subsubsection{Mise en place de dispositifs de sécurité renforcés}
% Les dispositifs de sécurité pour les véhicules, comme les systèmes d'alarme, les GPS de géolocalisation, et les dispositifs de verrouillage intelligents, sont essentiels pour prévenir les cambriolages. Ces technologies sont intégrées dans un cadre de gestion qui implique plusieurs acteurs : les propriétaires de véhicules, les forces de l'ordre, ainsi que les administrateurs des systèmes de sécurité.

% Les utilisateurs finaux, comme les propriétaires de véhicules, peuvent s'abonner à ces systèmes de sécurité et les configurer pour obtenir des alertes en temps réel via leurs smartphones ou plateformes en ligne. Ils peuvent signaler toute activité suspecte, soit manuellement, soit automatiquement en cas de tentative de vol, ce qui déclenche une intervention rapide.

% \subsubsection{Surveillance vidéo et patrouilles de sécurité}
% La surveillance par caméras, combinée à des patrouilles de sécurité, est une méthode efficace pour prévenir les cambriolages. Les caméras installées dans des zones sensibles telles que les parkings, les rues commerçantes ou les résidences sont contrôlées par des administrateurs de sécurité. Ces administrateurs supervisent en temps réel les images et détectent toute activité suspecte.

% Les utilisateurs tels que les policiers, les agents de sécurité, ou les superadmins peuvent recevoir des alertes automatiques en cas de détection de comportement anormal, ce qui leur permet d'agir rapidement. Les policiers, en particulier, jouent un rôle crucial dans l'analyse des vidéos et l'identification des criminels, souvent en collaboration avec des systèmes de géolocalisation GPS pour retracer les véhicules volés.

% \subsubsection{Sensibilisation des citoyens à la prévention}
% La prévention des cambriolages passe aussi par une stratégie de sensibilisation. Les administrateurs d'un système de gestion de sécurité jouent un rôle clé dans la diffusion d’informations et de conseils de prévention à travers des plateformes en ligne, des applications mobiles ou des campagnes de sensibilisation locales.

% Les citoyens peuvent s'inscrire sur des plateformes sécurisées où ils reçoivent des conseils personnalisés concernant la sécurité de leurs véhicules. De plus, des alertes locales peuvent être envoyées pour signaler des augmentations des cambriolages dans certaines zones. Les utilisateurs finaux peuvent également signaler des comportements suspects via ces plateformes, renforçant ainsi la vigilance communautaire.

% \subsubsection{Collaboration entre la police, les autorités locales et les citoyens}
% L’implication des forces de l'ordre, des administrateurs et des citoyens dans un système de gestion partagé est essentielle. La police, en tant qu'utilisateur ayant un accès direct aux informations de surveillance et aux alertes générées par le système de sécurité, peut intervenir de manière proactive. Les agents de la police peuvent être formés pour utiliser ces outils afin de surveiller les zones sensibles et de répondre plus rapidement aux incidents.

% Les superadmins ont un rôle de supervision générale. Ils sont responsables de la gestion de l’ensemble du système de sécurité, de la gestion des utilisateurs, et de la mise en œuvre de protocoles de sécurité avancés. En cas de situation critique, les superadmins peuvent superviser les actions des policiers, des agents de sécurité et des administrateurs pour assurer une réponse coordonnée.

% Les administrateurs, quant à eux, ont un rôle plus opérationnel et de gestion quotidienne. Ils gèrent l’installation et la configuration des dispositifs de sécurité, supervisent les alertes et veillent à ce que les utilisateurs finaux (les citoyens) suivent les protocoles de sécurité recommandés.

% \subsection{Discussion}
% L’intégration des différents rôles d’utilisateurs dans la gestion des cambriolages de véhicules permet de créer un environnement sécurisé et réactif. Les superadmins et administrateurs jouent un rôle de coordination technique et stratégique, tandis que les policiers, en tant qu’utilisateurs avec un accès privilégié aux informations critiques, sont en première ligne pour répondre aux incidents.

% Les résultats montrent que la coopération entre les utilisateurs finaux, les forces de l'ordre et les administrateurs peut considérablement améliorer la gestion des cambriolages de véhicules. Les systèmes de surveillance intégrés, lorsqu’ils sont associés à des patrouilles physiques et à une participation active de la communauté, augmentent les chances de détection et d'intervention avant que le dommage ne soit trop important. La gestion proactive des zones sensibles et l’analyse des données recueillies (comme les informations GPS ou vidéo) permettent de cibler les interventions et de renforcer la sécurité dans les zones à haut risque.

% La répartition claire des rôles et responsabilités, avec des actions coordonnées entre la police, les administrateurs et les superadmins, constitue une approche efficace pour réduire les cambriolages. En outre, la mise en place d'un système de gestion flexible et évolutif permet d'adapter les stratégies de sécurité aux nouvelles tendances criminelles, tout en répondant aux besoins spécifiques des utilisateurs.















\section*{Introduction}
Cette structure complète permet d'organiser efficacement l'application dédiée à la gestion des cambriolages tout en offrant une expérience utilisateur fluide et intuitive. Chaque interface a un rôle précis et contribue à l'objectif global d'amélioration de la sécurité et à la communication entre citoyens et forces de l'ordre.


\section{Presentation de l'application}
Du nom AlerteCar, elle permet de renforcer la sécurité publique en facilitant la gestion des cambriolages et des signalements de vols. Elle offre des outils adaptés pour les citoyens, les policiers et les administrateurs, avec une interface claire et un accès rapide aux services essentiels. L’objectif est de promouvoir une communication efficace entre les citoyens et les forces de l'ordre.


\section{Page d'Accueil}

\begin{center}
    \includegraphics[width=0.8\textwidth]{3-partie/accueil1.png} 
    \includegraphics[width=0.8\textwidth]{3-partie/accueil2.png} 
\end{center}



\textbf{Description:} La page d'accueil est la porte d'entrée de l'application. Elle offre une présentation générale et permet un accès rapide aux principales sections. L'objectif est de fournir une navigation claire dès l'arrivée sur l'application.



\textbf{Éléments:}
\begin{itemize}
    \item \textbf{Logo de l'application:} En haut à gauche pour une identification immédiate.
    \item \textbf{Menu de navigation:} Inclut des liens vers les principales pages comme l'Inscription, la Connexion, les services, les conseils, et l'à propos.
    \item \textbf{Informations sur les fonctionnalités principales:} Présentation succincte des services proposés par l'application ( declaration d'un cambriolage ,recherche de cambriolages, signalement d'un vehicule volé, etc.).
    % \item \textbf{Liens vers des ressources utiles:} Accès à la FAQ, le contact pour assistance, les conditions d'utilisation, etc.
\end{itemize}

\section{Inscription Utilisateur}
\textbf{Description:} Ce formulaire permet aux citoyens de s'inscrire pour utiliser l'application, en accédant à ces fonctionnalités.

\newpage
\begin{center}
    \includegraphics[width=0.8\textwidth]{3-partie/inscription1.png} 
    \includegraphics[width=0.8\textwidth]{3-partie/inscription2.png} 
\end{center}

\textbf{Éléments:}
\begin{itemize}
    \item \textbf{Champs:}NPI, Nom, prénom,tel , adresse ,e-mail, mot de passe.
    \item \textbf{Bouton "S'inscrire":} Soumet l'inscription.
    \item \textbf{Se connecter:} Si l'utilisateur est déjà inscrit, il peut se connecter .
\end{itemize}

% \section{Inscription Policier}
% \textbf{Description:} Formulaire destiné aux agents de police pour accéder à des fonctionnalités spécifiques, comme la gestion des rapports de cambriolage.

% \textbf{Éléments:}
% \begin{itemize}
%     \item \textbf{Champs:} Nom, prénom, numéro de badge, adresse e-mail, mot de passe.
%     \item \textbf{Bouton "S'inscrire":} Soumet l'inscription.
%     \item \textbf{Vérification des autorisations:} Processus de validation de l'inscription d'un agent de police, avec vérification de son numéro de badge.
% \end{itemize}

\section{Page de Connexion}

\begin{center}
    \includegraphics[width=0.8\textwidth]{3-partie/connexion1.png} 
    \includegraphics[width=0.8\textwidth]{3-partie/connexion2.png} 
\end{center}

\textbf{Description:} Interface permettant aux utilisateurs (citoyens ou policiers) de se connecter à leur compte pour accéder aux fonctionnalités spécifiques.

\textbf{Éléments:}
\begin{itemize}
    \item \textbf{Champs:} Adresse e-mail et le mot de passe.
    \item \textbf{Bouton "Se connecter":} Valide la demande de connexion.
    \item \textbf{Lien vers la récupération de mot de passe:} En cas d'oubli du mot de passe, l'utilisateur peut le réinitialiser.
\end{itemize}

\section{Interface Accueil utilisateur}

\begin{center}
    \includegraphics[width=0.8\textwidth]{3-partie/useraccueil.png} 
     \includegraphics[width=0.8\textwidth]{3-partie/connexion2.png} 
\end{center}

\textbf{Description:} Page où les utilisateurs peuvent gérer leurs informations personnelles, y compris les paramètres de sécurité.

\textbf{Éléments:}
\begin{itemize}
    \item \textbf{Affichage des informations personnelles:} Nom,prénom,tel, adresse ,e-mail .
    \item \textbf{Options pour modifier les informations personnelles:} Permet de mettre à jour les informations et de changer le mot de passe.
\end{itemize}



\section{Interface Déclaration de Vol}

\begin{center}
    \includegraphics[width=0.8\textwidth]{3-partie/declarer1.png} 
    \includegraphics[width=0.8\textwidth]{3-partie/declarer2.png} 
    \includegraphics[width=0.8\textwidth]{3-partie/connexion2.png} 
\end{center}

% \textbf{Description:} Permet aux utilisateurs de signaler un vol ou un cambriolage qu’ils ont observé ou subi.

% \textbf{Éléments:}
% \begin{itemize}
%     \item \textbf{Formulaire de déclaration:} Champ pour la date et l'heure du vol, le lieu, la description des objets volés.
%     \item \textbf{Option pour ajouter une photo:} Permet de télécharger une photo de l'incident ou des objets volés.
% \end{itemize}


\begin{itemize}
    \item \textbf{Formulaire de déclaration de cambriolage} : Un formulaire simple où l'utilisateur peut entrer les détails du cambriolage, comme la date, le lieu.
    \item \textbf{Suivi des déclarations} : Affichage de l'état actuel des déclarations effectuées par l'utilisateur avec des indicateurs visuels de statut ("Voler", "Retrouver").
\end{itemize}


\section{Interface Signalements}

\begin{center}
    \includegraphics[width=0.8\textwidth]{3-partie/signaler.png} 
\end{center}


\textbf{Description:} Permet aux utilisateurs ou policiers de signaler un véhicule retrouvé, potentiellement lié à un vol.

\textbf{Éléments:}
\begin{itemize}
    \item \textbf{Formulaire de signalement:} Détails du véhicule (marque, modèle, numéro d'immatriculation).
    \item \textbf{Option pour ajouter une photo:} Permet d’ajouter une photo du véhicule retrouvé.
\end{itemize}


\section{Interface Statistiques}

\begin{center}
    \includegraphics[width=0.8\textwidth]{3-partie/statistique1.png} 
     \includegraphics[width=0.8\textwidth]{3-partie/statistique2.png} 
\end{center}

\textbf{Description :} 
Page permettant aux utilisateurs de consulter des statistiques détaillées sur les déclarations de vol de véhicules. 
Les informations présentées incluent notamment : 
\begin{itemize}
    \item le \textbf{nombre total de véhicules retrouvés},  
    \item le \textbf{nombre total de véhicules non retrouvés},  
    \item la répartition des vols par \textbf{zones géographiques} (endroits où les vols sont les plus fréquents),  
    \item l’évolution du nombre de vols par \textbf{année}, \textbf{mois}, \textbf{semaine} et \textbf{jour},  
    \item des \textbf{graphiques comparatifs} facilitant la visualisation des tendances et des pics de criminalité,  
    % \item des \textbf{indicateurs clés} pour identifier les périodes et zones à haut risque.  
\end{itemize}


\section{Interface Profil}

\begin{center}
    \includegraphics[width=0.8\textwidth]{3-partie/profil.png} 
     \includegraphics[width=0.8\textwidth]{3-partie/connexion2.png} 
\end{center}

\textbf{Description:} Page où les utilisateurs peuvent gérer leurs informations personnelles, y compris les paramètres de sécurité.

\textbf{Éléments:}
\begin{itemize}
    \item \textbf{Affichage des informations personnelles:} Nom,prénom,tel, adresse ,e-mail .
    \item \textbf{Options pour modifier les informations personnelles:} Permet de mettre à jour les informations et de changer le mot de passe.
\end{itemize}


\section{Interface Accueil Police}

\begin{center}
    \includegraphics[width=0.8\textwidth]{3-partie/policeaccueil.png} 
     \includegraphics[width=0.8\textwidth]{3-partie/connexion2.png} 
\end{center}

\textbf{Description :}  
Page d’accueil destinée aux agents de police, offrant une interface centralisée pour :  
\begin{itemize}
    \item le \textbf{suivi en temps réel} des déclarations de vol,  
    \item la réception d’\textbf{alertes instantanées} lors de nouveaux cambriolages signalés,  
    \item l’accès rapide aux \textbf{détails des véhicules déclarés} (statut : volé, retrouvé),  
    % \item la \textbf{cartographie géolocalisée} des zones de forte criminalité,  
    % \item la génération de \textbf{rapports synthétiques} pour faciliter la prise de décision et l’organisation des patrouilles.  
\end{itemize}

% \textbf{Éléments:}
% \begin{itemize}
%     \item \textbf{Affichage des informations personnelles:} Nom,prénom,tel, adresse ,e-mail .
%     \item \textbf{Options pour modifier les informations personnelles:} Permet de mettre à jour les informations et de changer le mot de passe.
% \end{itemize}






\section{Interface Administrateur}

\begin{center}
    \includegraphics[width=0.8\textwidth]{3-partie/admin.png} 
\end{center}

\textbf{Description:} Outil réservé aux administrateurs pour gérer les données et rapports relatifs aux cambriolages.

\textbf{Éléments:}
\begin{itemize}
    \item \textbf{Acueil:} Affiche le suivi de ces declarations sur les cambriolages puid des alertes au nouveaux cambriolages.
    \item \textbf{Gestion des rapports:} Permet d'ajouter puis de consulter la liste des rapports.
    \item \textbf{Gestion des patrouilles:} Permet d'ajouter puis de consulter la liste des patrouilles.
    \item \textbf{Gestion des utilisateurs:} Ajouter , modifie  ou supprimer des comptes utilisateurs .
    \item \textbf{Gestion des polices:} Ajouter , modifie  ou supprimer des comptes polices .
\end{itemize}

\section{Interface Gestion des  Patrouilles}

\begin{center}
    \includegraphics[width=0.8\textwidth]{3-partie/gestionpatrouille.png} 
\end{center}


\textbf{Description:} interface  dédié aux patrouilles policières pour suivre leurs missions et interventions.

\textbf{Éléments:}
\begin{itemize}
    \item \textbf{Liste des missions assignées:} Détail des missions que les patrouilles doivent effectuer.
    \item \textbf{Enregistrement des interventions:} Permet aux patrouilles de saisir des rapports en temps réel sur leurs interventions.
\end{itemize}


\section{Interface Gestion des  Rapports}

\begin{center}
    \includegraphics[width=0.8\textwidth]{3-partie/gestionrapport.png} 
\end{center}

\textbf{Description :}  
Cette interface est dédiée à la gestion des rapports générés suite aux déclarations de vols.  
Elle permet aux agents de police et aux administrateurs de :  
\begin{itemize}
    \item consulter l’ensemble des rapports enregistrés,  
    \item rechercher un rapport spécifique par numéro, date, ou plaque d’immatriculation,  
    \item filtrer les rapports par statut (\textit{volé, retrouvé}),  
    % \item ajouter de nouvelles informations ou preuves liées à une déclaration,  
    \item exporter les rapports sous forme de documents PDF pour archivage ou transmission.  
\end{itemize}




\section{Interface Gestion des utilisateurs}
% \begin{itemize}
    % \item \textbf{Gestion des utilisateurs} : Interface permettant  l'ajout et la suppression de comptes, ainsi que l'édition des permissions.
    % \item \textbf{Rapports et statistiques} : Outils pour générer des rapports détaillés sur les tendances des cambriolages (par zone géographique, période, type de véhicule volé, etc.).
    % \item \textbf{Gestion des paramètres du site} : Interface de configuration des alertes, notifications, et paramètres de sécurité du site.
    % \item \textbf{Analyse des données} : Outils de visualisation des données pour suivre les performances du site, la résolution des cas et d'autres métriques importantes.
% \end{itemize}

\textbf{Description :}  
Cette interface est dédiée à l’administration des utilisateurs de la plateforme.  
Elle permet à l’administrateur ou aux agents autorisés de gérer les comptes et les accès.  

\begin{itemize}
    \item afficher la liste complète des utilisateurs inscrits,  
    \item ajouter de nouveaux utilisateurs avec leurs informations personnelles ,  
    % \item modifier les informations d’un utilisateur existant (nom, email, rôle, statut),  
    \item supprimer ou désactiver un compte en cas d’abus ou d’inactivité,  
    % \item attribuer des rôles spécifiques (\textit{citoyen, police, administrateur}),  
    \item réinitialiser le mot de passe d’un utilisateur.  
\end{itemize}




\section{Interface Gestion des Polices}

\begin{center}
    \includegraphics[width=0.8\textwidth]{3-partie/gestionpolice.png} 
\end{center}


\textbf{Description :}  
Cette interface est conçue pour gérer les comptes et activités des agents de police au sein du système.  
Elle permet un meilleur suivi, une répartition des tâches efficace et un contrôle des accès.  

\begin{itemize}
    \item afficher la liste complète des agents de police enregistrés,  
    \item ajouter un nouvel agent avec ses informations personnelles et son matricule,  
    \item modifier ou mettre à jour les informations d’un agent (nom, email, affectation),  
    \item activer ou désactiver le compte d’un agent en fonction de son statut,  
    \item attribuer des rôles ou responsabilités spécifiques (\textit{patrouille, gestion des rapports, supervision}),  
    \item suivre l’historique des actions et interventions de chaque agent.  
\end{itemize}



\section{Interface pour SuperAdmin}

\begin{center}
    \includegraphics[width=0.8\textwidth]{3-partie/admin.png} 
\end{center}



\textbf{Description:} Interface réservée aux superadministrateurs qui ont tous les droits d’accès pour gérer le système global de l'application.

\textbf{Éléments:}
\begin{itemize}
    \item \textbf{Gestion complète du système:} Gérer les utilisateurs, les permissions et les configurations globales de l'application.
    \item \textbf{Historique des actions:} Suivi complet des actions administratives effectuées dans le système.
\end{itemize}


% \textbf{Description:} Interface spécifique pour les administrateurs afin de gérer les utilisateurs, rapports  de l'application.

% \textbf{Éléments:}
% \begin{itemize}
%     \item \textbf{Gestion des utilisateurs:} Ajouter, modifier, ou supprimer des utilisateurs ou policiers.
%     \item \textbf{Accès aux rapports:} Gestion complète des rapports sur les cambriolages.
%     \item \textbf{Statistiques:} Accès aux données globales de l'application.
% \end{itemize}

\section{Interface Gestion des Administrateurs}


\begin{center}
    \includegraphics[width=0.8\textwidth]{3-partie/gestionadmin.png} 
\end{center}


\textbf{Description:} Interface réservée aux superadministrateurs qui ont tous les droits d’accès pour gérer le système global de l'application.
elle permet de gérer les comptes des administrateurs du système à la gestion des rôles .  

\begin{itemize}
    \item afficher la liste de tous les administrateurs enregistrés,  
    \item ajouter un nouvel administrateur avec ses informations (npi,nom,prenom,adresse, email),  
    \item modifier ou mettre à jour les informations d’un administrateur,  
    \item activer ou désactiver le compte d’un administrateur,  
    \item suivre l’historique des actions administratives (création, suppression, mises à jour).  
\end{itemize}


% \textbf{Éléments:}
% \begin{itemize}
%     \item \textbf{Gestion complète du système:} Gérer les utilisateurs, les permissions et les configurations globales de l'application.
%     \item \textbf{Historique des actions:} Suivi complet des actions administratives effectuées dans le système.
% \end{itemize}



% \section{Interface Mot de Passe Oublié}

% \begin{center}
%     \includegraphics[width=0.8\textwidth]{forgotpassword.jpg} 
% \end{center}


% \textbf{Description:} Permet aux utilisateurs de récupérer leur mot de passe en cas d'oubli.

% \textbf{Éléments:}
% \begin{itemize}
%     \item \textbf{Champ pour l'adresse e-mail:} Permet de récupérer l'adresse associée au compte.
%     \item \textbf{Instructions pour la réinitialisation:} Guide pour aider l’utilisateur à réinitialiser son mot de passe.
% \end{itemize}

% \section{Changer de Mot de Passe}

% \begin{center}
%     \includegraphics[width=0.8\textwidth]{changepassword.jpg} 
% \end{center}

% \textbf{Description:} Permet aux utilisateurs et policiers de modifier leur mot de passe pour des raisons de sécurité.

% \textbf{Éléments:}
% \begin{itemize}
%     \item \textbf{Champs pour l'ancien et le nouveau mot de passe:} Demande l'ancien mot de passe ainsi que le nouveau.
%     \item \textbf{Bouton "Changer":} Soumet la demande de changement de mot de passe.
% \end{itemize}



\subsection{Conclusion}
La gestion des cambriolages de véhicules bénéficie grandement d'une approche collaborative et d'une gestion efficace des rôles des utilisateurs. La technologie, en particulier les systèmes de surveillance et de géolocalisation, associée à des stratégies de prévention et à une coopération étroite entre les forces de l'ordre, les citoyens et les administrateurs, constitue un moyen puissant de réduire ces infractions. L’implication des superadmins et des administrateurs garantit une supervision technique optimale, permettant ainsi une réponse rapide et ciblée aux cambriolages de véhicules. Grâce à cette coordination, il devient possible de minimiser les risques, de protéger les biens des citoyens et d’améliorer la sécurité dans les zones à haut risque.
















% \begin{abstract}
% Ce rapport présente une analyse complète des cambriolages de véhicules, incluant des statistiques globales, une répartition géographique, des analyses temporelles, et un suivi des enquêtes. L'objectif est de fournir une vue d'ensemble des incidents, d'identifier les tendances et d'aider à la prise de décision en matière de sécurité publique.
% \end{abstract}

% \newpage

% \section{Introduction}
% Les cambriolages de véhicules représentent une menace importante pour la sécurité des biens. Ce rapport présente une analyse détaillée de la situation actuelle des cambriolages de véhicules, avec des informations sur la répartition géographique, les tendances temporelles et l'efficacité des mesures de prévention.

% \section{Statistiques Globales}
% \subsection{Nombre Total de Cambriolages de Véhicules}
% \begin{itemize}
%     \item Nombre total de cambriolages : 3500 incidents.
%     \item Nombre de cambriolages résolus : 1200.
%     \item Nombre de cambriolages en attente : 2300.
%     \item Taux de résolution : 34.29\%.
% \end{itemize}

% \subsection{Valeur des Biens Volés}
% La valeur estimée des biens volés dans les véhicules cambriolés est de 5,000,000 €.

% \section{Répartition par Localisation}
% La carte suivante montre les zones géographiques les plus affectées par les cambriolages de véhicules.

% \begin{figure}[h!]
%     \centering
%     \includegraphics[width=\textwidth]{carte_cambriolages.png}
%     \caption{Répartition des cambriolages de véhicules par zone géographique}
% \end{figure}

% \subsection{Carte de Chaleur des Cambriolages}
% La carte de chaleur ci-dessous montre les zones où les cambriolages de véhicules sont les plus fréquents.

% % Insérer un graphique ici avec pgfplots ou inclure une carte d'une image
% \begin{figure}[h!]
%     \centering
%     \begin{tikzpicture}
%         \begin{axis}[
%             width=0.8\textwidth,
%             height=0.5\textwidth,
%             title={Carte de Chaleur des Cambriolages},
%             xlabel={Longitude},
%             ylabel={Latitude},
%             colorbar
%         ]
%         % Exemple de données fictives
%         \addplot[scatter,only marks,point meta=explicit] coordinates {
%             (48.8566, 2.3522) [0.5]
%             (48.8706, 2.3858) [1.2]
%             (48.8530, 2.3694) [2.4]
%             (48.8767, 2.3327) [1.0]
%         };
%         \end{axis}
%     \end{tikzpicture}
%     \caption{Carte de chaleur des cambriolages de véhicules}
% \end{figure}

% \section{Analyse Temporelle des Cambriolages}
% \subsection{Cambriolages par Mois}
% Le graphique suivant présente le nombre de cambriolages de véhicules par mois au cours de l'année écoulée.

% \begin{figure}[h!]
%     \centering
%     \begin{tikzpicture}
%         \begin{axis}[
%             width=\textwidth,
%             title={Nombre de Cambriolages par Mois},
%             xlabel={Mois},
%             ylabel={Nombre de Cambriolages},
%             ybar
%         ]
%         \addplot coordinates {
%             (1, 150) (2, 120) (3, 180) (4, 140) (5, 100) (6, 130) (7, 160) (8, 170) (9, 190) (10, 200) (11, 180) (12, 210)
%         };
%         \end{axis}
%     \end{tikzpicture}
%     \caption{Nombre de cambriolages de véhicules par mois}
% \end{figure}

% \subsection{Cambriolages par Heure de la Journée}
% Le graphique ci-dessous montre la répartition des cambriolages de véhicules selon l'heure de la journée.

% \begin{figure}[h!]
%     \centering
%     \includegraphics[width=\textwidth]{graph_cambriolages_heure.png}
%     \caption{Répartition des cambriolages par heure de la journée}
% \end{figure}

% \section{Suivi des Enquêtes}
% \subsection{Statut des Enquêtes}
% \begin{itemize}
%     \item Enquêtes ouvertes : 2300
%     \item Enquêtes fermées : 1200
%     \item Arrestations effectuées : 500
%     \item Nombre d'enquêtes résolues avec arrestation : 150
% \end{itemize}

% \subsection{Temps Moyen de Résolution}
% Le temps moyen pour résoudre une enquête sur un cambriolage de véhicule est de 45 jours.

% \section{Conclusion}
% Ce rapport fournit un aperçu détaillé des cambriolages de véhicules. Les informations recueillies aideront à la prise de décision concernant la mise en place de nouvelles stratégies de prévention, ainsi qu'à la gestion efficace des enquêtes en cours.















% \section*{Récapitulatif des Fonctionnalités par Rôle}
% Le tableau ci-dessous présente les fonctionnalités principales disponibles pour chaque type d'utilisateur dans un site de gestion des cambriolages de véhicules.

% \begin{table}[ht!]
% \centering
% \begin{tabular}{|>{\raggedright}p{4cm}|c|c|c|}
% \hline
% \textbf{Fonctionnalité} & \textbf{Utilisateur} & \textbf{Police} & \textbf{Administrateur} \\
% \hline
% \textbf{Déclaration de cambriolage} & \checkmark & & \\
% \hline
% \textbf{Suivi de l'état des déclarations} & \checkmark & \checkmark & \\
% \hline
% \textbf{Consultation des statistiques} & \checkmark & \checkmark & \checkmark \\
% \hline
% \textbf{Gestion des enquêtes} & & \checkmark & \checkmark \\
% \hline
% \textbf{Gestion des utilisateurs} & & & \checkmark \\
% \hline
% \textbf{Suivi des statistiques de crime} & & \checkmark & \checkmark \\
% \hline
% \textbf{Gestion des alertes} & & \checkmark & \checkmark \\
% \hline
% \textbf{Accès à la carte géographique} & \checkmark & \checkmark & \checkmark \\
% \hline
% \textbf{Gestion des paramètres du site} & & & \checkmark \\
% \hline
% \textbf{Analyse de données} & & \checkmark & \checkmark \\
% \hline
% \end{tabular}
% \caption{Fonctionnalités par Rôle}
% \end{table}








% \section*{Introduction}
% Ce document présente les maquettes pour le tableau de bord d'un site de gestion des cambriolages de véhicules, avec des descriptions pour chaque rôle d'utilisateur : Citoyen, Police et Administrateur.

% \section*{Maquette pour l'Utilisateur (Citoyen)}



% \textbf{Exemple de maquette pour l'utilisateur :}

% \begin{verbatim}
%  --------------------------------------------------------
% |                    Mon Tableau de Bord                |
% | ------------------------------------------------------|
% | 1. [Déclarer un Cambriolage] [Voir l'historique]      |
% | 2. Statut des Déclarations (En Cours / Résolu)        |
% | 3. [Alertes de Sécurité]                              |
% |    - Cambriolage dans ma zone                         |
% | 4. Carte Interactive                                  |
% |    - [Vue de la carte avec les points des cambriolages]|
%  --------------------------------------------------------
% \end{verbatim}

% \section*{Maquette pour la Police (Forces de l'ordre)}

% \begin{itemize}
%     \item \textbf{Gestion des déclarations} : Un tableau de bord avec une liste des déclarations reçues, leur statut, et des options pour marquer les déclarations comme "enquête ouverte", "clôturée", etc.
%     \item \textbf{Suivi des enquêtes} : Accès aux détails des enquêtes ouvertes, avec des outils pour ajouter des commentaires, joindre des rapports et des pièces justificatives (preuves).
%     \item \textbf{Carte des cambriolages} : Une carte dynamique permettant à la police de visualiser la concentration des cambriolages dans différentes zones géographiques.
%     \item \textbf{Gestion des suspects et des arrestations} : Liste des suspects potentiels avec des informations détaillées sur les profils, les connexions avec d'autres enquêtes et un historique des actions entreprises.
% \end{itemize}

% \textbf{Exemple de maquette pour la police :}

% \begin{verbatim}
%  --------------------------------------------------------
% |                 Tableau de Bord de la Police           |
% | ------------------------------------------------------|
% | 1. [Gestion des Déclarations] [Suivi des Enquêtes]     |
% | 2. [Carte des Cambriolages]                           |
% |    - Carte avec zones de cambriolages fréquents        |
% | 3. [Liste des Suspects]                               |
% |    - Suspect 1: Détails & Actions                     |
% |    - Suspect 2: Détails & Actions                     |
%  --------------------------------------------------------
% \end{verbatim}

% \section*{Maquette pour l'Administrateur}


% \textbf{Exemple de maquette pour l'administrateur :}

% \begin{verbatim}
%  --------------------------------------------------------
% |                    Tableau de Bord Admin              |
% | ------------------------------------------------------|
% | 1. [Gestion des Utilisateurs] [Voir les Statistiques]  |
% | 2. [Gestion des Paramètres]                           |
% |    - Modifier les alertes et notifications            |
% | 3. [Analyse des Données]                              |
% |    - Graphiques sur les tendances des cambriolages    |
% | 4. Rapports de Sécurité                               |
% |    - Voir les rapports sur les zones à risque         |
%  --------------------------------------------------------
% \end{verbatim}

% \section*{Diagramme Global du Tableau de Bord}

% Voici une représentation schématique de la structure globale du tableau de bord pour le site de gestion des cambriolages de véhicules, avec les pages principales pour chaque rôle :

% \begin{verbatim}
%                            +--------------------------+
%                            |   Tableau de Bord Admin   |
%                            +--------------------------+
%                             /              |            \
%            +-------------------------+     |     +-------------------------+
%            |  Gestion des Utilisateurs |     |     |    Statistiques et Rapports |
%            +-------------------------+     |     +-------------------------+
%                     |                    |                   |
%       +-------------------------+    +-------------------------+   +--------------------------+
%       | Tableau de Bord Police   |    |  Tableau de Bord User    |   |  Paramètres du Site      |
%       +-------------------------+    +-------------------------+   +--------------------------+
%          /             |                |                     |              \
% +-----------------+  +---------------+  +---------------+  +-------------------------+
% | Gestion des     |  | Suivi des     |  | Déclaration   |  | Alertes de Sécurité      |
% | Déclarations    |  | Enquêtes      |  | Cambriolage   |  | et Notifications         |
% +-----------------+  +---------------+  +---------------+  +-------------------------+
% \end{verbatim}

% \section*{Conclusion}
% Les maquettes ci-dessus montrent les différentes pages et fonctionnalités du site de gestion des cambriolages de véhicules, réparties par rôle. Chaque page a été conçue pour répondre aux besoins spécifiques de chaque type d'utilisateur tout en offrant une interface claire et intuitive pour une gestion efficace des cambriolages.











% \section*{1. Header (En-tête)}

% \begin{itemize}
%     \item \textbf{Logo du site} : Sur la gauche, pour une identification facile.
%     \item \textbf{Barre de navigation} :
%     \begin{itemize}
%         \item Menu principal avec des liens vers :
%         \begin{itemize}
%             \item Page d'accueil
%             \item Signaler un cambriolage
%             \item Suivi des incidents
%             \item Statistiques et tendances
%             \item Conseils de prévention
%         \end{itemize}
%         \item \textbf{Connexion} : Accès pour les utilisateurs et administrateurs.
%         \item \textbf{Inscription} : Option pour les nouveaux utilisateurs.
%     \end{itemize}
% \end{itemize}

% \section*{2. Section principale : Introduction et appels à l'action}

% \begin{itemize}
%     \item \textbf{Slogan accrocheur} : Un message court qui explique l’objectif du site. Exemple : ``Protégez votre véhicule, signalez les cambriolages''.
%     \item \textbf{Options de signalement rapide} :
%     \begin{itemize}
%         \item Signaler un cambriolage : Un bouton large et bien visible permettant aux citoyens de signaler un vol de véhicule.
%         \item Voir les incidents récents : Un lien vers une carte interactive ou une liste des derniers cambriolages signalés.
%     \end{itemize}
% \end{itemize}

% \section*{3. Carte des incidents récents}

% \begin{itemize}
%     \item Affichage dynamique des incidents récents avec des épingles de localisation sur une carte.
%     \item \textbf{Filtres} : Permettre de filtrer par date, type de véhicule, ou zone géographique.
%     \item \textbf{Popup sur les incidents} : Cliquer sur une épingle affiche des détails sur le cambriolage (lieu, date, description).
% \end{itemize}

% \section*{4. Statistiques et tendances}

% \begin{itemize}
%     \item \textbf{Graphiques interactifs} : Diagrammes à barres ou courbes montrant les tendances des cambriolages (par mois, par type de véhicule, etc.).
%     \item \textbf{Carte des zones sensibles} : Affichage des zones où les cambriolages sont les plus fréquents.
% \end{itemize}

% \section*{5. Section des conseils de prévention}

% \begin{itemize}
%     \item \textbf{Prévention de cambriolage} : Des conseils pratiques pour éviter les vols (verrouillage des véhicules, installation de systèmes d'alarme, etc.).
%     \item \textbf{Vidéos ou guides} : Des vidéos explicatives sur la sécurisation des véhicules.
% \end{itemize}

% \section*{6. Footer (Pied de page)}

% \begin{itemize}
%     \item \textbf{Liens utiles} : Mentions légales, politique de confidentialité, et coordonnées.
%     \item \textbf{Réseaux sociaux} : Liens vers les comptes du site sur les réseaux sociaux.
% \end{itemize}

% \newpage

% \section*{Découpage détaillé des informations et fonctionnalités spécifiques pour chaque type d'utilisateur}

% \subsection*{1. Utilisateur citoyen / Victime}

% \begin{itemize}
%     \item \textbf{Accès} : Connexion via un formulaire de login.
%     \item \textbf{Dashboard utilisateur} :
%     \begin{itemize}
%         \item Mes incidents : Liste des cambriolages signalés par l'utilisateur, avec possibilité de suivre l'état de chaque incident (en cours, résolu, fermé).
%         \item Créer un signalement : Formulaire de signalement de cambriolage avec les informations suivantes :
%         \begin{itemize}
%             \item Date et heure du vol
%             \item Type et modèle de véhicule
%             \item Description du vol et des objets volés
%             \item Lieu du vol (géolocalisation ou adresse manuelle)
%             \item Photos des dégâts (facultatif mais recommandé)
%         \end{itemize}
%         \item Suivi de l’enquête : Un accès aux mises à jour données par la police concernant l'enquête.
%         \item Recevoir des alertes : Notifications en temps réel concernant les nouveaux cambriolages dans la zone géographique de l’utilisateur.
%         \item Conseils de sécurité personnalisés : Suggestions sur la manière d'éviter les vols.
%     \end{itemize}
    % \item \textbf{Interaction communautaire} :
    % \begin{itemize}
    %     \item Forum de discussion : Un espace pour discuter avec d'autres citoyens sur la sécurité des véhicules, partager des astuces, ou signaler des comportements suspects.
    %     \item Alertes communautaires : Option pour signaler un cambriolage ou une tentative suspecte dans leur quartier.
    % \end{itemize}
% \end{itemize}

% \subsection*{2. Forces de l'ordre (Police)}

% \begin{itemize}
%     \item \textbf{Accès} : Authentification via un login sécurisé avec un rôle spécifique.
%     \item \textbf{Dashboard policier} :
%     \begin{itemize}
%         \item Incidents en cours : Liste des cambriolages signalés avec leur statut d’enquête (en cours, résolu).
%         \item Investigation : Accès pour ajouter des notes, télécharger des preuves (photos, vidéos), et mettre à jour le statut de chaque enquête.
%         \item Suspects : Ajouter ou consulter des informations sur les suspects potentiels liés à des incidents.
%         \item Historique des enquêtes : Suivi des anciens incidents, en ajoutant des détails relatifs aux résultats de l'enquête.
%         \item Statistiques des cambriolages : Accès aux tendances des cambriolages dans leur secteur (nombre de cambriolages par mois, types de véhicules les plus volés, etc.).
%         \item Coordination avec les administrateurs : Partage d'informations avec les administrateurs ou autres forces de l'ordre.
%         \item Alertes en temps réel : Recevoir des notifications pour chaque nouveau cambriolage signalé, en particulier ceux dans leur zone d'action.
%     \end{itemize}
% \end{itemize}

% \subsection*{3. Administrateur du système}

% \begin{itemize}
%     \item \textbf{Accès} : Connexion avec des droits d'accès élevés pour gérer l'ensemble du système.
%     \item \textbf{Dashboard administrateur} :
%     \begin{itemize}
%         \item Gestion des utilisateurs : Créer, modifier ou supprimer des comptes utilisateurs (citoyens, policiers).
%         \item Modération des signalements : Examiner les signalements de cambriolages soumis par les citoyens pour s'assurer qu'ils sont appropriés.
%         \item Surveillance des enquêtes : Vérifier l'état des enquêtes et assurer leur progression en cas de retard ou de besoin d’assistance.
%         \item Statistiques avancées : Accéder à des rapports détaillés sur les cambriolages (par région, par période, par type de véhicule).
%         \item Gestion des rôles : Affecter des rôles et permissions aux utilisateurs (citoyen, policier, administrateur).
%         \item Gestion des alertes : Paramétrer les alertes pour les utilisateurs, définir les zones sensibles et autres notifications.
%     \end{itemize}
%     \item \textbf{Sécurité et gestion des données} :
%     \begin{itemize}
%         \item Protection des données utilisateurs : Assurer la conformité aux réglementations comme le RGPD (pour les utilisateurs européens).
%         \item Gestion des logs et des journaux d'activité : Suivre les actions effectuées par les utilisateurs pour garantir la sécurité du système.
%     \end{itemize}
% \end{itemize}












% \section*{1. Base de données des incidents de cambriolage}

% \begin{itemize}
%     \item \textbf{Date et heure de l'incident} : Enregistrement précis de la date et de l'heure.
%     \item \textbf{Lieu de l'incident} : Adresse ou géolocalisation de l'incident, incluant les coordonnées GPS.
%     \item \textbf{Type de véhicule} : Informations sur le type de véhicule (marque, modèle, couleur, immatriculation).
%     \item \textbf{Description du vol} : Détails sur les objets volés, les méthodes utilisées par les cambrioleurs, etc.
%     \item \textbf{Dégâts causés} : Éventuels dommages au véhicule (fenêtres cassées, serrure forcée, etc.).
%     \item \textbf{Victime/Propriétaire} : Informations sur le propriétaire du véhicule (nom, contact, etc.).
% \end{itemize}

% \section*{2. Signalement }

% \begin{itemize}
%     \item \textbf{Formulaire de signalement} : Permettre aux victimes de signaler le vol de manière simple et détaillée (avec téléchargement de photos des dégâts).
%     \item \textbf{Suivi des plaintes} : Un système pour suivre l’état de la plainte (en cours, résolu, etc.).
%     \item \textbf{Numéro de dossier} : Un identifiant unique pour chaque incident afin de faciliter le suivi.
% \end{itemize}

% \section*{3. Gestion des enquêtes}

% \begin{itemize}
%     \item \textbf{Historique des enquêtes} : Enregistrement des enquêtes menées par les forces de l’ordre ou d’autres organismes.
%     \item \textbf{Suivi des suspects} : Stockage des informations sur les suspects potentiels, leur description physique, leur mode opératoire, etc.
%     \item \textbf{Résultats des enquêtes} : Informations sur l’avancée des enquêtes et la résolution des cas.
% \end{itemize}

% \section*{4. Statistiques et rapports}

% \begin{itemize}
%     \item \textbf{Carte des incidents} : Afficher les cambriolages sur une carte interactive, permettant d’identifier les zones les plus touchées.
%     \item \textbf{Rapports de tendances} : Fournir des statistiques détaillées (nombre de cambriolages par mois, types de véhicules les plus ciblés, etc.).
%     \item \textbf{Alertes et notifications} : Un système d’alertes pour informer les propriétaires de véhicules des incidents dans leurs zones géographiques.
% \end{itemize}

% \section*{5. Gestion des utilisateurs}

% \begin{itemize}
%     \item \textbf{Comptes utilisateurs} : Création d’un espace pour les citoyens, les autorités, et les responsables de la gestion des cambriolages.
%     \item \textbf{Permissions d’accès} : Définir les niveaux d'accès (utilisateur, administrateur, force de l’ordre, etc.).
%     \item \textbf{Notifications par email/SMS} : Alerter les utilisateurs des nouvelles informations sur l'enquête ou les alertes de cambriolages.
% \end{itemize}

% \section*{6. Interaction avec les forces de l’ordre}

% \begin{itemize}
%     \item \textbf{Partage d’informations} : Un système sécurisé pour échanger des données avec les forces de l’ordre ou d'autres services concernés.
%     \item \textbf{État des enquêtes en temps réel} : Permettre aux autorités de mettre à jour l’état de chaque incident.
% \end{itemize}

% \section*{7. Sécurité et confidentialité des données}

% \begin{itemize}
%     \item \textbf{Protection des données} : Assurer la confidentialité des informations personnelles (en conformité avec la législation, comme le RGPD en Europe).
%     \item \textbf{Authentification sécurisée} : Mettre en place un système d’authentification fort (connexion avec un mot de passe, double authentification, etc.).
% \end{itemize}

% \section*{8. Sensibilisation et prévention}

% \begin{itemize}
%     \item \textbf{Conseils de prévention} : Offrir des conseils aux utilisateurs pour éviter les cambriolages de véhicules (par exemple, ne pas laisser d'objets de valeur à la vue).
%     \item \textbf{Alertes communautaires} : Créer une fonctionnalité permettant aux utilisateurs d'alerter leur communauté si un cambriolage est suspecté dans leur secteur.
% \end{itemize}

% \section*{9. Accessibilité mobile}

% \begin{itemize}
%     \item \textbf{Application mobile} : Fournir une version mobile pour permettre aux utilisateurs de signaler un vol et suivre l’avancement des enquêtes en temps réel.
% \end{itemize}

% \section*{10. Intégration avec les systèmes externes}

% \begin{itemize}
%     \item \textbf{Partenariat avec des assurances} : Intégrer des fonctionnalités pour permettre aux victimes de faire une déclaration à leur compagnie d’assurance directement via le site.
%     \item \textbf{Collaboration avec des plateformes de surveillance} : Si possible, lier la plateforme à des caméras de sécurité ou des systèmes de surveillance pour augmenter les chances de résoudre les enquêtes.
% \end{itemize}

% \section*{11. Interactivité avec la communauté}

% \begin{itemize}
%     \item \textbf{Forum ou discussions en ligne} : Permettre aux utilisateurs de discuter de leurs expériences et de partager des conseils de sécurité.
%     \item \textbf{Programmes de prévention communautaire} : Mettre en place des initiatives locales pour sensibiliser la communauté à la sécurité des véhicules.
% \end{itemize}

