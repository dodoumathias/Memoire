%--------------A VOTRE ATTENTION-------------%
% Les étudiants en master qui disposent de plus de 3 chapitres dans leurs travaux peuvent en complèter
% Les Membres doivent figurer dans la dernière version finale du mémoire après soutenance pour dépôt de mémoire

\documentclass{ifri}
\usepackage{titletoc}
\setlength{\glsdescwidth}{0.65\textwidth}
% \usepackage{lscape}

\usepackage{newunicodechar}
\newunicodechar{’}{'}  % remplace ’ par '

\usepackage{newunicodechar}
\newunicodechar{–}{--}  % remplace le tiret demi-cadratin



\typeMemoire{Diplôme de Licence en Informatique}
\optionFormation{Système d'Information et Réseau Informatique}
\etudiant{Mathias  Mahudo \textbf{DODOU}}
\titreDuMemoire{Gestion du Cambriolage de Véhicule au Bénin: Création d'un Système d'Alerte} %Implémention pour une meilleure sécurité dans les réseau LAN sous IPv6 // Proposition: Identification des vulnérabilités dans un reseau LAN IPv6 et mesures pour une meilleure sécurité.

\dateSoutenance{-}
%\promo{2\up{ème}}
\anneeScolaire{2024-2025}


%%maitre de mémoire
\encadrants{Ing Ange \textbf{ALAKONON}}

%% Membres du Jury
\jurys{%
\begin{tabular}{llll}
	Nom et prénoms du président & Grade & Entité & Président \\
	Nom et prénoms de l'examinateur & Grade & Entité & Examinateur \\
	Nom et prénoms du rapporteur & Grade & Entité & Rapporteur \\
\end{tabular}	
}


\hypersetup{
 pdftitle={Gestion des Cambriolages de Véhicules},
 pdfauthor={Mathias Mahudo DODOU},
 pdfsubject={Sécurité des véhicules},
 pdfkeywords={cambriolage, sécurité, technologie}  
 }

\color{bookColor}

%importation du glossaire
\loadglsentries{glossaire_reduit}

\begin{document}

\pageDeGarde
%\pageTitre


\pagecolor{white}

%% page vide
%\thispagestyle{empty}\ \clearpage


\selectlanguage{french}

% sommaire
\pagenumbering{roman}

\setcounter{tocdepth}{0}
\startlist{toc}
\printlist{toc}{}{\chapter*{Sommaire}}
\setcounter{tocdepth}{5}



%% rdedicaces
\dedicace

Je tiens à dédier ce travail à ma famille, pour leur soutien indéfectible, leur patience et leur encouragement tout au long de mes études.

À mes amis proches et collègues pour leur assistance et leur compréhension pendant la réalisation de ce mémoire.

Enfin, à tous ceux qui ont contribué, de près ou de loin, à l'accomplissement de ce projet. Merci !




\newpage

%% remerciements
\remerciements


Je tiens à exprimer ma gratitude la plus sincère à mon directeur de mémoire, Monsieur \textbf{AHOTIN Carmel}, pour sa direction précieuse, ses conseils avisés et sa disponibilité tout au long de la préparation de ce mémoire.

Je remercie également mes collègues et amis pour leur soutien moral, ainsi que pour les discussions constructives qui ont enrichi ma réflexion sur ce sujet.

Enfin, je voudrais remercier ma famille pour leur patience, leur amour et leur encouragement sans faille.

\newpage

% Résume
\resume
\selectlanguage{french}
\vspace*{-6cm}
\begin{abstract}

Ce mémoire traite de la gestion et du suivi  des cambriolages de véhicules au Bénin. Face à l’augmentation de ces actes, la protection des véhicules et des biens des citoyens nécessite la mise en place de solutions efficaces, rapides et accessibles. L’objectif principal de ce travail est de concevoir une plateforme numérique collaborative permettant d’améliorer la prévention, la détection et la prise en charge des vols de véhicules.

La plateforme proposée permet aux citoyens de déclarer un vol, de signaler des comportements suspects et de suivre l’évolution des incidents, tandis que les forces de l’ordre disposent d’outils centralisés pour analyser les informations et intervenir plus rapidement.

Pour atteindre cet objectif, une analyse des cas de cambriolages de véhicules et des solutions existantes a été réalisée. Les résultats obtenus montrent que la collaboration entre les citoyens et les forces de l’ordre, appuyée par un outil numérique adapté, permet d’améliorer la réactivité des interventions et d’augmenter les chances de récupération des véhicules.

Ce mémoire met ainsi en évidence l’importance d’une approche participative et technologique dans la lutte contre le cambriolage des véhicules et propose une solution simple et opérationnelle adaptée au contexte béninois.

\paragraph{}
\textbf{Mots clés}: Vol de véhicules, cambriolage, plateforme collaborative, forces de l’ordre.

\end{abstract}


\newpage
\thispagestyle{empty}
\selectlanguage{english}
\addcontentsline{toc}{chapter}{Abstract}
\begin{abstract}

This thesis addresses the management and monitoring of vehicle thefts and break-ins in Benin. With the increase in such incidents, protecting vehicles and citizens' property requires effective, rapid, and accessible solutions. The main objective of this work is to design a collaborative digital platform aimed at improving the prevention, detection, and handling of vehicle thefts.

The proposed platform allows citizens to report a theft, alert authorities to suspicious behavior, and track incidents in real time, while law enforcement agencies have centralized tools to analyze information and respond more efficiently.

To achieve this goal, an analysis of vehicle break-in cases and existing solutions was conducted. The results show that collaboration between citizens and law enforcement, supported by an appropriate digital tool, improves response times and increases the chances of recovering stolen vehicles.

This thesis highlights the importance of a participatory and technological approach in combating vehicle theft and proposes a simple and operational solution adapted to the Beninese context.

\paragraph{}
\textbf{Keywords}: Vehicle theft, burglary, collaborative platform, law enforcement.

\end{abstract}

\newpage

%liste des figures
\listoffigures
\newpage

%liste des tableaux
\listoftables
\newpage

%liste des algo
\selectlanguage{french}
\listofalgorithmes
\newpage

% Les sigles et acronymes
\setglossarystyle{altlist}
\printglossary[title=Liste des acronymes, toctitle=Liste des acronymes, type=\acronymtype]
\newpage

% Le glossaire proprement dit
%\setglossarystyle{super}
%\printglossary[type=main]


\pagenumbering{arabic}
\setcounter{page}{1}
%%introduction
% \introduction
% % bla bla bla \gls{acro} puis \Gls{acroglo} et enfin \gls{glossaire}

% % \chapter{Introduction}

% \section*{Contexte}

% Le cambriolage de véhicule est un phénomène de plus en plus préoccupant à travers le monde. Chaque année, des milliers de véhicules sont volés, représentant non seulement une perte financière significative pour les propriétaires, mais aussi un facteur de détérioration de la confiance et de sécurité au sein des communautés. Les cambrioleurs ont développé des méthodes de plus en plus sophistiquées, rendant difficile la détection et la prévention des vols. En conséquence, ce type de crime constitue un enjeu majeur pour la sécurité publique, incitant à l'exploration de nouvelles solutions pour y faire face.

% \section*{Problématique}

% Face à la sophistication croissante des méthodes de vol et à l’augmentation des cambriolages de véhicules, la prévention et la détection des véhicules volés deviennent des tâches de plus en plus complexes. Le manque de systèmes d'alerte rapide, de coordination entre les citoyens et les autorités locales, ainsi que l'inefficacité de certaines technologies existantes, accentuent la vulnérabilité des véhicules, notamment dans les zones urbaines à forte densité. La question centrale de ce mémoire est donc la suivante : comment mettre en place un système d'alerte rapide et communautaire pour améliorer la gestion des cambriolages de véhicules et ainsi réduire les pertes et renforcer la sécurité publique?

% \section*{Justificatif}

% Les statistiques sur les cambriolages de véhicules montrent une tendance inquiétante qui nécessite une action immédiate.Bien que des systèmes de sécurité traditionnels, comme les alarmes et les dispositifs antivols, soient utilisés, ils n’offrent pas une solution complète. De plus, la technologie a évolué rapidement, offrant de nouvelles opportunités d’innovation, telles que la géolocalisation et les applications web et mobiles, qui pourraient permettre de répondre plus efficacement aux cambriolages de véhicules. Un tel système pourrait permettre une mobilisation rapide des citoyens et des autorités locales, créant ainsi une chaîne de prévention plus forte et plus réactive.

% \section*{Objectif}

% Ce mémoire a pour objectif de proposer une solution innovante à travers la création d’un système d'alerte communautaire destiné à améliorer la gestion des cambriolages de véhicules. Ce système serait fondé sur l’utilisation de technologies modernes, telles que les applications web et
% la géolocalisation  pour alerter rapidement toute la communauté dès qu’un vol est détecté. En analysant l'état actuel du phénomène, les technologies disponibles et les meilleures pratiques de sécurité, nous proposerons un modèle opérationnel pour déployer ce système dans les communautés, en tenant compte des défis locaux et des solutions possibles.

% \section*{Organisation du document}
% Ce mémoire est structuré de la manière suivante:

% \begin{itemize}
  
%     \item \textbf{Chapitre 1: Technologies et solutions existantes}: Ce chapitre explore les technologies actuellement disponibles pour la gestion des cambriolages de véhicules, y compris les systèmes de sécurité et les solutions de géolocalisation.
%     \item \textbf{Chapitre 2: Modélisation et conception UML}: Ce chapitre propose un modèle pour un système d'alerte communautaire innovant, basé sur les technologies actuelles. Il détaille la conception du système, ses fonctionnalités principales, son architecture, ainsi que son mode d’implémentation.
%     \item \textbf{Chapitre 3: Résultats et Discussion }: Ce chapitre présentent les résultats de ce système dans les communautés, les contraintes techniques et logistiques, ainsi que les perspectives d’amélioration du système.

    
% \end{itemize}






\introduction

\section*{Contexte}

Le cambriolage de véhicules constitue un phénomène de plus en plus préoccupant à travers le monde. Chaque année, des milliers de véhicules sont volés, entraînant d’importantes pertes financières pour les propriétaires et une dégradation du sentiment de sécurité au sein des communautés. Les malfaiteurs utilisent désormais des méthodes sophistiquées, rendant difficile la détection et la prévention. Ce type de crime représente donc un enjeu majeur de sécurité publique.

\section*{Problématique}

La complexité croissante des méthodes de vol, combinée au manque de systèmes d’alerte rapide et de coordination entre citoyens et autorités locales, accentue la vulnérabilité des véhicules. Les dispositifs traditionnels (alarmes, antivols) restent insuffisants. La problématique centrale de ce mémoire est la suivante :  
\textit{Comment concevoir un système d’alerte communautaire efficace, capable de renforcer la prévention et la gestion des cambriolages de véhicules ?}

\section*{Justification}

L’évolution des technologies, notamment la géolocalisation et les applications web/mobiles, offre aujourd’hui de nouvelles perspectives pour améliorer la sécurité. Un système collaboratif, reposant sur la mobilisation rapide des citoyens et des autorités, pourrait constituer une réponse efficace aux limites des solutions existantes.

\section*{Objectif}

L’objectif de ce mémoire est de concevoir un modèle opérationnel de système d’alerte communautaire basé sur les technologies modernes. Ce dispositif vise à :  
\begin{itemize}
    \item faciliter la déclaration et le signalement rapide des cambriolages,  
    \item renforcer la coopération entre citoyens et autorités,  
    \item proposer une solution adaptable aux réalités locales.  
\end{itemize}

\section*{Organisation du document}

Ce mémoire est structuré en trois chapitres principaux:  
\begin{itemize}
    \item \textbf{Chapitre 1: Technologies et solutions existantes} – Présentation des dispositifs actuels de gestion des cambriolages de véhicules, incluant les systèmes de sécurité et les solutions de géolocalisation.  
    \item \textbf{Chapitre 2: Modélisation et conception UML} – Proposition d’un modèle innovant de système d’alerte communautaire, détaillant l’architecture, les fonctionnalités et l’implémentation.  
    \item \textbf{Chapitre 3: Résultats et discussion} – Analyse des résultats, des contraintes techniques et organisationnelles, ainsi que des perspectives d’amélioration.  
\end{itemize}

%\lhead[]{} \rhead[]{} \chead[]{}
\selectlanguage{french}
\fancyhead[L]{\tiny \leftmark}
\fancyhead[R]{\scriptsize \rightmark}
\fancyfoot[C]{\thepage}

\chapter{Revue de littérature}\label{chap:1}
\phantomsection
\addcontentsline{toc}{section}{Introduction}
\section*{Introduction}
% \section{Introduction}
% Le cambriolage de véhicules est un problème majeur de sécurité publique à l’échelle mondiale, et malgré l’implémentation de dispositifs de sécurité traditionnels, il demeure un défi complexe à surmonter. Dans ce contexte, des technologies innovantes ont été mises en place pour lutter contre ce fléau. Ce document présente les principales plateformes permettant la lutte contre le vol de véhicules, en mettant en avant leurs fonctionnalités, avantages et limites.



La déclaration de vol de véhicule est une étape essentielle pour engager les poursuites judiciaires, faciliter la recherche du véhicule et permettre l’indemnisation par l’assurance.  
Plusieurs plateformes numériques et services administratifs existent pour accompagner les victimes dans cette démarche.  
Ce document présente une analyse détaillée des principaux sites utilisés pour la déclaration de vol de véhicule.

% \section{Définition et Concepts Clés}
% Avant de discuter des solutions actuelles, il est essentiel de définir quelques concepts clés pour mieux comprendre les technologies et méthodes utilisées dans la lutte contre le vol de véhicules.

% \subsection{Systèmes de Sécurité pour Véhicules}
% Les systèmes de sécurité pour véhicules sont des dispositifs technologiques conçus pour les protéger contre le vol et le vandalisme. Ces systèmes incluent des alarmes, des dispositifs de verrouillage électronique, des dispositifs de géolocalisation, ainsi que d’autres mécanismes visant à prévenir ou à détecter un vol en cours.

% \subsection{Géolocalisation et Traçabilité}
% La géolocalisation permet de suivre en temps réel la position d’un véhicule, notamment grâce à des technologies comme le GPS (Global Positioning System). Ces dispositifs permettent aux autorités de localiser un véhicule volé et de faciliter sa récupération rapide.


\section*{Plateformes principales de gestion et suivi des vols de véhicules}


% \subsection{Interpol-Fichier des Véhicules Volés (FVV)}

% \textbf{Description:}  
% Interpol met à disposition la base de données \textit{Stolen Motor Vehicles Database (SMV)}, qui recense les véhicules volés à l’échelle internationale et facilite le suivi transfrontalier. Cette plateforme permet aux forces de l’ordre de consulter et d’échanger des informations sur les véhicules volés. \cite{interpol2023}

% \textbf{Fonctionnalités:}
% \begin{itemize}
%     \item Enregistrement des véhicules volés par les pays membres.
%     \item Consultation en temps réel par les forces de l’ordre.
%     \item Échange sécurisé d’informations entre États.
%     \item Identification des véhicules retrouvés à l’étranger.
%     \item Base de données mondiale des véhicules volés.
%     \item Accès en temps réel pour les forces de l’ordre.
%     \item Coopération internationale grâce aux Bureaux Nationaux Interpol (NCB).
% \end{itemize}

% \textbf{Avantages:}
% \begin{itemize}
%     \item Portée mondiale.
%     \item Améliore la rapidité des enquêtes transfrontalières.
%     \item Données fiables mises à jour régulièrement.
%     \item Couverture internationale permettant le suivi des véhicules volés dans plusieurs pays.  
%     \item Facilite la coopération entre forces de l’ordre et augmente les chances de récupération des véhicules.  
%     \item Accès centralisé aux informations sur les vols, réduisant les redondances.
% \end{itemize}

% \textbf{Limites:}
% \begin{itemize}


%      \item Accès réservé aux forces de l’ordre et aux agences partenaires, donc limité pour les particuliers.  
%     \item Dépendance à la mise à jour régulière des informations par chaque pays.  
%     \item Ne couvre pas les vols non déclarés ou informels.
% \end{itemize}





% \subsection{ Europol – Schengen Information System (SIS II)}
% \textbf{Description :}  
% Le \textit{Schengen Information System (SIS II)} est une base de données européenne utilisée par les États membres pour suivre les véhicules volés et autres informations relatives à la sécurité. Elle permet le signalement immédiat et la consultation en temps réel des véhicules volés dans l’espace Schengen. \cite{europol2023}
% \textbf{Fonctionnalités :}
% \begin{itemize}
%     \item Signalement immédiat des véhicules volés.
%     \item Consultation en temps réel par les autorités frontalières et policières.
%     \item Interconnexion avec d’autres systèmes de sécurité européens.
% \end{itemize}
% \textbf{Avantages :}  
% \begin{itemize}
%     \item Couverture efficace des pays membres de l’espace Schengen.  
%     \item Permet une réaction rapide des autorités grâce à la consultation en temps réel.  
%     \item Intègre les véhicules volés dans une approche globale de sécurité publique.
% \end{itemize}

% \textbf{Limites :}  
% \begin{itemize}
%     \item Réservé aux autorités publiques et aux agents habilités.  
%     \item Limité à l’espace Schengen, ne couvre pas les véhicules volés en dehors de cette zone.  
%     \item Nécessite une infrastructure informatique fiable pour un accès et une mise à jour efficaces.
% \end{itemize}

% \subsection{Service Public France – Déclarer un vol de véhicule}
% \textbf{Description :}  
% Le site officiel du \textit{Service Public} permet aux particuliers de déclarer rapidement un vol de véhicule en ligne, facilitant la prise en charge par les forces de l’ordre et la transmission aux compagnies d’assurance. \cite{servicepublic2023}

% \textbf{Fonctionnalités :}
% \begin{itemize}
%     \item Déclaration en ligne du vol de véhicule.
%     \item Génération d’un récépissé officiel.
%     \item Transmission automatique aux forces de l’ordre.
%     \item Utilisation du dossier pour les assurances.
% \end{itemize}

% \textbf{Avantages :}  
% \begin{itemize}
%     \item Accès facile pour tous les citoyens français, simplifiant la procédure de déclaration.  
%     \item Permet une traçabilité officielle des véhicules volés.  
%     \item Peut accélérer les démarches administratives et d’assurance.
% \end{itemize}

% \textbf{Limites :}  
% \begin{itemize}
%     \item Limité au territoire français.  
%     \item Ne fournit pas de suivi en temps réel ni de géolocalisation du véhicule.  
%     \item La récupération du véhicule dépend entièrement des forces de l’ordre et du suivi administratif.
% \end{itemize}


% \section{Plateforme de déclaration de vol/perte – DGPR (Bénin)}

% \subsection{Description}
% La plateforme de la \textit{Direction Générale de la Police Républicaine} (DGPR) du Bénin permet de déclarer en ligne des vols ou pertes, y compris potentiellement pour un véhicule.  
% Ce service est utilisé pour lancer une procédure officielle auprès de la police républicaine du Bénin.

% \subsection{Lien}
% \begin{itemize}
%     \item \url{https://www.dgpr.bj/declaration-de-vol-perte/}
% \end{itemize}

% \subsection{Fonctionnalités}
% \begin{itemize}
%     \item Formulaire en ligne pour déclarer un vol ou une perte.
%     \item Saisie des informations personnelles et de l’objet volé (éventuellement véhicule).
%     \item Transmission directement aux services de police compétents.
%     \item Possibilité d’être contacté par la police suite à la déclaration.
% \end{itemize}

% \subsection{Avantages}
% \begin{itemize}
%     \item Service officiel de la police nationale du Bénin.
%     \item Permet de déclarer rapidement un vol sans présence immédiate en commissariat.
%     \item Accessible via Internet.
% \end{itemize}

% \subsection{Limites}
% \begin{itemize}
%     \item La plate-forme ne garantit pas à elle seule une plainte complète (un suivi peut être requis).
%     \item Possible nécessité de se rendre physiquement au poste de police pour finaliser la procédure.
%     \item Interface limitée selon les capacités techniques de la plateforme.
% \end{itemize}

% %------------------------------------------------

% \section{Déclaration de vol de véhicule – Collectivité de Saint-Martin}

% \subsection{Description}
% La Collectivité de Saint-Martin propose une page dédiée aux démarches administratives pour les véhicules, incluant la procédure à suivre en cas de vol de véhicule à Saint-Martin (Antilles françaises). :contentReference[oaicite:2]{index=2}

% \subsection{Lien}
% \begin{itemize}
%     \item \url{https://www.comstmartin.fr/demarches_administratives}
% \end{itemize}

% \subsection{Fonctionnalités}
% \begin{itemize}
%     \item Informations sur la procédure de déclaration de vol de véhicule.
%     \item Indications claires pour effectuer le dépôt de plainte auprès de la Gendarmerie.
%     \item Instructions pour transmettre la déclaration à l’assurance et au Service des titres de circulation (via e-mail ou contact administratif). :contentReference[oaicite:3]{index=3}
%     \item Contacts utiles du \textit{Service des titres de circulation} (adresse, téléphone, e-mail). :contentReference[oaicite:4]{index=4}
% \end{itemize}

% \subsection{Avantages}
% \begin{itemize}
%     \item Adapté aux démarches spécifiques à Saint-Martin.
%     \item Fournit des informations claires sur les contacts administratifs locaux. :contentReference[oaicite:5]{index=5}
%     \item Permet de connaître l’ordre des démarches (police, assurance, services administratifs). :contentReference[oaicite:6]{index=6}
% \end{itemize}

% \subsection{Limites}
% \begin{itemize}
%     \item Il ne s’agit pas d’un dépôt de plainte en ligne automatisé.
%     \item L’usager doit effectuer physiquement certaines démarches (déposer plainte en gendarmerie).
%     \item Nécessite souvent l’envoi de courriels ou de documents physiques par le propriétaire. :contentReference[oaicite:7]{index=7}
% \end{itemize}

% %------------------------------------------------

% \section{DIGITPOL – Base internationale de véhicules volés}

% \subsection{Description}
% DIGITPOL est une plateforme internationale permettant d’enregistrer un véhicule volé dans une base de données mondiale.

% \subsection{Fonctionnalités}
% \begin{itemize}
%     \item Enregistrement international du véhicule volé.
%     \item Diffusion des informations aux partenaires internationaux.
%     \item Vérification du statut d’un véhicule.
% \end{itemize}

% \subsection{Avantages}
% \begin{itemize}
%     \item Portée internationale.
%     \item Utile pour les vols transfrontaliers.
%     \item Complément aux démarches nationales.
% \end{itemize}

% \subsection{Limites}
% \begin{itemize}
%     \item Ne constitue pas une plainte officielle.
%     \item Dépend de la coopération internationale.
% \end{itemize}







% \section{Plateformes principales de gestion et de suivi des vols de véhicules}







%------------------------------------------------

\section{Plateforme de déclaration de vol/perte – DGPR (Bénin)}

\subsection{Fonctionnalités}
La plateforme de la \textit{Direction Générale de la Police Républicaine (DGPR)} du Bénin permet aux citoyens de déclarer en ligne des vols ou pertes, y compris ceux concernant des véhicules \cite{dgpr2023}.

\subsection{Fonctionnalités :}
\begin{itemize}
    \item Formulaire de déclaration en ligne.
    \item Transmission aux services de police compétents.
    \item Possibilité de contact ultérieur par la police.
\end{itemize}

\subsection{Avantages :}
\begin{itemize}
    \item Service officiel de la police béninoise.
    \item Réduction des déplacements initiaux.
    \item Accessibilité via Internet.
\end{itemize}

\subsection{Limites :}
\begin{itemize}
    \item Une validation physique peut être exigée.
    \item Fonctionnalités numériques limitées.
\end{itemize}

%------------------------------------------------

\section{Interpol – Fichier des Véhicules Volés (FVV)}

\subsection{Description :}  
Interpol met à disposition la base de données \textit{Stolen Motor Vehicles Database (SMV)}, qui recense les véhicules volés à l’échelle internationale et facilite le suivi transfrontalier.  
Cette plateforme est utilisée exclusivement par les forces de l’ordre afin de consulter et d’échanger des informations sur les véhicules volés \cite{interpol2023}.

\subsection{Fonctionnalités :}
\begin{itemize}
    \item Enregistrement des véhicules volés par les pays membres.
    \item Consultation en temps réel par les forces de l’ordre.
    \item Échange sécurisé d’informations entre États.
    \item Identification des véhicules retrouvés à l’étranger.
    \item Coopération internationale via les Bureaux Nationaux Interpol (NCB).
\end{itemize}

\subsection{Avantages :}
\begin{itemize}
    \item Portée mondiale.
    \item Amélioration des enquêtes transfrontalières.
    \item Données fiables et régulièrement mises à jour.
    \item Centralisation des informations sur les véhicules volés.
\end{itemize}

\subsection{Limites :}
\begin{itemize}
    \item Accès strictement réservé aux forces de l’ordre.
    \item Dépendance à la qualité des mises à jour nationales.
    \item Les vols non déclarés ne sont pas pris en compte.
\end{itemize}
%------------------------------------------------


\section{DIGITPOL Automobile– Expertise et sécurité des véhicules}
\subsection{Description}
Digitpol Automotive est une unité spécialisée dédiée aux véhicules et à la sécurité, combinant criminalistique automobile et télématique avancée.  
Elle offre une expertise approfondie dans la localisation des véhicules volés, les enquêtes sur les fraudes à l’assurance, la sécurité embarquée et les applications télématiques sur mesure pour les services gouvernementaux et de sécurité \cite{digitpol2023}.  
L’équipe est composée d’experts hautement qualifiés, ayant une expérience dans les forces de l’ordre et les opérations techniques secrètes, capables de réaliser des enquêtes approfondies et de fournir des analyses médico-légales de véhicules.

\subsection{Fonctionnalités}
\begin{itemize}
    \item Lecture automatique des plaques d’immatriculation (LAPI).
    \item Développement et intégration de solutions IoT pour véhicules.
    \item Dispositifs audio discrets et installation de traceurs GPS.
    \item Applications télématiques embarquées et personnalisées pour services gouvernementaux.
    \item Expertises médico-légales de véhicules et analyses criminologiques.
    \item Enquêtes sur les fraudes à l’assurance et vols de véhicules.
    \item Localisation et traçage de véhicules volés.
\end{itemize}

\subsection{Avantages}
\begin{itemize}
    \item Expertise combinant criminalistique, télématique et sécurité des véhicules.
    \item Solutions sur mesure adaptées aux besoins gouvernementaux et sécuritaires.
    \item Renforcement de la prévention et de la détection des vols et fraudes.
    \item Capacité à mener des enquêtes complexes et à fournir des preuves fiables.
    \item Technologies avancées pour la localisation et le suivi des véhicules.
\end{itemize}

\subsection{Limites}
\begin{itemize}
    \item Les services sont principalement destinés aux autorités ou organisations spécialisées, pas aux particuliers.
    \item Dépendance à l’accès aux technologies embarquées et aux données télématiques.
    \item Ne remplace pas les procédures légales officielles pour déclarer un vol.
    \item Certaines fonctionnalités requièrent une expertise technique pour leur installation et utilisation.
\end{itemize}




%------------------------------------------------

\section{Europol – Schengen Information System (SIS II)}

\subsection{Description :}  
Le \textit{Schengen Information System (SIS II)} est une base de données européenne permettant aux États membres de signaler et de consulter les informations relatives aux véhicules volés dans l’espace Schengen \cite{europol2023}.

\textbf{Fonctionnalités :}
\begin{itemize}
    \item Signalement immédiat des véhicules volés.
    \item Consultation en temps réel par les autorités policières et frontalières.
    \item Interconnexion avec d’autres systèmes européens de sécurité.
\end{itemize}

\subsection{Avantages :}
\begin{itemize}
    \item Couverture efficace de l’espace Schengen.
    \item Réactivité accrue des forces de l’ordre.
    \item Intégration dans une stratégie globale de sécurité européenne.
\end{itemize}

\subsection{Limites :}
\begin{itemize}
    \item Europol ne reçoit pas directement les signalements du public.
    \item Elle ne dispose d’aucun pouvoir d’arrestation ou d’enquête autonome.
    \item Son action dépend entièrement de la coopération et des informations fournies par les États membres.
    \item Les interventions sont limitées aux affaires présentant une dimension internationale.
    \item Réservé aux autorités habilitées.
    \item Limité géographiquement à l’espace Schengen.
    \item Dépendance à une infrastructure informatique performante.
\end{itemize}

%------------------------------------------------

\section{Service Public France – Déclaration de vol de véhicule}

\subsection{Description :}  
La plainte en ligne permet aux victimes de déclarer à distance le vol de leur véhicule et d’obtenir les informations nécessaires pour les démarches auprès des forces de l’ordre et des compagnies d’assurance \cite{servicepublic2023}.
Ce dispositif officiel, gratuit et accessible via Internet, vise à simplifier le dépôt de plainte sans nécessiter un déplacement immédiat en commissariat ou en brigade de gendarmerie.  
Il s’applique aux vols de véhicules commis sur le territoire français et constitue une première étape essentielle pour les démarches administratives et assurantielles.

\subsection{Fonctionnalités}
\begin{itemize}
    \item Déclaration en ligne du vol de véhicule.
    \item Authentification sécurisée, notamment via FranceConnect.
    \item Formulaire détaillé permettant de renseigner les informations du véhicule (immatriculation, circonstances du vol, lieu, date).
    \item Génération d’un accusé de réception après validation de la plainte.
    \item Mise à disposition du procès-verbal de plainte dans l’espace usager.
    \item Possibilité d’être contacté par la police ou la gendarmerie pour compléter la déclaration.
\end{itemize}

\subsection{Avantages}
\begin{itemize}
    \item Gain de temps grâce au dépôt de plainte à distance.
    \item Réduction des déplacements initiaux en commissariat ou gendarmerie.
    \item Procédure officielle reconnue par les assurances.
    \item Accessibilité pour les résidents et les visiteurs étrangers victimes d’un vol en France.
\end{itemize}

\subsection{Limites}
\begin{itemize}
    \item Limité aux vols de véhicules dont l’auteur est inconnu.
    \item Un déplacement physique peut être exigé pour finaliser ou compléter la plainte.
    \item Ne permet pas le suivi en temps réel de la recherche du véhicule.
    \item L’annulation de la plainte ne peut pas être effectuée en ligne.
\end{itemize}


%------------------------------------------------

\section{Déclaration de vol de véhicule – Collectivité de Saint-Martin}

\subsection{Description :}  
La Collectivité de Saint-Martin met à disposition une page d’information détaillant les démarches administratives à effectuer en cas de vol de véhicule sur son territoire \cite{saintmartin2023}.

\textbf{Fonctionnalités :}
\begin{itemize}
    \item Présentation de la procédure officielle.
    \item Orientation vers la gendarmerie.
    \item Instructions pour les démarches auprès de l’assurance et des services administratifs.
\end{itemize}

\subsection{Avantages :}
\begin{itemize}
    \item Adapté au contexte local de Saint-Martin.
    \item Informations claires et structurées.
\end{itemize}

\textbf{Limites :}
\begin{itemize}
    \item Absence de dépôt de plainte entièrement en ligne.
    \item Démarches partiellement manuelles.
\end{itemize}






































% \subsection{ Vehicle Tracking Solutions (VTS)}
% \textbf{Description :}  
% VTS offre des services de suivi en temps réel et de récupération de véhicules pour particuliers et flottes professionnelles \cite{vts2023}.

% \textbf{Fonctionnalités :}
% \begin{itemize}
%     \item Géolocalisation en temps réel.
%     \item Alertes automatiques en cas de déplacement suspect.
%     \item Historique des trajets.
%     \item Gestion de flottes professionnelles.
% \end{itemize}

% \textbf{Avantages :}  
% \begin{itemize}
%     \item Suivi en temps réel via GPS.  
%     \item Notifications instantanées en cas de mouvement suspect.  
%     \item Convient aux particuliers et aux flottes.
% \end{itemize}

% \textbf{Limites :}  
% \begin{itemize}
%     \item Nécessite un abonnement et un équipement GPS.  
%     \item Dépend de la couverture réseau pour le suivi.
% \end{itemize}

% \subsection{LoJack Corporation}
% \textbf{Description :}  
% LoJack propose des solutions anti-vol avec géolocalisation pour récupérer rapidement les véhicules volés \cite{lojack2023}.

% \textbf{Fonctionnalités :}
% \begin{itemize}
%     \item Localisation du véhicule après vol.
%     \item Transmission des données aux autorités.
%     \item Suivi discret du véhicule.
% \end{itemize}


% \textbf{Avantages :}  
% \begin{itemize}
%     \item Technologie éprouvée de récupération rapide.  
%     \item Compatible avec la police et les autorités locales.  
%     \item Installation simple pour particuliers et flottes.
% \end{itemize}

% \textbf{Limites :}  
% \begin{itemize}
%     \item Abonnement nécessaire.  
%     \item Fonctionne mieux dans les zones couvertes par le réseau LoJack.
% \end{itemize}

% \subsection{Carlock}
% \textbf{Description :}  
% Carlock fournit un suivi en temps réel des véhicules avec alertes instantanées en cas de mouvement non autorisé \cite{carlock2023}.

% \textbf{Fonctionnalités :}
% \begin{itemize}
%     \item Détection de mouvement non autorisé.
%     \item Alertes instantanées sur smartphone.
%     \item Suivi GPS en temps réel.
%     \item Historique des déplacements.
% \end{itemize}

% \textbf{Avantages :}  
% \begin{itemize}
%     \item Alertes immédiates sur smartphone.  
%     \item Suivi en temps réel pour les flottes et particuliers.  
%     \item Historique complet des déplacements.
% \end{itemize}

% \textbf{Limites :}  
% \begin{itemize}
%     \item Nécessite un abonnement et un boîtier installé dans le véhicule.  
%     \item Dépend du réseau mobile pour les notifications.
% \end{itemize}

% \subsection{Whistle}
% \textbf{Description :}  
% Whistle est un système de suivi intelligent de véhicules, avec alertes instantanées et rapports d’activité \cite{whistle2023}.

% \textbf{Fonctionnalités :}
% \begin{itemize}
%     \item Suivi GPS précis.
%     \item Alertes en cas de vol ou déplacement suspect.
%     \item Rapports et statistiques d’utilisation.
% \end{itemize}



% \textbf{Avantages :}  
% \begin{itemize}
%     \item Suivi GPS précis et notifications en temps réel.  
%     \item Historique des déplacements consultable à tout moment.  
%     \item Compatible avec les smartphones.
% \end{itemize}

% \textbf{Limites :}  
% \begin{itemize}
%     \item Abonnement requis pour le service complet.  
%     \item Dépendance à la couverture réseau et à l’installation correcte du dispositif.
% \end{itemize}

% \subsection{Genetec – ANPR Systems}
% \textbf{Description :}  
% Genetec propose des systèmes de reconnaissance automatique des plaques d’immatriculation (ANPR) pour identifier les véhicules volés \cite{genetec2022}.

% \textbf{Fonctionnalités :}
% \begin{itemize}
%     \item Lecture automatique des plaques d’immatriculation.
%     \item Comparaison avec les bases de données policières.
%     \item Détection en temps réel des véhicules recherchés.
% \end{itemize}


% \textbf{Avantages :}  
% \begin{itemize}
%     \item Repérage automatisé des véhicules volés sur routes et parkings.  
%     \item Intégration avec les bases de données de la police.  
%     \item Surveillance passive 24/7.
% \end{itemize}

% \textbf{Limites :}  
% \begin{itemize}
%     \item Installation coûteuse.  
%     \item Nécessite des caméras et infrastructure technique appropriée.
% \end{itemize}









% % \section{Analyse des Plateformes de Lutte contre le Vol de Véhicules}

% \subsection{Europol-Système d’Information Schengen (SIS II)}

% \textbf{Description:}  
% Le SIS II est un système européen permettant le partage des signalements de véhicules volés entre les pays de l’espace Schengen. Il facilite la coopération policière transfrontalière.

% \textbf{Fonctionnalités:}
% \begin{itemize}
%     \item Signalement et recherche de véhicules volés.
%     \item Interconnexion avec les bases policières nationales.
%     \item Accès instantané aux informations pour toutes les polices Schengen.
% \end{itemize}

% \textbf{Avantages:}
% \begin{itemize}
%     \item Couverture complète de l’espace Schengen.
%     \item Facilite l’interopérabilité entre les pays européens.
% \end{itemize}

% \textbf{Limites:}
% \begin{itemize}
%     \item Réservé à l’Europe (zone Schengen).
%     \item Ne couvre pas les vols en dehors de cette zone.
% \end{itemize}

% \subsection{Stolen Vehicle Recovery (SVR)}

% \textbf{Description:}  
% SVR est une solution de suivi GPS pour localiser un véhicule volé en temps réel. Elle repose sur des boîtiers installés dans le véhicule et connectés aux réseaux GSM/GPS

% \textbf{Fonctionnalités:}
% \begin{itemize}
%     \item Suivi GPS en temps réel.
%     \item Notifications automatiques en cas de vol.
%     \item Historique des déplacements.
% \end{itemize}

% \textbf{Avantages:}
% \begin{itemize}
%     \item Rapidité de localisation et de récupération.
%     \item Compatible avec différents types de véhicules.
% \end{itemize}

% \textbf{Limites:}
% \begin{itemize}
%     \item Abonnement requis.
%     \item Dépendance à la couverture GSM/GPS
% \end{itemize}
% Les services de géolocalisation permettent d’améliorer significativement la récupération des véhicules volés \cite{vts2023}.

% \subsection{LoJack}

% \textbf{Description:}  
% LoJack est une solution de récupération de véhicules volés utilisant un émetteur radio caché. Contrairement au GPS, il fonctionne même dans des zones à faible signal.

% \textbf{Fonctionnalités:}
% \begin{itemize}
%     \item Émetteur radio intégré au véhicule.
%     \item Localisation par les forces de l’ordre.
%     \item Activation après signalement du vol.
% \end{itemize}

% \textbf{Avantages:}
% \begin{itemize}
%     \item Système discret et difficile à neutraliser.
%     \item Ne dépend pas d’un réseau GPS
% \end{itemize}

% \textbf{Limites:}
% \begin{itemize}
%     \item Présence limitée géographiquement (principalement USA).
%     \item Installation professionnelle obligatoire.
% \end{itemize}

% Les solutions de récupération de véhicules basées sur la géolocalisation contribuent efficacement à la lutte contre le vol automobile \cite{lojack2023}.


% \subsection{Applications Mobiles Communautaires (Carlock, Whistle)}

% \textbf{Description:}  
% Ces applications permettent aux propriétaires de véhicules de recevoir des alertes et de suivre leur voiture grâce à un smartphone et un boîtier connecté.

% \textbf{Fonctionnalités:}
% \begin{itemize}
%     \item Alertes en cas de mouvement suspect.
%     \item Suivi GPS via smartphone.
%     \item Partage d’informations avec une communauté.
% \end{itemize}

% \textbf{Avantages:}
% \begin{itemize}
%     \item Accessibles au grand public.
%     \item Interface simple et intuitive.
% \end{itemize}

% \textbf{Limites:}
% \begin{itemize}
%     \item Dépendance au smartphone de l’utilisateur.
%     \item Efficacité variable selon les applications.
% \end{itemize}

% Les systèmes de suivi en temps réel améliorent la sécurité des véhicules \cite{carlock2023,whistle2023}.


% \subsection{ANPR -Reconnaissance Automatique de Plaques}

% \textbf{Description:}  
% Les systèmes ANPR utilisent des caméras intelligentes pour lire et comparer les plaques d’immatriculation avec les bases de données de véhicules volés.

% \textbf{Fonctionnalités:}
% \begin{itemize}
%     \item Scan automatique des plaques.
%     \item Détection en temps réel.
%     \item Alertes aux forces de l’ordre.
% \end{itemize}

% \textbf{Avantages:}
% \begin{itemize}
%     \item Détection automatisée et rapide.
%     \item Large couverture grâce aux caméras fixes et mobiles.
% \end{itemize}

% \textbf{Limites:}
% \begin{itemize}
%     \item Coût élevé des infrastructures.
%     \item Problèmes de confidentialité.
% \end{itemize}
% Les systèmes de reconnaissance automatique des plaques d’immatriculation permettent de localiser et suivre efficacement les véhicules \cite{genetec2022}.


% \subsection{Carfax}

% \textbf{Description:}  
% Carfax est une base de données sur l’historique des véhicules, principalement utilisée en Amérique du Nord, qui permet de vérifier si un véhicule a déjà été volé.

% \textbf{Fonctionnalités:}
% \begin{itemize}
%     \item Rapports d’historique de véhicules.
%     \item Vérification des vols signalés.
%     \item Intégration avec bases policières et assurances.
% \end{itemize}

% \textbf{Avantages:}
% \begin{itemize}
%     \item Réduit le risque d’acheter un véhicule volé.
%     \item Service reconnu par les acheteurs et concessionnaires.
% \end{itemize}

% \textbf{Limites:}
% \begin{itemize}
%     \item Service payant.
%     \item Couverture limitée hors Amérique du Nord.
% \end{itemize}
% Les rapports d’historique de véhicules fournissent des informations cruciales pour évaluer la sécurité et l’état des véhicules \cite{carfax2023}.



% \subsection{Plateformes Locales de Signalement (France)}

% \textbf{Description:}  
% En France, plusieurs plateformes officielles permettent aux citoyens de déclarer un vol de véhicule et d’accéder aux informations centralisées.

% \textbf{Fonctionnalités:}
% \begin{itemize}
%     \item Déclaration en ligne du vol.
%     \item Consultation publique des véhicules volés.
%     \item Lien direct avec la police et la gendarmerie.
% \end{itemize}

% \textbf{Avantages:}
% \begin{itemize}
%     \item Accessibles à tous les citoyens.
%     \item Procédure simplifiée de signalement.
% \end{itemize}

% \textbf{Limites:}
% \begin{itemize}
%     \item Portée limitée à un seul pays.
%     \item Efficacité dépendante des forces locales.
% \end{itemize}
% Les démarches officielles pour déclarer un vol de véhicule sont décrites sur le site du Service Public \cite{servicepublic2023}.




% \section{Enjeux et Défis}

% Malgré la diversité des solutions existantes, plusieurs défis persistent :

% \begin{itemize}
%     \item \textbf{Interopérabilité des systèmes} : difficulté à faire communiquer les plateformes entre pays et acteurs privés.
%     \item \textbf{Protection des données personnelles} : risque de surveillance excessive et d’atteinte à la vie privée.
%     \item \textbf{Inégalités d’accès technologique} : coûts élevés limitant l’adoption dans les pays en développement.
%     \item \textbf{Dépendance technologique} : vulnérabilité face au brouillage GPS ou aux sabotages.
% \end{itemize}


\section{Conclusion}

Le vol de véhicules demeure un défi majeur de sécurité publique nécessitant une approche multidimensionnelle. Les plateformes de lutte contre le vol de véhicules, qu’elles soient institutionnelles, technologiques ou communautaires, jouent un rôle essentiel dans la prévention, la détection et la récupération des véhicules volés.

Aucune solution unique ne peut répondre à l’ensemble des problématiques liées au vol automobile. Une combinaison de technologies, associée à une coopération internationale renforcée et à une sensibilisation des usagers, constitue la stratégie la plus efficace pour réduire durablement ce phénomène.


\chapter{Modélisation et Comception UML}\label{chap:2}



\section{Introduction}
Dans cette section, nous allons aborder la modélisation et la conception du système en prenant en compte les différentes méthodologies et outils de modélisation disponibles. Nous justifierons le choix de la méthode UML, un standard largement utilisé dans la conception de systèmes informatiques, pour la modélisation de notre projet.

\section{UML (Unified Modeling Language)}
\gls{UML} est une méthode de modélisation orientée objet qui permet de représenter visuellement des systèmes complexes. Il est principalement utilisé pour modéliser les logiciels à travers des diagrammes illustrant les aspects statiques et dynamiques du système. UML se compose de plusieurs types de diagrammes, y compris les diagrammes de cas d’utilisation, les diagrammes de classes et les diagrammes de séquences.

\section{Choix de la méthode de modélisation}
Pour ce projet, nous avons opté pour l'utilisation de la méthode UML en raison de sa capacité à modéliser des systèmes orientés objet et de sa popularité dans le domaine du développement logiciel moderne. UML permet de créer des diagrammes  de cas d'utilisations, le diagramme de classes, des diagrammes de séquences  qui sont essentiels pour une bonne compréhension du système dans sa globalité.

\section{Modélisation UML}

\subsection{Outils de Modélisation}
Nous avons utilisé l’outil \textbf{\gls{Dia}}
pour réaliser les diagrammes de cas d’utilisation, de classes et de séquences.  
Il s’agit d’un exemple parmi plusieurs outils disponibles pour la modélisation UML, et il facilite la clarté de la documentation ainsi que la standardisation des modèles, contribuant à une meilleure compréhension et maintenance du système.

\subsection{Identification des acteurs du système}
L’identification des acteurs du système est une étape clé dans la modélisation d’un système basé sur UML. Un acteur représente un rôle joué par un utilisateur ou un autre système qui interagit avec le système à modéliser. Dans notre système, les acteurs principaux sont les citoyens, les administrateurs et la police. Ces acteurs interagiront avec le système pour déclarer un vol, signaler un véhicule retrouvé, accéder à des données ou effectuer des actions de sécurité.



\newpage
\subsection{ Diagramme de cas d’utilisation}
Le diagramme de cas d'utilisation est utilisé pour décrire les interactions entre les utilisateurs (acteurs) et le système. Chaque cas d'utilisation représente une fonctionnalité du système, et l’interaction entre l’acteur.


	% \textbf{\underline{le Diagramme de Cas d’Utilisation du système} }\\	
	
\begin{figure}[h]
\includegraphics[width = 16.5cm,height = 16cm]{Diagramme1.png}
\caption{le Diagramme de Cas d’Utilisation du système.}
 \label{fig:exemple_figure}
\end{figure}
%\vspace{0.5cm}



\begin{enumerate}
    \item Creer un compte\\
	\textbf{\underline{Description textuelle} }\\	
	\begin{itemize}
		\item L'utilisateur cré un compte en fournissant ces informations personnelles.
		\item L'application vérifie sur si le compte existe déjà.
		\item Si le compte existe déjà ,l'application affiche un message d'erreur
		\item Si le compte n'existe pas , l'application crée le compte en affichant un message de confirmation.  
	\end{itemize}
	\item Créer Declaration\\
	\textbf{\underline{Description textuelle} }\\
		\begin{itemize}			
			\item L'utilisateur cré une declaration en indiquant le type de la propriété volé ainsi que les informations disponibles qui l'identifient.
		\end{itemize}
	\item Signaler vehicule\\
	\textbf{\underline{Description textuelle} }\\
	\begin{itemize}
		\item L'utilisateur peut signaler la localisation d'un vehicule recherché.
	\end{itemize}
	\item Payer\\
	\textbf{\underline{Description textuelle} }\\
	\begin{itemize}
		\item Les utilisateurs devrons payer pour chaque declaration de vol.
	\end{itemize}
	\item Se renseigner\\
	\textbf{\underline{Description textuelle} }\\
	\begin{itemize}
		\item Les utilisateurs peuvent se renseigner par une discussion instantané au près de la police .
	\end{itemize}
	
	\item Consulter liste declaration\\
	\textbf{\underline{Description textuelle} }\\
	\begin{itemize}
		\item Les utilisateurs peuvent consulter la liste de toutes les declaration.
	\end{itemize}
	\item Gerer Declaration\\
	\textbf{\underline{Description textuelle} }\\
	\begin{itemize}
		\item L'orsqu'un vehicule déclaré est retrouvé , l'agent policier ou Administrateur modifie l'etat de la declaration concernée.
	\end{itemize}
	\item Gérer Rapport\\
	\textbf{\underline{Description textuelle} }\\
	\begin{itemize}
		\item Créer un rapport de vol quotidien contenant toutes les déclarations de vols effectuées.
	\end{itemize}
	\item Gérer compte\\
	\textbf{\underline{Description textuelle} }\\
	\begin{itemize}
		\item L'Administrateur ou le superadministrateur est responsable de la gestion des comptes utilisateurs et polices.Il peut suprimer , bloquer un compte.
	\end{itemize}
	\begin{itemize}
		\item Le superadministrateur est responsable de la gestion des administrateurs.Il peut ajouter , modifier et supprimer un administrateur.
	\end{itemize}
\end{enumerate}
\newpage

\subsection{Diagramme des classes}
Le diagramme de classes est utilisé pour décrire les objets du système et leurs relations. Dans ce diagramme, chaque classe représente une entité du système, et les relations entre ces classes (comme l'héritage, l'association, etc.) sont clairement indiquées. Par exemple, une classe "Véhicule" pourrait être liée à une classe "Propriétaire" avec une relation d'association.

\textbf{\underline{Le Diagramme de Classe du système} }
\begin{figure}[h]
\includegraphics[width = 17cm,height = 16cm]{classe.png}
\caption{Le Diagramme de Classe du système}
\label{fig:exemple_figure2}
\end{figure}
\vspace{0.5cm}
\begin{enumerate}
	\item Utilisateur\\
     	\textbf{\underline{Description textuelle} }\\
		\begin{itemize}
			\item La classe Utilisateur représente les différents types d'utilisateurs du système.
		\end{itemize}
		\item Declaration\\
	\textbf{\underline{Description textuelle} }\\
		\begin{itemize}
			\item La classe Declaration contient les informations sur les déclarations de vol, avec un lien vers les utilisateurs concernés.
		\end{itemize}
		\item Rapport\\
	\textbf{\underline{Description textuelle} }\\
		\begin{itemize}
			\item La classe Rapport regroupe les déclarations pour une date donnée, avec des listes pour les nouvelles, modifiées et résolues.
		\end{itemize}
		\item Police\\
	\textbf{\underline{Description textuelle} }\\
		\begin{itemize}
			\item La classe Police regroupe les differents types de force de l'ordre. 
		\end{itemize}
		\item Admin\\
	\textbf{\underline{Description textuelle} }\\
		\begin{itemize}
			\item La classe Admin représente les administrateurs du système. 
		\end{itemize}
	\end{enumerate}
\vspace{0.5cm}
Les liens entre les classes montrent les relations entre les différents éléments du système.\\
Ce diagramme de classe capture les principales entités et leurs interactions pour la gestion des cambriolages de véhicules dans ce système.\par
%\vspace{0.5cm}
\newpage





\subsection{Diagramme de séquences}
Le diagramme de séquences illustre l’ordre des messages échangés entre les objets du système pendant l’exécution d’un scénario particulier. Ce type de diagramme permet de comprendre comment les acteurs et le système interagissent au fil du temps pour accomplir une tâche donnée.  
Dans notre projet, les principaux cas d’utilisation représentés sont: déclarer un vol, diffuser une notice et alerter sur un véhicule recherché.

\textbf{\underline{Déclarer un vol:}}  
Le diagramme de séquence du scénario  Déclarer un vol  montre l’interaction entre l’utilisateur et le système.  
\begin{itemize}
    \item L’utilisateur remplit un formulaire en fournissant les informations sur le véhicule et les détails du vol (lieu, date, description).
    \item Le système vérifie la validité des données, enregistre la déclaration et associe l’information au compte de l’utilisateur.
    \item Une confirmation est ensuite envoyée à l’utilisateur, attestant que la déclaration a bien été prise en compte.
\end{itemize}

\begin{figure}[h]
    \centering
    \includegraphics[width=16cm, height=14.5cm]{sequence1.png}
    \caption{Diagramme de séquence pour le cas d’utilisation  Déclarer un vol .}
\end{figure}

\newpage

\textbf{\underline{Notice et diffusion:}}  
Le diagramme de séquence pour  Notice et diffusion  illustre le processus de gestion de l’information après la déclaration d’un vol.  
\begin{itemize}
    \item Le système génère automatiquement une notice contenant les informations essentielles sur le véhicule déclaré. La couleur de diffusion varie selon le statut de la déclaration : 
		\begin{itemize}
			\item \textbf{Rouge}: lorsque le véhicule est déclaré \textit{volé},  
			\item \textbf{Jaune}: lorsqu’un véhicule est \textit{signalé} mais pas encore retrouvé,  
			\item \textbf{Vert}: lorsque le véhicule est \textit{retrouvé}.  
		\end{itemize}

    \item Cette notice est diffusée en temps réel vers les utilisateurs de cette application web.
    % \item L’administrateur peut intervenir pour valider, modifier ou compléter les informations diffusées.
\end{itemize}

\begin{figure}[h]
    \centering
    \includegraphics[width=16cm, height=16cm]{sequence3.png}
    \caption{Diagramme de séquence pour le cas d’utilisation  Notice et diffusion .}
\end{figure}

\newpage

\textbf{\underline{Alerter sur un véhicule recherché :}}  
Le diagramme de séquence du scénario  Alerter sur un véhicule recherché  met en évidence la manière dont le système traite une alerte lorsqu’un utilisateur ou une autorité signale un véhicule suspect.  
\begin{itemize}
    \item L’utilisateur envoie une alerte avec des informations
    \item Le système compare les données reçues avec la base des véhicules déclarés volés.
    \item Si une correspondance est trouvée, une notification est envoyée aux polices et aux administrateurs.
    % \item L’administrateur peut ensuite initier des actions comme la confirmation de l’alerte ou la communication avec les forces de l’ordre.
\end{itemize}

\begin{figure}[h]
    \centering
    \includegraphics[width=16cm, height=15cm]{sequence2.png}
    \caption{Diagramme de séquence pour le cas d’utilisation  Alerter sur un véhicule recherché.}
\end{figure}


\newpage


\section{Choix Techniques}

Le développement du système de gestion et de déclaration de cambriolages de véhicules repose sur des choix technologiques adaptés aux besoins du projet. Ces choix sont guidés par des critères tels que la performance, la sécurité, la scalabilité et la maintenabilité.

% \subsection{ \gls{Langages} et  \gls{Frameworks} }
\subsection{\texorpdfstring{\gls{Langages} et \gls{Frameworks}}{Langages et Frameworks}}


% \subsubsection{\gls{Front-End}: ReactJS}
\subsubsection{\texorpdfstring{\gls{Front-End}: ReactJS}{Front-End: ReactJS}}

Pour l’interface utilisateur, nous avons choisi \textbf{ReactJS} \cite{reactjs}, une bibliothèque JavaScript permettant la création d’interfaces dynamiques et réactives.  
Ses principaux atouts sont : la réutilisabilité des composants, la réduction de la redondance du code et l’optimisation des performances grâce au \gls{DOM} virtuel.  
Ces caractéristiques améliorent l’expérience utilisateur et facilitent la maintenance.

\subsubsection{Back-End: NestJS avec GraphQL et Prisma}
Le \gls{Back-End} repose sur \textbf{NestJS} \cite{nestjs}, un framework modulaire basé sur \textbf{\gls{Node.js}} et \textbf{\gls{TypeScript}}, adapté aux applications évolutives.  
Nous avons adopté \textbf{GraphQL} \cite{graphql} avec l’approche \textbf{\gls{Code First}}, où le schéma est généré automatiquement à partir des classes TypeScript. Cela assure une forte cohérence entre le code et l’API.  

La gestion des données est assurée par \textbf{Prisma} \cite{prisma}, un \gls{ORM} moderne qui simplifie les opérations en base, génère automatiquement un client typé et facilite les migrations.  

\textbf{Avantages principaux :}  
\begin{itemize}
    \item Structure modulaire et maintenable avec NestJS
    \item Requêtes optimisées : GraphQL ne renvoie que les données nécessaires  
    \item Gestion simplifiée des données et typage strict grâce à Prisma
    \item Sécurité renforcée avec la validation et l’injection de dépendances  
\end{itemize}

\subsection{Base de Données}
Le projet utilise \textbf{PostgreSQL} \cite{postgresql}, un \gls{SGBDR} \gls{open-source} reconnu pour sa fiabilité et sa conformité aux standards SQL.  
Il offre une gestion avancée des transactions.

\subsection{Architecture du Système et Conteneurisation}
L’architecture adoptée est de type \textbf{\gls{client-serveur}} et conteneurisée avec \textbf{Docker} \cite{docker} :
\begin{itemize}
    \item \textbf{Front-end :} ReactJS communique avec le serveur via des requêtes et mutations GraphQL \cite{reactjs, graphql}.  
    \item \textbf{Back-end :} NestJS implémente l’API GraphQL (Code First) et interagit avec la base via Prisma \cite{nestjs, prisma}.  
    \item \textbf{Base de données :} PostgreSQL, assurant robustesse et intégrité des données \cite{postgresql}.  
    \item \textbf{Infrastructure :} Nginx agit comme \gls{reverse proxy} pour la gestion des requêtes HTTP/HTTPS et l’équilibrage de charge \cite{nginx}.  
    \item \textbf{Docker :} chaque composant (front-end, back-end, base de données) est isolé dans des conteneurs, garantissant portabilité, déploiement simplifié et cohérence entre les environnements \cite{docker}.
\end{itemize}

Cette architecture modulaire et conteneurisée garantit la scalabilité, la portabilité et une maintenance simplifiée.

\subsection{Sécurité}
La sécurité est un aspect central du projet, renforcé par plusieurs couches de protection :  
\begin{itemize}
    \item \textbf{Authentification sécurisée} avec \gls{JSON} Web Tokens (JWT) et expiration automatique des sessions
    \item \textbf{Chiffrement des communications} via \gls{HTTPS} et \gls{TLS}
    \item \textbf{Contrôle d’accès} basé sur les rôles et permissions granulaires
    \item \textbf{Hashage des mots de passe} avec bcrypt \cite{bcrypt} et salage renforcé
    \item \textbf{Sécurité des conteneurs Docker} : isolation des services, limitation des privilèges et mise à jour régulière des images
    \item \textbf{Protection contre les injections et attaques} : validation stricte des entrées utilisateur, prévention des injections \gls{SQL} et \gls{XSS}
    \item \textbf{Journalisation et audit} : traçabilité de toutes les actions sensibles pour détecter les anomalies
\end{itemize}

Ces mesures assurent la confidentialité, l’intégrité et la disponibilité des données.


\subsection{Avantages des Choix Techniques}
\begin{itemize}
    \item Interfaces dynamiques et performantes grâce à ReactJS
    \item \gls{API} robuste et typée avec NestJS, \gls{GraphQL} et Prisma
    \item Données fiables et extensibles via PostgreSQL
    \item Sécurité renforcée par JWT, HTTPS, \gls{bcrypt} et bonnes pratiques Docker
    \item Architecture modulaire, scalable et conteneurisée avec Docker et \gls{Nginx}
    \item Documentation claire grâce à UML
\end{itemize}

\subsection{Limites}
\begin{itemize}
    \item \textbf{Complexité technique} nécessitant une expertise en JavaScript/TypeScript, NestJS et Prisma  
    \item \textbf{Courbe d’apprentissage élevée} pour GraphQL, Docker et l’architecture sécurisée  
    \item \textbf{Consommation de ressources} plus importante avec PostgreSQL et Docker que des solutions légères
\end{itemize}


\section{Conclusion}
L'utilisation combinée des technologies modernes telles que \gls{ReactJS}, \gls{NestJS}, \gls{PostgreSQL}, \gls{Prisma} et \gls{Docker}, ainsi que des bonnes pratiques de sécurité, nous permet de concevoir un système robuste et performant pour la gestion des cambriolages de véhicules. Ce système facilitera la collaboration entre les utilisateurs, les administrateurs et la police, tout en garantissant la sécurité, la fiabilité et la portabilité des données traitées.






































































































































% \section{Conclusion}
% L’utilisation combinée des technologies modernes telles que \gls{react}, \gls{nestjs}, \gls{postgresql}, \gls{prisma} et \gls{docker}, associée à l’adoption de bonnes pratiques de sécurité, a permis de concevoir un système robuste et performant dédié à la gestion des cambriolages de véhicules.  
% Ce système favorise la collaboration entre les utilisateurs, les administrateurs et les forces de l’ordre, tout en garantissant la sécurité, la fiabilité et la portabilité des données traitées.

% Le frontend est développé avec \gls{react}.
% Le backend utilise \gls{nestjs} et \gls{postgresql}.
% Le déploiement est réalisé avec \gls{docker}.




















% \section{Introduction}
% Dans cette section, nous allons aborder la modélisation et la conception du système en prenant en compte les différentes méthodologies et outils de modélisation disponibles. Nous comparerons deux approches populaires, à savoir UML et MERISE, et justifierons le choix de la méthode la plus adaptée pour le projet. Ensuite, nous nous pencherons sur la mise en place de la modélisation UML, un standard largement utilisé dans la conception de systèmes informatiques.


% \section{Comparaison entre UML et MERISE}
% UML (Unified Modeling Language) et MERISE sont deux méthodologies populaires utilisées pour la modélisation des systèmes d’information. Cependant, elles diffèrent dans leur approche et leur utilisation.

% \subsection*{UML}
% UML est une méthode de modélisation orientée objet qui permet de représenter visuellement des systèmes complexes. Il est principalement utilisé pour modéliser les logiciels à travers des diagrammes qui illustrent les aspects statiques et dynamiques du système. UML se compose de plusieurs types de diagrammes, y compris les diagrammes de cas d’utilisation, les diagrammes de classes, et les diagrammes de séquences.

% \subsection*{MERISE}
% MERISE, quant à lui, est une méthode de modélisation orientée processus et données. Elle est principalement utilisée dans le cadre de la conception de bases de données et de systèmes d'information en général. Elle distingue trois niveaux :stratégique, conceptuel et logique. MERISE met l'accent sur la structuration des données et leur organisation dans le cadre de processus métiers.

% \subsection*{Comparaison}
% UML est plus flexible et centré sur les objets, ce qui est un avantage lorsqu'il s'agit de modéliser des systèmes informatiques complexes. MERISE, bien qu'il soit un peu plus ancien, est souvent préféré dans des contextes où l'organisation des données est primordiale. Le choix entre les deux méthodologies dépendra du type de projet et de la nature des exigences du système à modéliser.

% \section{ Choix de la méthode de modélisation}
% Après avoir comparé les deux méthodologies de modélisation, nous avons opté pour l'utilisation de la méthode UML pour ce projet. Cela est dû à sa capacité à modéliser des systèmes orientés objet et à sa popularité dans le domaine du développement logiciel moderne. UML permet de créer des diagrammes de classes, des diagrammes de séquences, et des cas d’utilisation qui sont essentiels pour une bonne compréhension du système dans sa globalité.

% \section{ Modélisation UML}
% \subsection{ Identification des acteurs du système}
% L’identification des acteurs du système est une étape clé dans la modélisation d’un système basé sur UML.Un acteur représente un rôle joué par un utilisateur ou un autre système qui interagit avec le système à modéliser. Dans notre système, les acteurs principaux sont les utilisateurs, les administrateurs et la police. Ces acteurs interagiront avec le système pour déclarer un vol, signaler un vehicule retrouvé, accéder à des données ou effectuer des actions de sécurité.




% \section{Choix Techniques}

% Le développement du système de gestion et de déclaration de cambriolages de véhicules repose sur des choix technologiques adaptés aux besoins du projet. Ces choix sont guidés par des critères tels que la performance, la sécurité, la scalabilité et la maintenabilité.

% \subsection{Langages et Frameworks}

% \subsubsection{Front-End: ReactJS}
% Pour l’interface utilisateur, nous avons choisi \textbf{ReactJS}, une bibliothèque JavaScript permettant la création d’interfaces dynamiques et réactives.  
% Ses principaux atouts sont: la réutilisabilité des composants, la réduction de la redondance du code et l’optimisation des performances grâce au DOM virtuel.  
% Ces caractéristiques améliorent l’expérience utilisateur et facilitent la maintenance.

% \subsubsection{Back-End: NestJS avec GraphQL et Prisma}
% Le back-end repose sur \textbf{NestJS}, un framework modulaire basé sur \textbf{Node.js} et \textbf{TypeScript}, adapté aux applications évolutives.  
% Nous avons adopté \textbf{GraphQL} avec l’approche \textbf{Code First}, où le schéma est généré automatiquement à partir des classes TypeScript. Cela assure une forte cohérence entre le code et l’API (Application Programming Interface).  

% La gestion des données est assurée par \textbf{Prisma}, un ORM moderne qui simplifie les opérations en base, génère automatiquement un client typé et facilite les migrations.  

% \textbf{Avantages principaux:}  
% \begin{itemize}
%     \item Structure modulaire et maintenable avec NestJS
%     \item Requêtes optimisées: GraphQL ne renvoie que les données nécessaires  
%     \item Gestion simplifiée des données et typage strict grâce à Prisma
%     \item Sécurité renforcée avec la validation et l’injection de dépendances  
% \end{itemize}

% La combinaison NestJS–GraphQL–Prisma constitue une solution moderne et robuste pour la construction d’API performantes.

% \subsection{Base de Données}
% Le projet utilise \textbf{PostgreSQL}, un SGBDR open-source reconnu pour sa fiabilité et sa conformité aux standards SQL.  
% Il offre une gestion avancée des transactions et, avec l’extension PostGIS, permet l’intégration de fonctionnalités géospatiales, utiles pour localiser les véhicules volés.

% \subsection{Architecture du Système}
% L’architecture adoptée est de type \textbf{client-serveur}:
% \begin{itemize}
%     \item \textbf{Front-end:} ReactJS communique avec le serveur via des requêtes et mutations GraphQL.  
%     \item \textbf{Back-end:} NestJS implémente l’API GraphQL (Code First) et interagit avec la base via Prisma.  
%     \item \textbf{Base de données:} PostgreSQL, assurant robustesse et intégrité des données.  
%     \item \textbf{Infrastructure:} Nginx agit comme reverse proxy pour la gestion des requêtes HTTP/HTTPS et l’équilibrage de charge.  
% \end{itemize}

% Cette architecture modulaire garantit la scalabilité et une maintenance simplifiée.

% \subsection{Sécurité}
% La sécurité est un aspect central du projet, avec les mesures suivantes:  
% \begin{itemize}
%     \item \textbf{Authentification sécurisée} avec JSON Web Tokens (JWT)
%     \item \textbf{Chiffrement des communications} via HTTPS
%     \item \textbf{Contrôle d’accès} basé sur les rôles
%     \item \textbf{Hashage des mots de passe} avec bcrypt
% \end{itemize}
% Ces mécanismes assurent la confidentialité et l’intégrité des données.

% \subsection{Outils de Modélisation}
% La conception a été réalisée en \textbf{UML}, avec l’outil \textbf{StarUML} pour les diagrammes de cas d’utilisation, de classes et de séquences.  
% Cet outil facilite la documentation et la standardisation des modèles.

% \subsection{Avantages des Choix Techniques}
% \begin{itemize}
%     \item Interfaces dynamiques et performantes grâce à ReactJS
%     \item API robuste et typée avec NestJS, GraphQL et Prisma
%     \item Données fiables et extensibles via PostgreSQL
%     \item Sécurité renforcée par JWT, HTTPS et bcrypt
%     \item Architecture modulaire et scalable avec Nginx
%     \item Documentation claire grâce à UML
% \end{itemize}

% \subsection{Limites}
% \begin{itemize}
%     \item \textbf{Complexité technique} nécessitant une expertise en JavaScript/TypeScript, NestJS et Prisma.  
%     \item \textbf{Courbe d’apprentissage élevée} pour GraphQL et l’architecture sécurisée.  
%     \item \textbf{Consommation de ressources} plus importante avec PostgreSQL que des solutions légères comme SQLite.  
% \end{itemize}

% \section{Conclusion}
% L'utilisation combinée des méthodologies MERISE et UML, ainsi que des technologies modernes telles que ReactJS, NestJS, PostgreSQL, Prisma et Docker, nous permet de concevoir un système robuste et performant pour la gestion des cambriolages de véhicules. Ce système facilitera la collaboration entre les utilisateurs, les administrateurs et la police, tout en garantissant la sécurité et la fiabilité des données traitées.

% \section{Perspectives et Améliorations Futures}

% Les perspectives futures de la lutte contre le vol et le cambriolage de véhicules reposent sur plusieurs axes technologiques et organisationnels majeurs :

% \begin{itemize}
%     \item \textbf{Intégration de l’intelligence artificielle (IA)} : utilisation d’algorithmes d’apprentissage automatique pour analyser les données issues des caméras, capteurs et plateformes de signalement, afin de détecter automatiquement les comportements suspects et de prédire les zones à haut risque de vol.

%     \item \textbf{Systèmes de surveillance intelligents} : déploiement de caméras intelligentes capables d’effectuer une reconnaissance automatique des plaques d’immatriculation et d’identifier des scénarios de cambriolage en temps réel.

%     \item \textbf{Généralisation de l’Internet des Objets (IoT)} : intégration de capteurs connectés (détecteurs de mouvement, d’ouverture de portières, de bris de vitre) dans les véhicules pour une surveillance continue et l’envoi d’alertes instantanées aux propriétaires et aux forces de l’ordre.

%     \item \textbf{Plateformes hybrides public–privé} : renforcement de la collaboration entre les forces de sécurité, les compagnies d’assurance, les collectivités locales et les acteurs privés afin d’améliorer le partage et l’exploitation des informations.

%     \item \textbf{Interconnexion des bases de données} : mise en réseau des bases de données nationales et internationales pour faciliter la traçabilité des véhicules volés et accélérer les procédures de récupération.

%     \item \textbf{Utilisation de la blockchain} : sécurisation des données de signalement des vols grâce à des registres distribués garantissant l’intégrité, la traçabilité et la transparence des informations.

%     \item \textbf{Applications mobiles et systèmes collaboratifs} : développement de solutions participatives permettant aux citoyens de signaler rapidement les incidents et de contribuer activement à la prévention du cambriolage de véhicules.

% \end{itemize}

\chapter{Résultats et Discussion}\label{chap:3}
% \addcontentsline{toc}{section}{Introduction}
% \section*{Introduction}

% \section{-}
% blablabla

% \section{-}
% \begin{algorithm}
% 	\KwData{$x$}
% 	\KwResult{$r$}
% 	\Begin{
% 		\If{$x \neq 0$}{
% 			$ r \leftarrow 1/x$\;
% 		}
% 	}
	
% 	\caption{Inverse}\label{alg:Inverse}
% \end{algorithm}

% \addcontentsline{toc}{section}{Conclusion}
% \section*{Conclusion}





% \subsection{Introduction}
% Les cambriolages de véhicules sont des infractions qui nuisent à la sécurité publique, et leur gestion nécessite une approche organisée impliquant différentes parties prenantes. Cette gestion englobe les dispositifs de sécurité, la surveillance, mais aussi la coopération entre les citoyens, les forces de l'ordre, les autorités locales et des acteurs privés. Le rôle des différents utilisateurs de systèmes de gestion de sécurité, tels que les administrateurs, la police et les superadmins, est crucial pour assurer une surveillance effective et une réponse rapide aux incidents de cambriolage. Ce chapitre explore les résultats obtenus concernant l'intégration de ces rôles dans les stratégies de prévention et de réponse aux cambriolages de véhicules.

% \subsection{Présentation des mesures de gestion des cambriolages}

% \subsubsection{Mise en place de dispositifs de sécurité renforcés}
% Les dispositifs de sécurité pour les véhicules, comme les systèmes d'alarme, les GPS de géolocalisation, et les dispositifs de verrouillage intelligents, sont essentiels pour prévenir les cambriolages. Ces technologies sont intégrées dans un cadre de gestion qui implique plusieurs acteurs : les propriétaires de véhicules, les forces de l'ordre, ainsi que les administrateurs des systèmes de sécurité.

% Les utilisateurs finaux, comme les propriétaires de véhicules, peuvent s'abonner à ces systèmes de sécurité et les configurer pour obtenir des alertes en temps réel via leurs smartphones ou plateformes en ligne. Ils peuvent signaler toute activité suspecte, soit manuellement, soit automatiquement en cas de tentative de vol, ce qui déclenche une intervention rapide.

% \subsubsection{Surveillance vidéo et patrouilles de sécurité}
% La surveillance par caméras, combinée à des patrouilles de sécurité, est une méthode efficace pour prévenir les cambriolages. Les caméras installées dans des zones sensibles telles que les parkings, les rues commerçantes ou les résidences sont contrôlées par des administrateurs de sécurité. Ces administrateurs supervisent en temps réel les images et détectent toute activité suspecte.

% Les utilisateurs tels que les policiers, les agents de sécurité, ou les superadmins peuvent recevoir des alertes automatiques en cas de détection de comportement anormal, ce qui leur permet d'agir rapidement. Les policiers, en particulier, jouent un rôle crucial dans l'analyse des vidéos et l'identification des criminels, souvent en collaboration avec des systèmes de géolocalisation GPS pour retracer les véhicules volés.

% \subsubsection{Sensibilisation des citoyens à la prévention}
% La prévention des cambriolages passe aussi par une stratégie de sensibilisation. Les administrateurs d'un système de gestion de sécurité jouent un rôle clé dans la diffusion d’informations et de conseils de prévention à travers des plateformes en ligne, des applications mobiles ou des campagnes de sensibilisation locales.

% Les citoyens peuvent s'inscrire sur des plateformes sécurisées où ils reçoivent des conseils personnalisés concernant la sécurité de leurs véhicules. De plus, des alertes locales peuvent être envoyées pour signaler des augmentations des cambriolages dans certaines zones. Les utilisateurs finaux peuvent également signaler des comportements suspects via ces plateformes, renforçant ainsi la vigilance communautaire.

% \subsubsection{Collaboration entre la police, les autorités locales et les citoyens}
% L’implication des forces de l'ordre, des administrateurs et des citoyens dans un système de gestion partagé est essentielle. La police, en tant qu'utilisateur ayant un accès direct aux informations de surveillance et aux alertes générées par le système de sécurité, peut intervenir de manière proactive. Les agents de la police peuvent être formés pour utiliser ces outils afin de surveiller les zones sensibles et de répondre plus rapidement aux incidents.

% Les superadmins ont un rôle de supervision générale. Ils sont responsables de la gestion de l’ensemble du système de sécurité, de la gestion des utilisateurs, et de la mise en œuvre de protocoles de sécurité avancés. En cas de situation critique, les superadmins peuvent superviser les actions des policiers, des agents de sécurité et des administrateurs pour assurer une réponse coordonnée.

% Les administrateurs, quant à eux, ont un rôle plus opérationnel et de gestion quotidienne. Ils gèrent l’installation et la configuration des dispositifs de sécurité, supervisent les alertes et veillent à ce que les utilisateurs finaux (les citoyens) suivent les protocoles de sécurité recommandés.

% \subsection{Discussion}
% L’intégration des différents rôles d’utilisateurs dans la gestion des cambriolages de véhicules permet de créer un environnement sécurisé et réactif. Les superadmins et administrateurs jouent un rôle de coordination technique et stratégique, tandis que les policiers, en tant qu’utilisateurs avec un accès privilégié aux informations critiques, sont en première ligne pour répondre aux incidents.

% Les résultats montrent que la coopération entre les utilisateurs finaux, les forces de l'ordre et les administrateurs peut considérablement améliorer la gestion des cambriolages de véhicules. Les systèmes de surveillance intégrés, lorsqu’ils sont associés à des patrouilles physiques et à une participation active de la communauté, augmentent les chances de détection et d'intervention avant que le dommage ne soit trop important. La gestion proactive des zones sensibles et l’analyse des données recueillies (comme les informations GPS ou vidéo) permettent de cibler les interventions et de renforcer la sécurité dans les zones à haut risque.

% La répartition claire des rôles et responsabilités, avec des actions coordonnées entre la police, les administrateurs et les superadmins, constitue une approche efficace pour réduire les cambriolages. En outre, la mise en place d'un système de gestion flexible et évolutif permet d'adapter les stratégies de sécurité aux nouvelles tendances criminelles, tout en répondant aux besoins spécifiques des utilisateurs.















\section*{Introduction}
Cette structure complète permet d'organiser efficacement l'application dédiée à la gestion des cambriolages tout en offrant une expérience utilisateur fluide et intuitive. Chaque interface a un rôle précis et contribue à l'objectif global d'amélioration de la sécurité et à la communication entre citoyens et forces de l'ordre.


\section{Presentation de l'application}
Du nom AlerteCar, elle permet de renforcer la sécurité publique en facilitant la gestion des cambriolages et des signalements de vols. Elle offre des outils adaptés pour les citoyens, les policiers et les administrateurs, avec une interface claire et un accès rapide aux services essentiels. L’objectif est de promouvoir une communication efficace entre les citoyens et les forces de l'ordre.


\section{Page d'Accueil}

\begin{center}
    \includegraphics[width=0.8\textwidth]{3-partie/accueil1.png} 
    \includegraphics[width=0.8\textwidth]{3-partie/accueil2.png} 
\end{center}



\textbf{Description:} La page d'accueil est la porte d'entrée de l'application. Elle offre une présentation générale et permet un accès rapide aux principales sections. L'objectif est de fournir une navigation claire dès l'arrivée sur l'application.



\textbf{Éléments:}
\begin{itemize}
    \item \textbf{Logo de l'application:} En haut à gauche pour une identification immédiate.
    \item \textbf{Menu de navigation:} Inclut des liens vers les principales pages comme l'Inscription, la Connexion, les services, les conseils, et l'à propos.
    \item \textbf{Informations sur les fonctionnalités principales:} Présentation succincte des services proposés par l'application ( declaration d'un cambriolage ,recherche de cambriolages, signalement d'un vehicule volé, etc.).
    % \item \textbf{Liens vers des ressources utiles:} Accès à la FAQ, le contact pour assistance, les conditions d'utilisation, etc.
\end{itemize}

\section{Inscription Utilisateur}
\textbf{Description:} Ce formulaire permet aux citoyens de s'inscrire pour utiliser l'application, en accédant à ces fonctionnalités.

\newpage
\begin{center}
    \includegraphics[width=0.8\textwidth]{3-partie/inscription1.png} 
    \includegraphics[width=0.8\textwidth]{3-partie/inscription2.png} 
\end{center}

\textbf{Éléments:}
\begin{itemize}
    \item \textbf{Champs:}NPI, Nom, prénom,tel , adresse ,e-mail, mot de passe.
    \item \textbf{Bouton "S'inscrire":} Soumet l'inscription.
    \item \textbf{Se connecter:} Si l'utilisateur est déjà inscrit, il peut se connecter .
\end{itemize}

% \section{Inscription Policier}
% \textbf{Description:} Formulaire destiné aux agents de police pour accéder à des fonctionnalités spécifiques, comme la gestion des rapports de cambriolage.

% \textbf{Éléments:}
% \begin{itemize}
%     \item \textbf{Champs:} Nom, prénom, numéro de badge, adresse e-mail, mot de passe.
%     \item \textbf{Bouton "S'inscrire":} Soumet l'inscription.
%     \item \textbf{Vérification des autorisations:} Processus de validation de l'inscription d'un agent de police, avec vérification de son numéro de badge.
% \end{itemize}

\section{Page de Connexion}

\begin{center}
    \includegraphics[width=0.8\textwidth]{3-partie/connexion1.png} 
    \includegraphics[width=0.8\textwidth]{3-partie/connexion2.png} 
\end{center}

\textbf{Description:} Interface permettant aux utilisateurs (citoyens ou policiers) de se connecter à leur compte pour accéder aux fonctionnalités spécifiques.

\textbf{Éléments:}
\begin{itemize}
    \item \textbf{Champs:} Adresse e-mail et le mot de passe.
    \item \textbf{Bouton "Se connecter":} Valide la demande de connexion.
    \item \textbf{Lien vers la récupération de mot de passe:} En cas d'oubli du mot de passe, l'utilisateur peut le réinitialiser.
\end{itemize}

\section{Interface Accueil utilisateur}

\begin{center}
    \includegraphics[width=0.8\textwidth]{3-partie/useraccueil.png} 
     \includegraphics[width=0.8\textwidth]{3-partie/connexion2.png} 
\end{center}

\textbf{Description:} Page où les utilisateurs peuvent gérer leurs informations personnelles, y compris les paramètres de sécurité.

\textbf{Éléments:}
\begin{itemize}
    \item \textbf{Affichage des informations personnelles:} Nom,prénom,tel, adresse ,e-mail .
    \item \textbf{Options pour modifier les informations personnelles:} Permet de mettre à jour les informations et de changer le mot de passe.
\end{itemize}



\section{Interface Déclaration de Vol}

\begin{center}
    \includegraphics[width=0.8\textwidth]{3-partie/declarer1.png} 
    \includegraphics[width=0.8\textwidth]{3-partie/declarer2.png} 
    \includegraphics[width=0.8\textwidth]{3-partie/connexion2.png} 
\end{center}

% \textbf{Description:} Permet aux utilisateurs de signaler un vol ou un cambriolage qu’ils ont observé ou subi.

% \textbf{Éléments:}
% \begin{itemize}
%     \item \textbf{Formulaire de déclaration:} Champ pour la date et l'heure du vol, le lieu, la description des objets volés.
%     \item \textbf{Option pour ajouter une photo:} Permet de télécharger une photo de l'incident ou des objets volés.
% \end{itemize}


\begin{itemize}
    \item \textbf{Formulaire de déclaration de cambriolage} : Un formulaire simple où l'utilisateur peut entrer les détails du cambriolage, comme la date, le lieu.
    \item \textbf{Suivi des déclarations} : Affichage de l'état actuel des déclarations effectuées par l'utilisateur avec des indicateurs visuels de statut ("Voler", "Retrouver").
\end{itemize}


\section{Interface Signalements}

\begin{center}
    \includegraphics[width=0.8\textwidth]{3-partie/signaler.png} 
\end{center}


\textbf{Description:} Permet aux utilisateurs ou policiers de signaler un véhicule retrouvé, potentiellement lié à un vol.

\textbf{Éléments:}
\begin{itemize}
    \item \textbf{Formulaire de signalement:} Détails du véhicule (marque, modèle, numéro d'immatriculation).
    \item \textbf{Option pour ajouter une photo:} Permet d’ajouter une photo du véhicule retrouvé.
\end{itemize}


\section{Interface Statistiques}

\begin{center}
    \includegraphics[width=0.8\textwidth]{3-partie/statistique1.png} 
     \includegraphics[width=0.8\textwidth]{3-partie/statistique2.png} 
\end{center}

\textbf{Description :} 
Page permettant aux utilisateurs de consulter des statistiques détaillées sur les déclarations de vol de véhicules. 
Les informations présentées incluent notamment : 
\begin{itemize}
    \item le \textbf{nombre total de véhicules retrouvés},  
    \item le \textbf{nombre total de véhicules non retrouvés},  
    \item la répartition des vols par \textbf{zones géographiques} (endroits où les vols sont les plus fréquents),  
    \item l’évolution du nombre de vols par \textbf{année}, \textbf{mois}, \textbf{semaine} et \textbf{jour},  
    \item des \textbf{graphiques comparatifs} facilitant la visualisation des tendances et des pics de criminalité,  
    % \item des \textbf{indicateurs clés} pour identifier les périodes et zones à haut risque.  
\end{itemize}


\section{Interface Profil}

\begin{center}
    \includegraphics[width=0.8\textwidth]{3-partie/profil.png} 
     \includegraphics[width=0.8\textwidth]{3-partie/connexion2.png} 
\end{center}

\textbf{Description:} Page où les utilisateurs peuvent gérer leurs informations personnelles, y compris les paramètres de sécurité.

\textbf{Éléments:}
\begin{itemize}
    \item \textbf{Affichage des informations personnelles:} Nom,prénom,tel, adresse ,e-mail .
    \item \textbf{Options pour modifier les informations personnelles:} Permet de mettre à jour les informations et de changer le mot de passe.
\end{itemize}


\section{Interface Accueil Police}

\begin{center}
    \includegraphics[width=0.8\textwidth]{3-partie/policeaccueil.png} 
     \includegraphics[width=0.8\textwidth]{3-partie/connexion2.png} 
\end{center}

\textbf{Description :}  
Page d’accueil destinée aux agents de police, offrant une interface centralisée pour :  
\begin{itemize}
    \item le \textbf{suivi en temps réel} des déclarations de vol,  
    \item la réception d’\textbf{alertes instantanées} lors de nouveaux cambriolages signalés,  
    \item l’accès rapide aux \textbf{détails des véhicules déclarés} (statut : volé, retrouvé),  
    % \item la \textbf{cartographie géolocalisée} des zones de forte criminalité,  
    % \item la génération de \textbf{rapports synthétiques} pour faciliter la prise de décision et l’organisation des patrouilles.  
\end{itemize}

% \textbf{Éléments:}
% \begin{itemize}
%     \item \textbf{Affichage des informations personnelles:} Nom,prénom,tel, adresse ,e-mail .
%     \item \textbf{Options pour modifier les informations personnelles:} Permet de mettre à jour les informations et de changer le mot de passe.
% \end{itemize}






\section{Interface Administrateur}

\begin{center}
    \includegraphics[width=0.8\textwidth]{3-partie/admin.png} 
\end{center}

\textbf{Description:} Outil réservé aux administrateurs pour gérer les données et rapports relatifs aux cambriolages.

\textbf{Éléments:}
\begin{itemize}
    \item \textbf{Acueil:} Affiche le suivi de ces declarations sur les cambriolages puid des alertes au nouveaux cambriolages.
    \item \textbf{Gestion des rapports:} Permet d'ajouter puis de consulter la liste des rapports.
    \item \textbf{Gestion des patrouilles:} Permet d'ajouter puis de consulter la liste des patrouilles.
    \item \textbf{Gestion des utilisateurs:} Ajouter , modifie  ou supprimer des comptes utilisateurs .
    \item \textbf{Gestion des polices:} Ajouter , modifie  ou supprimer des comptes polices .
\end{itemize}

\section{Interface Gestion des  Patrouilles}

\begin{center}
    \includegraphics[width=0.8\textwidth]{3-partie/gestionpatrouille.png} 
\end{center}


\textbf{Description:} interface  dédié aux patrouilles policières pour suivre leurs missions et interventions.

\textbf{Éléments:}
\begin{itemize}
    \item \textbf{Liste des missions assignées:} Détail des missions que les patrouilles doivent effectuer.
    \item \textbf{Enregistrement des interventions:} Permet aux patrouilles de saisir des rapports en temps réel sur leurs interventions.
\end{itemize}


\section{Interface Gestion des  Rapports}

\begin{center}
    \includegraphics[width=0.8\textwidth]{3-partie/gestionrapport.png} 
\end{center}

\textbf{Description :}  
Cette interface est dédiée à la gestion des rapports générés suite aux déclarations de vols.  
Elle permet aux agents de police et aux administrateurs de :  
\begin{itemize}
    \item consulter l’ensemble des rapports enregistrés,  
    \item rechercher un rapport spécifique par numéro, date, ou plaque d’immatriculation,  
    \item filtrer les rapports par statut (\textit{volé, retrouvé}),  
    % \item ajouter de nouvelles informations ou preuves liées à une déclaration,  
    \item exporter les rapports sous forme de documents PDF pour archivage ou transmission.  
\end{itemize}




\section{Interface Gestion des utilisateurs}
% \begin{itemize}
    % \item \textbf{Gestion des utilisateurs} : Interface permettant  l'ajout et la suppression de comptes, ainsi que l'édition des permissions.
    % \item \textbf{Rapports et statistiques} : Outils pour générer des rapports détaillés sur les tendances des cambriolages (par zone géographique, période, type de véhicule volé, etc.).
    % \item \textbf{Gestion des paramètres du site} : Interface de configuration des alertes, notifications, et paramètres de sécurité du site.
    % \item \textbf{Analyse des données} : Outils de visualisation des données pour suivre les performances du site, la résolution des cas et d'autres métriques importantes.
% \end{itemize}

\textbf{Description :}  
Cette interface est dédiée à l’administration des utilisateurs de la plateforme.  
Elle permet à l’administrateur ou aux agents autorisés de gérer les comptes et les accès.  

\begin{itemize}
    \item afficher la liste complète des utilisateurs inscrits,  
    \item ajouter de nouveaux utilisateurs avec leurs informations personnelles ,  
    % \item modifier les informations d’un utilisateur existant (nom, email, rôle, statut),  
    \item supprimer ou désactiver un compte en cas d’abus ou d’inactivité,  
    % \item attribuer des rôles spécifiques (\textit{citoyen, police, administrateur}),  
    \item réinitialiser le mot de passe d’un utilisateur.  
\end{itemize}




\section{Interface Gestion des Polices}

\begin{center}
    \includegraphics[width=0.8\textwidth]{3-partie/gestionpolice.png} 
\end{center}


\textbf{Description :}  
Cette interface est conçue pour gérer les comptes et activités des agents de police au sein du système.  
Elle permet un meilleur suivi, une répartition des tâches efficace et un contrôle des accès.  

\begin{itemize}
    \item afficher la liste complète des agents de police enregistrés,  
    \item ajouter un nouvel agent avec ses informations personnelles et son matricule,  
    \item modifier ou mettre à jour les informations d’un agent (nom, email, affectation),  
    \item activer ou désactiver le compte d’un agent en fonction de son statut,  
    \item attribuer des rôles ou responsabilités spécifiques (\textit{patrouille, gestion des rapports, supervision}),  
    \item suivre l’historique des actions et interventions de chaque agent.  
\end{itemize}



\section{Interface pour SuperAdmin}

\begin{center}
    \includegraphics[width=0.8\textwidth]{3-partie/admin.png} 
\end{center}



\textbf{Description:} Interface réservée aux superadministrateurs qui ont tous les droits d’accès pour gérer le système global de l'application.

\textbf{Éléments:}
\begin{itemize}
    \item \textbf{Gestion complète du système:} Gérer les utilisateurs, les permissions et les configurations globales de l'application.
    \item \textbf{Historique des actions:} Suivi complet des actions administratives effectuées dans le système.
\end{itemize}


% \textbf{Description:} Interface spécifique pour les administrateurs afin de gérer les utilisateurs, rapports  de l'application.

% \textbf{Éléments:}
% \begin{itemize}
%     \item \textbf{Gestion des utilisateurs:} Ajouter, modifier, ou supprimer des utilisateurs ou policiers.
%     \item \textbf{Accès aux rapports:} Gestion complète des rapports sur les cambriolages.
%     \item \textbf{Statistiques:} Accès aux données globales de l'application.
% \end{itemize}

\section{Interface Gestion des Administrateurs}


\begin{center}
    \includegraphics[width=0.8\textwidth]{3-partie/gestionadmin.png} 
\end{center}


\textbf{Description:} Interface réservée aux superadministrateurs qui ont tous les droits d’accès pour gérer le système global de l'application.
elle permet de gérer les comptes des administrateurs du système à la gestion des rôles .  

\begin{itemize}
    \item afficher la liste de tous les administrateurs enregistrés,  
    \item ajouter un nouvel administrateur avec ses informations (npi,nom,prenom,adresse, email),  
    \item modifier ou mettre à jour les informations d’un administrateur,  
    \item activer ou désactiver le compte d’un administrateur,  
    \item suivre l’historique des actions administratives (création, suppression, mises à jour).  
\end{itemize}


% \textbf{Éléments:}
% \begin{itemize}
%     \item \textbf{Gestion complète du système:} Gérer les utilisateurs, les permissions et les configurations globales de l'application.
%     \item \textbf{Historique des actions:} Suivi complet des actions administratives effectuées dans le système.
% \end{itemize}



% \section{Interface Mot de Passe Oublié}

% \begin{center}
%     \includegraphics[width=0.8\textwidth]{forgotpassword.jpg} 
% \end{center}


% \textbf{Description:} Permet aux utilisateurs de récupérer leur mot de passe en cas d'oubli.

% \textbf{Éléments:}
% \begin{itemize}
%     \item \textbf{Champ pour l'adresse e-mail:} Permet de récupérer l'adresse associée au compte.
%     \item \textbf{Instructions pour la réinitialisation:} Guide pour aider l’utilisateur à réinitialiser son mot de passe.
% \end{itemize}

% \section{Changer de Mot de Passe}

% \begin{center}
%     \includegraphics[width=0.8\textwidth]{changepassword.jpg} 
% \end{center}

% \textbf{Description:} Permet aux utilisateurs et policiers de modifier leur mot de passe pour des raisons de sécurité.

% \textbf{Éléments:}
% \begin{itemize}
%     \item \textbf{Champs pour l'ancien et le nouveau mot de passe:} Demande l'ancien mot de passe ainsi que le nouveau.
%     \item \textbf{Bouton "Changer":} Soumet la demande de changement de mot de passe.
% \end{itemize}



\subsection{Conclusion}
La gestion des cambriolages de véhicules bénéficie grandement d'une approche collaborative et d'une gestion efficace des rôles des utilisateurs. La technologie, en particulier les systèmes de surveillance et de géolocalisation, associée à des stratégies de prévention et à une coopération étroite entre les forces de l'ordre, les citoyens et les administrateurs, constitue un moyen puissant de réduire ces infractions. L’implication des superadmins et des administrateurs garantit une supervision technique optimale, permettant ainsi une réponse rapide et ciblée aux cambriolages de véhicules. Grâce à cette coordination, il devient possible de minimiser les risques, de protéger les biens des citoyens et d’améliorer la sécurité dans les zones à haut risque.
















% \begin{abstract}
% Ce rapport présente une analyse complète des cambriolages de véhicules, incluant des statistiques globales, une répartition géographique, des analyses temporelles, et un suivi des enquêtes. L'objectif est de fournir une vue d'ensemble des incidents, d'identifier les tendances et d'aider à la prise de décision en matière de sécurité publique.
% \end{abstract}

% \newpage

% \section{Introduction}
% Les cambriolages de véhicules représentent une menace importante pour la sécurité des biens. Ce rapport présente une analyse détaillée de la situation actuelle des cambriolages de véhicules, avec des informations sur la répartition géographique, les tendances temporelles et l'efficacité des mesures de prévention.

% \section{Statistiques Globales}
% \subsection{Nombre Total de Cambriolages de Véhicules}
% \begin{itemize}
%     \item Nombre total de cambriolages : 3500 incidents.
%     \item Nombre de cambriolages résolus : 1200.
%     \item Nombre de cambriolages en attente : 2300.
%     \item Taux de résolution : 34.29\%.
% \end{itemize}

% \subsection{Valeur des Biens Volés}
% La valeur estimée des biens volés dans les véhicules cambriolés est de 5,000,000 €.

% \section{Répartition par Localisation}
% La carte suivante montre les zones géographiques les plus affectées par les cambriolages de véhicules.

% \begin{figure}[h!]
%     \centering
%     \includegraphics[width=\textwidth]{carte_cambriolages.png}
%     \caption{Répartition des cambriolages de véhicules par zone géographique}
% \end{figure}

% \subsection{Carte de Chaleur des Cambriolages}
% La carte de chaleur ci-dessous montre les zones où les cambriolages de véhicules sont les plus fréquents.

% % Insérer un graphique ici avec pgfplots ou inclure une carte d'une image
% \begin{figure}[h!]
%     \centering
%     \begin{tikzpicture}
%         \begin{axis}[
%             width=0.8\textwidth,
%             height=0.5\textwidth,
%             title={Carte de Chaleur des Cambriolages},
%             xlabel={Longitude},
%             ylabel={Latitude},
%             colorbar
%         ]
%         % Exemple de données fictives
%         \addplot[scatter,only marks,point meta=explicit] coordinates {
%             (48.8566, 2.3522) [0.5]
%             (48.8706, 2.3858) [1.2]
%             (48.8530, 2.3694) [2.4]
%             (48.8767, 2.3327) [1.0]
%         };
%         \end{axis}
%     \end{tikzpicture}
%     \caption{Carte de chaleur des cambriolages de véhicules}
% \end{figure}

% \section{Analyse Temporelle des Cambriolages}
% \subsection{Cambriolages par Mois}
% Le graphique suivant présente le nombre de cambriolages de véhicules par mois au cours de l'année écoulée.

% \begin{figure}[h!]
%     \centering
%     \begin{tikzpicture}
%         \begin{axis}[
%             width=\textwidth,
%             title={Nombre de Cambriolages par Mois},
%             xlabel={Mois},
%             ylabel={Nombre de Cambriolages},
%             ybar
%         ]
%         \addplot coordinates {
%             (1, 150) (2, 120) (3, 180) (4, 140) (5, 100) (6, 130) (7, 160) (8, 170) (9, 190) (10, 200) (11, 180) (12, 210)
%         };
%         \end{axis}
%     \end{tikzpicture}
%     \caption{Nombre de cambriolages de véhicules par mois}
% \end{figure}

% \subsection{Cambriolages par Heure de la Journée}
% Le graphique ci-dessous montre la répartition des cambriolages de véhicules selon l'heure de la journée.

% \begin{figure}[h!]
%     \centering
%     \includegraphics[width=\textwidth]{graph_cambriolages_heure.png}
%     \caption{Répartition des cambriolages par heure de la journée}
% \end{figure}

% \section{Suivi des Enquêtes}
% \subsection{Statut des Enquêtes}
% \begin{itemize}
%     \item Enquêtes ouvertes : 2300
%     \item Enquêtes fermées : 1200
%     \item Arrestations effectuées : 500
%     \item Nombre d'enquêtes résolues avec arrestation : 150
% \end{itemize}

% \subsection{Temps Moyen de Résolution}
% Le temps moyen pour résoudre une enquête sur un cambriolage de véhicule est de 45 jours.

% \section{Conclusion}
% Ce rapport fournit un aperçu détaillé des cambriolages de véhicules. Les informations recueillies aideront à la prise de décision concernant la mise en place de nouvelles stratégies de prévention, ainsi qu'à la gestion efficace des enquêtes en cours.















% \section*{Récapitulatif des Fonctionnalités par Rôle}
% Le tableau ci-dessous présente les fonctionnalités principales disponibles pour chaque type d'utilisateur dans un site de gestion des cambriolages de véhicules.

% \begin{table}[ht!]
% \centering
% \begin{tabular}{|>{\raggedright}p{4cm}|c|c|c|}
% \hline
% \textbf{Fonctionnalité} & \textbf{Utilisateur} & \textbf{Police} & \textbf{Administrateur} \\
% \hline
% \textbf{Déclaration de cambriolage} & \checkmark & & \\
% \hline
% \textbf{Suivi de l'état des déclarations} & \checkmark & \checkmark & \\
% \hline
% \textbf{Consultation des statistiques} & \checkmark & \checkmark & \checkmark \\
% \hline
% \textbf{Gestion des enquêtes} & & \checkmark & \checkmark \\
% \hline
% \textbf{Gestion des utilisateurs} & & & \checkmark \\
% \hline
% \textbf{Suivi des statistiques de crime} & & \checkmark & \checkmark \\
% \hline
% \textbf{Gestion des alertes} & & \checkmark & \checkmark \\
% \hline
% \textbf{Accès à la carte géographique} & \checkmark & \checkmark & \checkmark \\
% \hline
% \textbf{Gestion des paramètres du site} & & & \checkmark \\
% \hline
% \textbf{Analyse de données} & & \checkmark & \checkmark \\
% \hline
% \end{tabular}
% \caption{Fonctionnalités par Rôle}
% \end{table}








% \section*{Introduction}
% Ce document présente les maquettes pour le tableau de bord d'un site de gestion des cambriolages de véhicules, avec des descriptions pour chaque rôle d'utilisateur : Citoyen, Police et Administrateur.

% \section*{Maquette pour l'Utilisateur (Citoyen)}



% \textbf{Exemple de maquette pour l'utilisateur :}

% \begin{verbatim}
%  --------------------------------------------------------
% |                    Mon Tableau de Bord                |
% | ------------------------------------------------------|
% | 1. [Déclarer un Cambriolage] [Voir l'historique]      |
% | 2. Statut des Déclarations (En Cours / Résolu)        |
% | 3. [Alertes de Sécurité]                              |
% |    - Cambriolage dans ma zone                         |
% | 4. Carte Interactive                                  |
% |    - [Vue de la carte avec les points des cambriolages]|
%  --------------------------------------------------------
% \end{verbatim}

% \section*{Maquette pour la Police (Forces de l'ordre)}

% \begin{itemize}
%     \item \textbf{Gestion des déclarations} : Un tableau de bord avec une liste des déclarations reçues, leur statut, et des options pour marquer les déclarations comme "enquête ouverte", "clôturée", etc.
%     \item \textbf{Suivi des enquêtes} : Accès aux détails des enquêtes ouvertes, avec des outils pour ajouter des commentaires, joindre des rapports et des pièces justificatives (preuves).
%     \item \textbf{Carte des cambriolages} : Une carte dynamique permettant à la police de visualiser la concentration des cambriolages dans différentes zones géographiques.
%     \item \textbf{Gestion des suspects et des arrestations} : Liste des suspects potentiels avec des informations détaillées sur les profils, les connexions avec d'autres enquêtes et un historique des actions entreprises.
% \end{itemize}

% \textbf{Exemple de maquette pour la police :}

% \begin{verbatim}
%  --------------------------------------------------------
% |                 Tableau de Bord de la Police           |
% | ------------------------------------------------------|
% | 1. [Gestion des Déclarations] [Suivi des Enquêtes]     |
% | 2. [Carte des Cambriolages]                           |
% |    - Carte avec zones de cambriolages fréquents        |
% | 3. [Liste des Suspects]                               |
% |    - Suspect 1: Détails & Actions                     |
% |    - Suspect 2: Détails & Actions                     |
%  --------------------------------------------------------
% \end{verbatim}

% \section*{Maquette pour l'Administrateur}


% \textbf{Exemple de maquette pour l'administrateur :}

% \begin{verbatim}
%  --------------------------------------------------------
% |                    Tableau de Bord Admin              |
% | ------------------------------------------------------|
% | 1. [Gestion des Utilisateurs] [Voir les Statistiques]  |
% | 2. [Gestion des Paramètres]                           |
% |    - Modifier les alertes et notifications            |
% | 3. [Analyse des Données]                              |
% |    - Graphiques sur les tendances des cambriolages    |
% | 4. Rapports de Sécurité                               |
% |    - Voir les rapports sur les zones à risque         |
%  --------------------------------------------------------
% \end{verbatim}

% \section*{Diagramme Global du Tableau de Bord}

% Voici une représentation schématique de la structure globale du tableau de bord pour le site de gestion des cambriolages de véhicules, avec les pages principales pour chaque rôle :

% \begin{verbatim}
%                            +--------------------------+
%                            |   Tableau de Bord Admin   |
%                            +--------------------------+
%                             /              |            \
%            +-------------------------+     |     +-------------------------+
%            |  Gestion des Utilisateurs |     |     |    Statistiques et Rapports |
%            +-------------------------+     |     +-------------------------+
%                     |                    |                   |
%       +-------------------------+    +-------------------------+   +--------------------------+
%       | Tableau de Bord Police   |    |  Tableau de Bord User    |   |  Paramètres du Site      |
%       +-------------------------+    +-------------------------+   +--------------------------+
%          /             |                |                     |              \
% +-----------------+  +---------------+  +---------------+  +-------------------------+
% | Gestion des     |  | Suivi des     |  | Déclaration   |  | Alertes de Sécurité      |
% | Déclarations    |  | Enquêtes      |  | Cambriolage   |  | et Notifications         |
% +-----------------+  +---------------+  +---------------+  +-------------------------+
% \end{verbatim}

% \section*{Conclusion}
% Les maquettes ci-dessus montrent les différentes pages et fonctionnalités du site de gestion des cambriolages de véhicules, réparties par rôle. Chaque page a été conçue pour répondre aux besoins spécifiques de chaque type d'utilisateur tout en offrant une interface claire et intuitive pour une gestion efficace des cambriolages.











% \section*{1. Header (En-tête)}

% \begin{itemize}
%     \item \textbf{Logo du site} : Sur la gauche, pour une identification facile.
%     \item \textbf{Barre de navigation} :
%     \begin{itemize}
%         \item Menu principal avec des liens vers :
%         \begin{itemize}
%             \item Page d'accueil
%             \item Signaler un cambriolage
%             \item Suivi des incidents
%             \item Statistiques et tendances
%             \item Conseils de prévention
%         \end{itemize}
%         \item \textbf{Connexion} : Accès pour les utilisateurs et administrateurs.
%         \item \textbf{Inscription} : Option pour les nouveaux utilisateurs.
%     \end{itemize}
% \end{itemize}

% \section*{2. Section principale : Introduction et appels à l'action}

% \begin{itemize}
%     \item \textbf{Slogan accrocheur} : Un message court qui explique l’objectif du site. Exemple : ``Protégez votre véhicule, signalez les cambriolages''.
%     \item \textbf{Options de signalement rapide} :
%     \begin{itemize}
%         \item Signaler un cambriolage : Un bouton large et bien visible permettant aux citoyens de signaler un vol de véhicule.
%         \item Voir les incidents récents : Un lien vers une carte interactive ou une liste des derniers cambriolages signalés.
%     \end{itemize}
% \end{itemize}

% \section*{3. Carte des incidents récents}

% \begin{itemize}
%     \item Affichage dynamique des incidents récents avec des épingles de localisation sur une carte.
%     \item \textbf{Filtres} : Permettre de filtrer par date, type de véhicule, ou zone géographique.
%     \item \textbf{Popup sur les incidents} : Cliquer sur une épingle affiche des détails sur le cambriolage (lieu, date, description).
% \end{itemize}

% \section*{4. Statistiques et tendances}

% \begin{itemize}
%     \item \textbf{Graphiques interactifs} : Diagrammes à barres ou courbes montrant les tendances des cambriolages (par mois, par type de véhicule, etc.).
%     \item \textbf{Carte des zones sensibles} : Affichage des zones où les cambriolages sont les plus fréquents.
% \end{itemize}

% \section*{5. Section des conseils de prévention}

% \begin{itemize}
%     \item \textbf{Prévention de cambriolage} : Des conseils pratiques pour éviter les vols (verrouillage des véhicules, installation de systèmes d'alarme, etc.).
%     \item \textbf{Vidéos ou guides} : Des vidéos explicatives sur la sécurisation des véhicules.
% \end{itemize}

% \section*{6. Footer (Pied de page)}

% \begin{itemize}
%     \item \textbf{Liens utiles} : Mentions légales, politique de confidentialité, et coordonnées.
%     \item \textbf{Réseaux sociaux} : Liens vers les comptes du site sur les réseaux sociaux.
% \end{itemize}

% \newpage

% \section*{Découpage détaillé des informations et fonctionnalités spécifiques pour chaque type d'utilisateur}

% \subsection*{1. Utilisateur citoyen / Victime}

% \begin{itemize}
%     \item \textbf{Accès} : Connexion via un formulaire de login.
%     \item \textbf{Dashboard utilisateur} :
%     \begin{itemize}
%         \item Mes incidents : Liste des cambriolages signalés par l'utilisateur, avec possibilité de suivre l'état de chaque incident (en cours, résolu, fermé).
%         \item Créer un signalement : Formulaire de signalement de cambriolage avec les informations suivantes :
%         \begin{itemize}
%             \item Date et heure du vol
%             \item Type et modèle de véhicule
%             \item Description du vol et des objets volés
%             \item Lieu du vol (géolocalisation ou adresse manuelle)
%             \item Photos des dégâts (facultatif mais recommandé)
%         \end{itemize}
%         \item Suivi de l’enquête : Un accès aux mises à jour données par la police concernant l'enquête.
%         \item Recevoir des alertes : Notifications en temps réel concernant les nouveaux cambriolages dans la zone géographique de l’utilisateur.
%         \item Conseils de sécurité personnalisés : Suggestions sur la manière d'éviter les vols.
%     \end{itemize}
    % \item \textbf{Interaction communautaire} :
    % \begin{itemize}
    %     \item Forum de discussion : Un espace pour discuter avec d'autres citoyens sur la sécurité des véhicules, partager des astuces, ou signaler des comportements suspects.
    %     \item Alertes communautaires : Option pour signaler un cambriolage ou une tentative suspecte dans leur quartier.
    % \end{itemize}
% \end{itemize}

% \subsection*{2. Forces de l'ordre (Police)}

% \begin{itemize}
%     \item \textbf{Accès} : Authentification via un login sécurisé avec un rôle spécifique.
%     \item \textbf{Dashboard policier} :
%     \begin{itemize}
%         \item Incidents en cours : Liste des cambriolages signalés avec leur statut d’enquête (en cours, résolu).
%         \item Investigation : Accès pour ajouter des notes, télécharger des preuves (photos, vidéos), et mettre à jour le statut de chaque enquête.
%         \item Suspects : Ajouter ou consulter des informations sur les suspects potentiels liés à des incidents.
%         \item Historique des enquêtes : Suivi des anciens incidents, en ajoutant des détails relatifs aux résultats de l'enquête.
%         \item Statistiques des cambriolages : Accès aux tendances des cambriolages dans leur secteur (nombre de cambriolages par mois, types de véhicules les plus volés, etc.).
%         \item Coordination avec les administrateurs : Partage d'informations avec les administrateurs ou autres forces de l'ordre.
%         \item Alertes en temps réel : Recevoir des notifications pour chaque nouveau cambriolage signalé, en particulier ceux dans leur zone d'action.
%     \end{itemize}
% \end{itemize}

% \subsection*{3. Administrateur du système}

% \begin{itemize}
%     \item \textbf{Accès} : Connexion avec des droits d'accès élevés pour gérer l'ensemble du système.
%     \item \textbf{Dashboard administrateur} :
%     \begin{itemize}
%         \item Gestion des utilisateurs : Créer, modifier ou supprimer des comptes utilisateurs (citoyens, policiers).
%         \item Modération des signalements : Examiner les signalements de cambriolages soumis par les citoyens pour s'assurer qu'ils sont appropriés.
%         \item Surveillance des enquêtes : Vérifier l'état des enquêtes et assurer leur progression en cas de retard ou de besoin d’assistance.
%         \item Statistiques avancées : Accéder à des rapports détaillés sur les cambriolages (par région, par période, par type de véhicule).
%         \item Gestion des rôles : Affecter des rôles et permissions aux utilisateurs (citoyen, policier, administrateur).
%         \item Gestion des alertes : Paramétrer les alertes pour les utilisateurs, définir les zones sensibles et autres notifications.
%     \end{itemize}
%     \item \textbf{Sécurité et gestion des données} :
%     \begin{itemize}
%         \item Protection des données utilisateurs : Assurer la conformité aux réglementations comme le RGPD (pour les utilisateurs européens).
%         \item Gestion des logs et des journaux d'activité : Suivre les actions effectuées par les utilisateurs pour garantir la sécurité du système.
%     \end{itemize}
% \end{itemize}












% \section*{1. Base de données des incidents de cambriolage}

% \begin{itemize}
%     \item \textbf{Date et heure de l'incident} : Enregistrement précis de la date et de l'heure.
%     \item \textbf{Lieu de l'incident} : Adresse ou géolocalisation de l'incident, incluant les coordonnées GPS.
%     \item \textbf{Type de véhicule} : Informations sur le type de véhicule (marque, modèle, couleur, immatriculation).
%     \item \textbf{Description du vol} : Détails sur les objets volés, les méthodes utilisées par les cambrioleurs, etc.
%     \item \textbf{Dégâts causés} : Éventuels dommages au véhicule (fenêtres cassées, serrure forcée, etc.).
%     \item \textbf{Victime/Propriétaire} : Informations sur le propriétaire du véhicule (nom, contact, etc.).
% \end{itemize}

% \section*{2. Signalement }

% \begin{itemize}
%     \item \textbf{Formulaire de signalement} : Permettre aux victimes de signaler le vol de manière simple et détaillée (avec téléchargement de photos des dégâts).
%     \item \textbf{Suivi des plaintes} : Un système pour suivre l’état de la plainte (en cours, résolu, etc.).
%     \item \textbf{Numéro de dossier} : Un identifiant unique pour chaque incident afin de faciliter le suivi.
% \end{itemize}

% \section*{3. Gestion des enquêtes}

% \begin{itemize}
%     \item \textbf{Historique des enquêtes} : Enregistrement des enquêtes menées par les forces de l’ordre ou d’autres organismes.
%     \item \textbf{Suivi des suspects} : Stockage des informations sur les suspects potentiels, leur description physique, leur mode opératoire, etc.
%     \item \textbf{Résultats des enquêtes} : Informations sur l’avancée des enquêtes et la résolution des cas.
% \end{itemize}

% \section*{4. Statistiques et rapports}

% \begin{itemize}
%     \item \textbf{Carte des incidents} : Afficher les cambriolages sur une carte interactive, permettant d’identifier les zones les plus touchées.
%     \item \textbf{Rapports de tendances} : Fournir des statistiques détaillées (nombre de cambriolages par mois, types de véhicules les plus ciblés, etc.).
%     \item \textbf{Alertes et notifications} : Un système d’alertes pour informer les propriétaires de véhicules des incidents dans leurs zones géographiques.
% \end{itemize}

% \section*{5. Gestion des utilisateurs}

% \begin{itemize}
%     \item \textbf{Comptes utilisateurs} : Création d’un espace pour les citoyens, les autorités, et les responsables de la gestion des cambriolages.
%     \item \textbf{Permissions d’accès} : Définir les niveaux d'accès (utilisateur, administrateur, force de l’ordre, etc.).
%     \item \textbf{Notifications par email/SMS} : Alerter les utilisateurs des nouvelles informations sur l'enquête ou les alertes de cambriolages.
% \end{itemize}

% \section*{6. Interaction avec les forces de l’ordre}

% \begin{itemize}
%     \item \textbf{Partage d’informations} : Un système sécurisé pour échanger des données avec les forces de l’ordre ou d'autres services concernés.
%     \item \textbf{État des enquêtes en temps réel} : Permettre aux autorités de mettre à jour l’état de chaque incident.
% \end{itemize}

% \section*{7. Sécurité et confidentialité des données}

% \begin{itemize}
%     \item \textbf{Protection des données} : Assurer la confidentialité des informations personnelles (en conformité avec la législation, comme le RGPD en Europe).
%     \item \textbf{Authentification sécurisée} : Mettre en place un système d’authentification fort (connexion avec un mot de passe, double authentification, etc.).
% \end{itemize}

% \section*{8. Sensibilisation et prévention}

% \begin{itemize}
%     \item \textbf{Conseils de prévention} : Offrir des conseils aux utilisateurs pour éviter les cambriolages de véhicules (par exemple, ne pas laisser d'objets de valeur à la vue).
%     \item \textbf{Alertes communautaires} : Créer une fonctionnalité permettant aux utilisateurs d'alerter leur communauté si un cambriolage est suspecté dans leur secteur.
% \end{itemize}

% \section*{9. Accessibilité mobile}

% \begin{itemize}
%     \item \textbf{Application mobile} : Fournir une version mobile pour permettre aux utilisateurs de signaler un vol et suivre l’avancement des enquêtes en temps réel.
% \end{itemize}

% \section*{10. Intégration avec les systèmes externes}

% \begin{itemize}
%     \item \textbf{Partenariat avec des assurances} : Intégrer des fonctionnalités pour permettre aux victimes de faire une déclaration à leur compagnie d’assurance directement via le site.
%     \item \textbf{Collaboration avec des plateformes de surveillance} : Si possible, lier la plateforme à des caméras de sécurité ou des systèmes de surveillance pour augmenter les chances de résoudre les enquêtes.
% \end{itemize}

% \section*{11. Interactivité avec la communauté}

% \begin{itemize}
%     \item \textbf{Forum ou discussions en ligne} : Permettre aux utilisateurs de discuter de leurs expériences et de partager des conseils de sécurité.
%     \item \textbf{Programmes de prévention communautaire} : Mettre en place des initiatives locales pour sensibiliser la communauté à la sécurité des véhicules.
% \end{itemize}



% \include{perspectives}
%%conclusion
\conclusion

La gestion des cambriolages de véhicules nécessite une approche proactive qui intègre à la fois la technologie et l'engagement des communautés. Le système d'alerte que nous proposons, fondé sur une application web, représente une solution innovante permettant une réactivité accrue face aux actes de vol. En permettant aux citoyens, aux forces de l'ordre et aux administrateurs d'interagir efficacement, ce système favorise un environnement plus sécurisé.

Les résultats obtenus lors des tests pilotes montrent une amélioration significative de la rapidité de réaction et de la couverture des alertes. Cependant, des recherches futures devraient se concentrer sur l'amélioration continue du système, notamment par l'intégration de nouvelles technologies, comme l'intelligence artificielle ou les dispositifs de surveillance plus avancés. De plus, l'adaptation du système aux besoins spécifiques des différentes régions et l'extension de sa portée géographique sont essentielles pour maximiser son efficacité.

Ainsi, bien que des progrès aient été réalisés, il est crucial de maintenir un processus d'innovation et d'adaptation pour faire face aux évolutions constantes des menaces liées à la sécurité des véhicules. Le système d'alerte communautaire offre une base solide pour une collaboration accrue entre les citoyens, les autorités locales et les technologistes, contribuant ainsi à un renforcement global de la sécurité publique.
\cite{ehrig2006graph}
% 
\lhead[]{} \rhead[]{} \chead[]{}

%%biblio
\addcontentsline{toc}{chapter}{Bibliographie}
\nocite{*}
\bibliographystyle{abbrv}
\bibliography{biblio}



%\include{annexe}

\newpage
\tableofcontents

\end{document}
