% \introduction
% % bla bla bla \gls{acro} puis \Gls{acroglo} et enfin \gls{glossaire}

% % \chapter{Introduction}

% \section*{Contexte}

% Le cambriolage de véhicule est un phénomène de plus en plus préoccupant à travers le monde. Chaque année, des milliers de véhicules sont volés, représentant non seulement une perte financière significative pour les propriétaires, mais aussi un facteur de détérioration de la confiance et de sécurité au sein des communautés. Les cambrioleurs ont développé des méthodes de plus en plus sophistiquées, rendant difficile la détection et la prévention des vols. En conséquence, ce type de crime constitue un enjeu majeur pour la sécurité publique, incitant à l'exploration de nouvelles solutions pour y faire face.

% \section*{Problématique}

% Face à la sophistication croissante des méthodes de vol et à l’augmentation des cambriolages de véhicules, la prévention et la détection des véhicules volés deviennent des tâches de plus en plus complexes. Le manque de systèmes d'alerte rapide, de coordination entre les citoyens et les autorités locales, ainsi que l'inefficacité de certaines technologies existantes, accentuent la vulnérabilité des véhicules, notamment dans les zones urbaines à forte densité. La question centrale de ce mémoire est donc la suivante : comment mettre en place un système d'alerte rapide et communautaire pour améliorer la gestion des cambriolages de véhicules et ainsi réduire les pertes et renforcer la sécurité publique?

% \section*{Justificatif}

% Les statistiques sur les cambriolages de véhicules montrent une tendance inquiétante qui nécessite une action immédiate.Bien que des systèmes de sécurité traditionnels, comme les alarmes et les dispositifs antivols, soient utilisés, ils n’offrent pas une solution complète. De plus, la technologie a évolué rapidement, offrant de nouvelles opportunités d’innovation, telles que la géolocalisation et les applications web et mobiles, qui pourraient permettre de répondre plus efficacement aux cambriolages de véhicules. Un tel système pourrait permettre une mobilisation rapide des citoyens et des autorités locales, créant ainsi une chaîne de prévention plus forte et plus réactive.

% \section*{Objectif}

% Ce mémoire a pour objectif de proposer une solution innovante à travers la création d’un système d'alerte communautaire destiné à améliorer la gestion des cambriolages de véhicules. Ce système serait fondé sur l’utilisation de technologies modernes, telles que les applications web et
% la géolocalisation  pour alerter rapidement toute la communauté dès qu’un vol est détecté. En analysant l'état actuel du phénomène, les technologies disponibles et les meilleures pratiques de sécurité, nous proposerons un modèle opérationnel pour déployer ce système dans les communautés, en tenant compte des défis locaux et des solutions possibles.

% \section*{Organisation du document}
% Ce mémoire est structuré de la manière suivante:

% \begin{itemize}
  
%     \item \textbf{Chapitre 1: Technologies et solutions existantes}: Ce chapitre explore les technologies actuellement disponibles pour la gestion des cambriolages de véhicules, y compris les systèmes de sécurité et les solutions de géolocalisation.
%     \item \textbf{Chapitre 2: Modélisation et conception UML}: Ce chapitre propose un modèle pour un système d'alerte communautaire innovant, basé sur les technologies actuelles. Il détaille la conception du système, ses fonctionnalités principales, son architecture, ainsi que son mode d’implémentation.
%     \item \textbf{Chapitre 3: Résultats et Discussion }: Ce chapitre présentent les résultats de ce système dans les communautés, les contraintes techniques et logistiques, ainsi que les perspectives d’amélioration du système.

    
% \end{itemize}






\introduction

\section*{Contexte}

Le cambriolage de véhicules constitue un phénomène de plus en plus préoccupant à travers le monde. Chaque année, des milliers de véhicules sont volés, entraînant d’importantes pertes financières pour les propriétaires et une dégradation du sentiment de sécurité au sein des communautés. Les malfaiteurs utilisent désormais des méthodes sophistiquées, rendant difficile la détection et la prévention. Ce type de crime représente donc un enjeu majeur de sécurité publique.

\section*{Problématique}

La complexité croissante des méthodes de vol, combinée au manque de systèmes d’alerte rapide et de coordination entre citoyens et autorités locales, accentue la vulnérabilité des véhicules. Les dispositifs traditionnels (alarmes, antivols) restent insuffisants. La problématique centrale de ce mémoire est la suivante :  
\textit{Comment concevoir un système d’alerte communautaire efficace, capable de renforcer la prévention et la gestion des cambriolages de véhicules ?}

\section*{Justification}

L’évolution des technologies, notamment la géolocalisation et les applications web/mobiles, offre aujourd’hui de nouvelles perspectives pour améliorer la sécurité. Un système collaboratif, reposant sur la mobilisation rapide des citoyens et des autorités, pourrait constituer une réponse efficace aux limites des solutions existantes.

\section*{Objectif}

L’objectif de ce mémoire est de concevoir un modèle opérationnel de système d’alerte communautaire basé sur les technologies modernes. Ce dispositif vise à :  
\begin{itemize}
    \item faciliter la déclaration et le signalement rapide des cambriolages,  
    \item renforcer la coopération entre citoyens et autorités,  
    \item proposer une solution adaptable aux réalités locales.  
\end{itemize}

\section*{Organisation du document}

Ce mémoire est structuré en trois chapitres principaux:  
\begin{itemize}
    \item \textbf{Chapitre 1: Technologies et solutions existantes} – Présentation des dispositifs actuels de gestion des cambriolages de véhicules, incluant les systèmes de sécurité et les solutions de géolocalisation.  
    \item \textbf{Chapitre 2: Modélisation et conception UML} – Proposition d’un modèle innovant de système d’alerte communautaire, détaillant l’architecture, les fonctionnalités et l’implémentation.  
    \item \textbf{Chapitre 3: Résultats et discussion} – Analyse des résultats, des contraintes techniques et organisationnelles, ainsi que des perspectives d’amélioration.  
\end{itemize}
