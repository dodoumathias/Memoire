
\introduction

\section*{Contexte et justification}

 Les moyens de déplacement sont des biens indispensables dans la vie de l’homme, facilitant les activités quotidiennes, professionnelles et sociales. Cependant, des individus mal intentionnés exploitent diverses méthodes, de plus en plus sophistiquées, pour cambrioler ces moyens de déplacements en particulier les vehicules, transformant ces biens essentiels en cibles privilégiées de la criminalité.

 Ce type de criminalité constitue aujourd’hui un phénomène préoccupant à travers le monde ou chaque année, des vehicules sont cambriolés, entraînant d’importantes pertes financières pour les propriétaires et une dégradation du sentiment de sécurité au sein des communautés. Malgré l’évolution des dispositifs de protection, la complexité croissante des techniques utilisées par les malfaiteurs rend la prévention et la détection de ces actes de plus en plus difficiles.

Dans ce contexte, la lutte contre le cambriolage de véhicules représente un enjeu majeur de sécurité publique. L’essor des technologies numériques offre de nouvelles opportunités pour concevoir des solutions innovantes. La mise en place d’un système collaboratif, favorisant la mobilisation rapide des citoyens et des forces de l'ordre, permettrait d’améliorer la sécurité des véhicules, d’optimiser la gestion des incidents et de soutenir la prise de décision à travers l’analyse statistique des données.

\section*{Problématique}

La complexité croissante des méthodes de cambriolage, combinée au manque de systèmes d’alerte rapide et de coordination entre citoyens et les forces de l'ordres, accentue la vulnérabilité des véhicules. Les dispositifs traditionnels (alarmes, antivols) restent insuffisants. La problématique centrale de ce mémoire est la suivante :  
\textit{Comment concevoir un système d’alerte communautaire efficace, capable de renforcer la prévention, la gestion et le suivi des cambriolages de véhicules ?}


\section*{Objectif}

L’objectif de ce travail est de concevoir un modèle opérationnel de système d’alerte communautaire basé sur les technologies modernes. Ce dispositif vise à :  
\begin{itemize}
    \item Faciliter la déclaration, le signalement  et le suivi des cambriolages;  
    \item Renforcer la coopération entre citoyens et les forces de l'ordre;  
    \item Mettre a la disposition des forces de l'ordre, les outils d'analyse et les données pour améliorer la réactivité et l'efficacité de l'intervention;
    \item Exploiter les données statistiques sur le cambriolage  de véhicules afin d’éclairer la prise de décision, d’anticiper les risques et d’optimiser les actions de prévention et d’intervention;
    \item Augmenter le taux de récupération des véhicules cambriolés;
    \item Renforcer la sécurité des véhicules et la confiance des citoyens.
\end{itemize}

\section*{Organisation du document}

Ce document est structuré en trois chapitres principaux:  
\begin{itemize}
    \item \textbf{Chapitre 1: Technologies et solutions existantes} \\
     Présentation des dispositifs actuels de gestion des cambriolages de véhicules, incluant les systèmes de sécurité et les solutions de géolocalisation.  
    \item \textbf{Chapitre 2: Modélisation et conception UML} \\
    Proposition d’un modèle innovant de système d’alerte communautaire, détaillant l’architecture, les fonctionnalités et l’implémentation.  
    \item \textbf{Chapitre 3: Résultats et discussion} \\
     Analyse des résultats, des contraintes techniques et organisationnelles, ainsi que des perspectives d’amélioration.  
\end{itemize}
