\resume
\selectlanguage{french}
\vspace*{-6cm}
\begin{abstract}

Ce mémoire traite de la gestion et du suivi  des cambriolages de véhicules au Bénin. Face à l’augmentation de ces actes, la protection des véhicules et des biens des citoyens nécessite la mise en place de solutions efficaces, rapides et accessibles. L’objectif principal de ce travail est de concevoir une plateforme numérique collaborative permettant d’améliorer la prévention, la détection et la prise en charge des vols de véhicules.

La plateforme proposée permet aux citoyens de déclarer un vol, de signaler des comportements suspects et de suivre l’évolution des incidents, tandis que les forces de l’ordre disposent d’outils centralisés pour analyser les informations et intervenir plus rapidement.

Pour atteindre cet objectif, une analyse des cas de cambriolages de véhicules et des solutions existantes a été réalisée. Les résultats obtenus montrent que la collaboration entre les citoyens et les forces de l’ordre, appuyée par un outil numérique adapté, permet d’améliorer la réactivité des interventions et d’augmenter les chances de récupération des véhicules.

Ce mémoire met ainsi en évidence l’importance d’une approche participative et technologique dans la lutte contre le cambriolage des véhicules et propose une solution simple et opérationnelle adaptée au contexte béninois.

\paragraph{}
\textbf{Mots clés}: Vol de véhicules, cambriolage, plateforme collaborative, forces de l’ordre.

\end{abstract}


\newpage
\thispagestyle{empty}
\selectlanguage{english}
\addcontentsline{toc}{chapter}{Abstract}
\begin{abstract}

This thesis addresses the management and monitoring of vehicle thefts and break-ins in Benin. With the increase in such incidents, protecting vehicles and citizens' property requires effective, rapid, and accessible solutions. The main objective of this work is to design a collaborative digital platform aimed at improving the prevention, detection, and handling of vehicle thefts.

The proposed platform allows citizens to report a theft, alert authorities to suspicious behavior, and track incidents in real time, while law enforcement agencies have centralized tools to analyze information and respond more efficiently.

To achieve this goal, an analysis of vehicle break-in cases and existing solutions was conducted. The results show that collaboration between citizens and law enforcement, supported by an appropriate digital tool, improves response times and increases the chances of recovering stolen vehicles.

This thesis highlights the importance of a participatory and technological approach in combating vehicle theft and proposes a simple and operational solution adapted to the Beninese context.

\paragraph{}
\textbf{Keywords}: Vehicle theft, burglary, collaborative platform, law enforcement.

\end{abstract}
